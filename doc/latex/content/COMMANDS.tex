% =============================================================================
% COMMANDS.tex – Entwickler-Referenz für Befehle und Diagnose
% Modulares Fragment (kein \documentclass, kein \begin{document})
% Stand: 2026-01-08 | Version: 2.5.2
% =============================================================================

\section{Befehlsreferenz}
\label{sec:commands}

Diese Referenz fasst alle wichtigen Befehle für Deployment, Build, Test und Diagnose zusammen. Wir beginnen mit einer Schnellreferenz und gehen dann ins Detail.

% -----------------------------------------------------------------------------
\subsection{Schnellreferenz}
\label{subsec:cmd-schnellreferenz}

\begin{table}[H]
  \centering
  \caption{Wichtigste Befehle auf einen Blick}
  \label{tab:cmd-schnellreferenz}
  \begin{tabularx}{\textwidth}{@{}l X@{}}
    \toprule
    \textbf{Aktion} & \textbf{Befehl} \\
    \midrule
    Server starten & \texttt{python server.py} \\
    Dashboard öffnen & \texttt{http://rover:8080/} \\
    Status abfragen & \texttt{curl http://rover:8080/status | jq} \\
    Health-Check & \texttt{curl http://rover:8080/health} \\
    Test-Play & \texttt{curl http://rover:8080/test/play/5} \\
    Serial-Monitor & \texttt{cat /dev/serial/by-id/usb-Espressif*} \\
    Firmware flashen & \texttt{pio run -t upload} \\
    Service Status & \texttt{sudo systemctl status selection-panel} \\
    Service Logs & \texttt{journalctl -u selection-panel -f} \\
    \bottomrule
  \end{tabularx}
\end{table}

% -----------------------------------------------------------------------------
\subsection{Server starten}
\label{subsec:cmd-server}

\begin{infobox}[URL-Hinweis]
  \texttt{rover} funktioniert auf allen Geräten (Mac, iPhone, iPad) dank mDNS. Alternative: IP-Adresse direkt verwenden (\texttt{http://192.168.1.24:8080/}).
\end{infobox}

\begin{lstlisting}[style=shell]
# Auf dem Pi
ssh rover
cd ~/selection-panel
source venv/bin/activate
python server.py
\end{lstlisting}

\textbf{Erwartete Ausgabe:}

\begin{lstlisting}[style=shell,numbers=none]
==================================================
Auswahlpanel Server v2.5.2 (PROTOTYPE)
==================================================
Medien: 10 erwartet (IDs: 001-010)
Serial: /dev/serial/by-id/usb-Espressif_USB_JTAG_serial_debug_unit_...
HTTP:   http://0.0.0.0:8080/
ESP32 lokale LED: aktiviert
==================================================
Serial verbunden
\end{lstlisting}

% -----------------------------------------------------------------------------
\subsection{Serial direkt testen}
\label{subsec:cmd-serial}

\subsubsection{Serial-Port finden}

\begin{lstlisting}[style=shell]
# Stabiler Pfad (empfohlen)
ls -la /dev/serial/by-id/usb-Espressif*

# Fallback
ls -la /dev/ttyACM*
\end{lstlisting}

\subsubsection{Empfangen (ohne Server)}

\begin{lstlisting}[style=shell]
# Port konfigurieren (by-id Pfad verwenden)
SERIAL_PORT=$(ls /dev/serial/by-id/usb-Espressif* 2>/dev/null | head -1)
stty -F $SERIAL_PORT 115200 raw -echo

# Serial-Monitor (Ctrl+C zum Beenden)
cat $SERIAL_PORT
\end{lstlisting}

\subsubsection{Befehle senden}

\begin{lstlisting}[style=shell]
# Verbindungstest
echo "PING" > $SERIAL_PORT
echo "STATUS" > $SERIAL_PORT
echo "VERSION" > $SERIAL_PORT
echo "HELP" > $SERIAL_PORT
\end{lstlisting}

\subsubsection{LED-Befehle (1-basiert!)}

\begin{lstlisting}[style=shell]
# Einzelne LED (one-hot)
echo "LEDSET 001" > $SERIAL_PORT   # LED 1 ein
echo "LEDSET 005" > $SERIAL_PORT   # LED 5 ein
echo "LEDSET 010" > $SERIAL_PORT   # LED 10 ein

# Additiv
echo "LEDON 001" > $SERIAL_PORT    # LED 1 ein (additiv)
echo "LEDON 002" > $SERIAL_PORT    # LED 2 ein (additiv)
echo "LEDOFF 001" > $SERIAL_PORT   # LED 1 aus

# Alle LEDs
echo "LEDALL" > $SERIAL_PORT       # Alle ein
echo "LEDCLR" > $SERIAL_PORT       # Alle aus

# LED-Test (Lauflicht)
echo "TEST" > $SERIAL_PORT
echo "STOP" > $SERIAL_PORT         # Test stoppen
\end{lstlisting}

\subsubsection{Diagnose-Befehle}

\begin{lstlisting}[style=shell]
# Status zeigt CURLED (aktuelle LED)
echo "STATUS" > $SERIAL_PORT
# Ausgabe:
# LEDS 0000100000
# CURLED 5          <-- Aktuelle LED (1-basiert)
# BTNS 0000000000
# HEAP 372756
# QOVFL 0
# MODE PROTOTYPE
\end{lstlisting}

\subsubsection{Screen (interaktiv)}

\begin{lstlisting}[style=shell]
screen $SERIAL_PORT 115200
\end{lstlisting}

\begin{tipbox}[Screen beenden]
  \texttt{Ctrl+A}, dann \texttt{K}, dann \texttt{Y}
\end{tipbox}

% -----------------------------------------------------------------------------
\subsection{HTTP-Endpoints}
\label{subsec:cmd-http}

\begin{lstlisting}[style=shell]
# Status (JSON)
curl http://rover:8080/status | jq

# Health-Check (200 = healthy, 503 = degraded)
curl -w "%{http_code}" http://rover:8080/health

# Tastendruck simulieren (1-basiert!)
curl http://rover:8080/test/play/1
curl http://rover:8080/test/play/5
curl http://rover:8080/test/play/10

# Wiedergabe stoppen
curl http://rover:8080/test/stop
\end{lstlisting}

\subsubsection{Status-Response (v2.5.2)}

\begin{lstlisting}[style=json,caption={Beispiel-Antwort von /status}]
{
  "version": "2.5.2",
  "mode": "prototype",
  "num_media": 10,
  "current_button": 5,
  "ws_clients": 1,
  "serial_connected": true,
  "serial_port": "/dev/serial/by-id/usb-Espressif_USB_JTAG_...",
  "media_missing": 0,
  "missing_files": [],
  "esp32_local_led": true
}
\end{lstlisting}

% -----------------------------------------------------------------------------
\subsection{Deployment (Mac → Pi)}
\label{subsec:cmd-deployment}

\subsubsection{Mit rsync}

\begin{lstlisting}[style=shell]
cd ~/selection-panel

rsync -avz --delete \
    --exclude='firmware' \
    --exclude='hardwaretest_firmware' \
    --exclude='venv' \
    --exclude='.git' \
    --exclude='__pycache__' \
    . pi@rover:/home/pi/selection-panel/
\end{lstlisting}

\begin{table}[H]
  \centering
  \caption{rsync-Flags}
  \label{tab:cmd-rsync}
  \begin{tabularx}{0.7\textwidth}{@{}l X@{}}
    \toprule
    \textbf{Flag} & \textbf{Bedeutung} \\
    \midrule
    \texttt{-a} & Archiv-Modus (Rechte, Zeiten erhalten) \\
    \texttt{-v} & Verbose (zeigt Dateien) \\
    \texttt{-z} & Komprimiert Übertragung \\
    \texttt{--delete} & Löscht Dateien auf Ziel, die lokal fehlen \\
    \bottomrule
  \end{tabularx}
\end{table}

\subsubsection{Mit Git (empfohlen)}

\begin{lstlisting}[style=shell]
# Auf dem Mac
git add -A && git commit -m "..." && git push

# Auf dem Pi
ssh rover "cd ~/selection-panel && git pull && sudo systemctl restart selection-panel"
\end{lstlisting}

% -----------------------------------------------------------------------------
\subsection{Server-Steuerung (systemd)}
\label{subsec:cmd-systemd}

\begin{lstlisting}[style=shell]
# Starten
sudo systemctl start selection-panel

# Stoppen
sudo systemctl stop selection-panel

# Neu starten
sudo systemctl restart selection-panel

# Status pruefen
sudo systemctl status selection-panel

# Autostart aktivieren/deaktivieren
sudo systemctl enable selection-panel
sudo systemctl disable selection-panel
\end{lstlisting}

\subsubsection{Logs (journalctl)}

\begin{lstlisting}[style=shell]
# Live-Logs
journalctl -u selection-panel -f

# Letzte 50 Zeilen
journalctl -u selection-panel -n 50

# Letzte Stunde
journalctl -u selection-panel --since "1 hour ago"

# Heute
journalctl -u selection-panel --since today
\end{lstlisting}

% -----------------------------------------------------------------------------
\subsection{Firmware flashen (Mac)}
\label{subsec:cmd-firmware}

\begin{lstlisting}[style=shell]
cd ~/selection-panel/firmware

# Kompilieren
pio run

# Flashen
pio run -t upload

# Serial-Monitor (PlatformIO)
pio device monitor

# Flash + Monitor
pio run -t upload -t monitor

# Clean Build
pio run -t clean
\end{lstlisting}

% -----------------------------------------------------------------------------
\subsection{Medien verwalten}
\label{subsec:cmd-medien}

\subsubsection{Prüfen (1-basiert: 001–010)}

\begin{lstlisting}[style=shell]
# Auf dem Pi
cd ~/selection-panel

# Medien auflisten
ls -la media/

# Medien-Check
for i in $(seq -w 1 10); do
    echo -n "0$i: "
    [ -f "media/0$i.jpg" ] && echo -n "JPG OK " || echo -n "JPG -- "
    [ -f "media/0$i.mp3" ] && echo -n "MP3 OK" || echo -n "MP3 --"
    echo
done

# Anzahl pruefen
ls media/*.jpg 2>/dev/null | wc -l
ls media/*.mp3 2>/dev/null | wc -l
\end{lstlisting}

\subsubsection{Generieren}

\begin{lstlisting}[style=shell]
# Test-Medien generieren (auf Mac)
./scripts/generate_test_media.sh 10    # Prototyp
./scripts/generate_test_media.sh 100   # Produktion
\end{lstlisting}

% -----------------------------------------------------------------------------
\subsection{Diagnose}
\label{subsec:cmd-diagnose}

\subsubsection{USB / Serial}

\begin{lstlisting}[style=shell]
# USB-Geraete
ls -la /dev/serial/by-id/usb-Espressif*
lsusb | grep -i espressif

# Wer nutzt den Serial-Port?
sudo fuser /dev/serial/by-id/usb-Espressif*

# Port freigeben (falls blockiert)
sudo systemctl stop selection-panel
sudo systemctl stop microros-agent.service  # falls vorhanden
\end{lstlisting}

\subsubsection{Netzwerk}

\begin{lstlisting}[style=shell]
# IP-Adressen
ip addr | grep 192.168
hostname -I

# Hostname
hostname
\end{lstlisting}

\subsubsection{Python}

\begin{lstlisting}[style=shell]
# Python-Version
python3 --version

# Pakete im venv
~/selection-panel/venv/bin/pip list

# Detailliert
~/selection-panel/venv/bin/pip freeze
\end{lstlisting}

\subsubsection{System-Info}

\begin{lstlisting}[style=shell]
# Pi Hardware-Modell
cat /proc/device-tree/model

# Pi OS Version
cat /etc/os-release

# Kernel
uname -r

# Speicherplatz
df -h /

# RAM
free -h
\end{lstlisting}

% -----------------------------------------------------------------------------
\subsection{Schnelltest (Komplettablauf)}
\label{subsec:cmd-schnelltest}

Der folgende Ablauf testet das gesamte System:

\begin{enumerate}
  \item \textbf{Server starten} (Terminal 1):
\begin{lstlisting}[style=shell,numbers=none]
ssh rover
cd ~/selection-panel && source venv/bin/activate && python server.py
\end{lstlisting}

  \item \textbf{Dashboard öffnen} (Mac):
\begin{lstlisting}[style=shell,numbers=none]
open http://rover:8080/
\end{lstlisting}

  \item \textbf{Sound aktivieren:} Button im Browser klicken → Medien werden vorgeladen

  \item \textbf{Alle 10 Taster durchdrücken:}
    \begin{itemize}
      \item Jeder Taster sollte Bild + Ton abspielen
      \item LED leuchtet sofort (< \SI{1}{\milli\second})
      \item Wiedergabe startet aus Cache (< \SI{50}{\milli\second})
    \end{itemize}

  \item \textbf{Status prüfen:}
\begin{lstlisting}[style=shell,numbers=none]
curl http://rover:8080/status | jq
\end{lstlisting}
\end{enumerate}

% -----------------------------------------------------------------------------
\subsection{Erwartete Ausgaben}
\label{subsec:cmd-ausgaben}

\subsubsection{Server-Log (erfolgreich)}

\begin{lstlisting}[style=shell,numbers=none]
Button 1 gedrueckt
GET /media/001.mp3 HTTP/1.1 206
Wiedergabe 1 beendet -> LEDs aus
\end{lstlisting}

\subsubsection{Dashboard Debug-Panel}

\begin{lstlisting}[style=shell,numbers=none]
Preloading 10 Medien...
Preload abgeschlossen: 10/10 OK (1823ms)
RX: {"type": "play", "id": 1}
Bild aus Cache: 001 (instant)
Audio aus Cache gestartet: 001 (12ms)
Audio beendet: 1
TX: {"type":"ended","id":1}
\end{lstlisting}

\subsubsection{Serial-Monitor}

\begin{lstlisting}[style=shell,numbers=none]
PRESS 001
RELEASE 001
\end{lstlisting}

\subsubsection{Status-Endpoint}

\begin{lstlisting}[style=json,numbers=none]
{
  "version": "2.5.2",
  "mode": "prototype",
  "num_media": 10,
  "current_button": null,
  "ws_clients": 1,
  "serial_connected": true,
  "esp32_local_led": true
}
\end{lstlisting}
