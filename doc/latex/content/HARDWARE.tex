% =============================================================================
% HARDWARE.tex – Selection Panel Hardware
% =============================================================================
% Einbindung: % =============================================================================
% HARDWARE.tex – Selection Panel Hardware
% =============================================================================
% Einbindung: % =============================================================================
% HARDWARE.tex – Selection Panel Hardware
% =============================================================================
% Einbindung: % =============================================================================
% HARDWARE.tex – Selection Panel Hardware
% =============================================================================
% Einbindung: \input{content/HARDWARE.tex}
% Version: 2.5.2 | Stand: 2026-01-11
% =============================================================================

\section{Hardware}
\label{sec:hardware}

Dieses Kapitel dokumentiert den 10-Button-Prototyp des Selection Panels. Die Schaltung
lässt sich auf 100 Taster skalieren, indem wir die Daisy-Chain der Schieberegister
verlängern.

% -----------------------------------------------------------------------------
\subsection{Komponentenübersicht}
\label{subsec:hw-komponenten}

\begin{table}[H]
\centering
\caption{Stückliste des 10-Button-Prototyps}
\label{tab:hw-komponenten}
\begin{tabularx}{\textwidth}{@{}l l r X@{}}
\toprule
\textbf{Komponente} & \textbf{Typ} & \textbf{Anzahl} & \textbf{Funktion} \\
\midrule
XIAO ESP32-S3       & Mikrocontroller         & 1  & Steuerlogik, USB-CDC \\
Raspberry Pi 5      & SBC + Netzteil + microSD & 1 & Server, Dashboard, Medien \\
CD4021B             & 8-Bit PISO Schieberegister & 2 & Taster einlesen \\
74HC595             & 8-Bit SIPO Schieberegister & 2 & LEDs ansteuern \\
Taster              & $6 \times 6$\,\si{\milli\meter} Tactile Switch & 10 & Benutzereingabe \\
LED                 & \SI{5}{\milli\meter}, versch.\ Farben & 10 & Statusanzeige \\
Widerstand          & \SIrange{330}{3000}{\ohm} & 10 & LED-Strombegrenzung \\
Widerstand          & \SI{10}{\kilo\ohm}       & 10 & Pull-up für Taster \\
Kondensator         & \SI{100}{\nano\farad} (Keramik) & 4 & Stützkondensatoren \\
\bottomrule
\end{tabularx}
\end{table}

% -----------------------------------------------------------------------------
\subsection{Pin-Belegung XIAO ESP32-S3}
\label{subsec:hw-pinout}

\begin{figure}[H]
\centering
\begin{lstlisting}[basicstyle=\ttfamily\footnotesize, frame=single, numbers=none]
                  +-------------+
                  |    USB-C    |
                  |             |
 LED_RCK    D0 ---|  1       14 |--- 5V
 BTN_PS     D1 ---|  2       13 |--- GND
 LED_OE     D2 ---|  3       12 |--- 3V3
            D3 ---|  4       11 |--- D10  LED_MOSI
            D4 ---|  5       10 |--- D9   BTN_MISO
            D5 ---|  6        9 |--- D8   SCK
            D6 ---|  7        8 |--- D7
                  |             |
                  +-------------+
\end{lstlisting}
\caption{Pin-Belegung des XIAO ESP32-S3}
\label{fig:xiao-pinout}
\end{figure}

\begin{table}[H]
\centering
\caption{GPIO-Zuordnung für SPI und Steuerung}
\label{tab:gpio-zuordnung}
\begin{tabularx}{\textwidth}{@{}l l l X@{}}
\toprule
\textbf{Pin} & \textbf{Signal} & \textbf{Funktion} & \textbf{Ziel} \\
\midrule
\pin{D0}  & \texttt{LED\_RCK}  & Latch (STCP)              & 74HC595 Pin 12 \\
\pin{D1}  & \texttt{BTN\_PS}   & Parallel/Serial Control   & CD4021B Pin 9 \\
\pin{D2}  & \texttt{LED\_OE}   & Output Enable (PWM, active-low) & 74HC595 Pin 13 \\
\pin{D8}  & \texttt{SCK}       & SPI Clock                 & Beide Chip-Typen \\
\pin{D9}  & \texttt{BTN\_MISO} & Daten von CD4021B         & CD4021B \#0 Pin 3 (Q8) \\
\pin{D10} & \texttt{LED\_MOSI} & Daten zu 74HC595          & 74HC595 \#0 Pin 14 (SER) \\
\bottomrule
\end{tabularx}
\end{table}

% -----------------------------------------------------------------------------
\subsection{Byte-Mapping (Firmware \texorpdfstring{$\leftrightarrow$}{<->} Hardware)}
\label{subsec:hw-byte-mapping}

Die Asymmetrie zwischen Ein- und Ausgabe ergibt sich aus der Hardware: Der CD4021B
schiebt sein erstes Sample (PI-8) als erstes Bit heraus, während der 74HC595 das
zuletzt empfangene Bit auf QA (Ausgang 0) legt.

\begin{table}[H]
\centering
\caption{Bit-Zuordnung der Schieberegister}
\label{tab:byte-mapping}
\begin{tabular}{@{}l l l@{}}
\toprule
\textbf{IC} & \textbf{Byte 0} & \textbf{Byte 1} \\
\midrule
74HC595 (LEDs)    & LED 1--8 = Bit 0--7 & LED 9--10 = Bit 0--1 \\
CD4021 (Taster)   & BTN 1--8 = Bit 7--0 & BTN 9--10 = Bit 7--6 \\
\bottomrule
\end{tabular}
\end{table}

\begin{infobox}[Firmware-Abstraktion]
Die Firmware verwendet in \filep{bitops.h} die Formel \lstinline[style=arduino]|btn_bit(id) = 7 - ((id - 1) % 8)|,
um diese Zuordnung zu abstrahieren. Damit arbeitet die Logik-Schicht einheitlich mit
Button-IDs von 1--100.
\end{infobox}

% -----------------------------------------------------------------------------
\subsection{Schaltplan-Übersicht}
\label{subsec:hw-schaltplan}

\begin{figure}[H]
\centering
\begin{lstlisting}[basicstyle=\ttfamily\scriptsize, frame=single, numbers=none]
                        +--------------------------------------------------+
                        |                  OUTPUT (LEDs)                   |
    +---------+         |                                                  |
    | Pi 5    |         |   +----------+  QH'     +----------+             |
    |         |  USB-C  |   | 74HC595  |--------->| 74HC595  |             |
    |         |<------->|   |   #0     |   SER    |   #1     |             |
    +---------+         |   |          |          |          |             |
                        |   | QA-QH    |          | QA-QB    |             |
        +---------+     |   |   |      |          |   |      |             |
        |  XIAO   |     |   | LED 1-8  |          | LED 9-10 |             |
        | ESP32S3 |     |   +----------+          +----------+             |
        |         |     +--------------------------------------------------+
        | D10 ----+----------> SER #0 (MOSI)
        | D8  ----+----------> SCK (beide)
        | D0  ----+----------> RCK (beide)
        | D2  ----+----------> OE (beide)
        |         |
        | D9  <---+---------- Q8 #0 (MISO)
        | D8  ----+----------> CLK (beide)
        | D1  ----+----------> P/S (beide)
        |         |     +--------------------------------------------------+
        +---------+     |                  INPUT (Taster)                  |
                        |                                                  |
                        |   +----------+          +----------+             |
                        |   | CD4021B  |<-- DS ---| CD4021B  |<-- DS -- 3V3|
                        |   |   #0     |    Q8    |   #1     |             |
                        |   |          |<---------|          |             |
                        |   | PI 8-1   |          | PI 8-7   |             |
                        |   |   |      |          |   |      |             |
                        |   | BTN 1-8  |          | BTN 9-10 |             |
                        |   +----------+          +----------+             |
                        |        |                                         |
                        |        v                                         |
                        |      Q8 --> D9 (MISO)                            |
                        +--------------------------------------------------+
\end{lstlisting}
\caption{Gesamtübersicht der Schaltung}
\label{fig:schaltplan-uebersicht}
\end{figure}

% -----------------------------------------------------------------------------
\subsection{74HC595 Detailschaltplan (LED-Ausgang)}
\label{subsec:hw-74hc595}

Der 74HC595 ist ein Serial-In/Parallel-Out Schieberegister mit Latch. Die Daten
werden seriell eingetaktet und erst beim Latch-Impuls (RCK) an die Ausgänge
übernommen.

\begin{figure}[H]
\centering
\begin{lstlisting}[basicstyle=\ttfamily\scriptsize, frame=single, numbers=none]
                     74HC595 (#0)                    74HC595 (#1)
                  +-------------+                 +-------------+
    LED 2  <------|  1 QB  VCC 16|--- 3V3 LED 10 <|  1 QB  VCC 16|--- 3V3
    LED 3  <------|  2 QC  QA  15|--> LED 1       |  2 QC  QA  15|--> LED 9
    LED 4  <------|  3 QD  SER 14|<-- D10 (MOSI)  |  3 QD  SER 14|<-- #0.QH'
    LED 5  <------|  4 QE  OE  13|<-- D2 (PWM)    |  4 QE  OE  13|<-- D2 (PWM)
    LED 6  <------|  5 QF  RCK 12|<-- D0 (RCK)    |  5 QF  RCK 12|<-- D0 (RCK)
    LED 7  <------|  6 QG  SCK 11|<-- D8 (SCK)    |  6 QG  SCK 11|<-- D8 (SCK)
    LED 8  <------|  7 QH  CLR 10|--- 3V3         |  7 QH  CLR 10|--- 3V3
          GND ----|  8 GND QH'  9|---> #1.SER     |  8 GND QH'  9|    (n.c.)
                  +------+------+                 +------+------+
                         |                               |
                        === C1                          === C2
                        100nF                           100nF
                         |                               |
                        GND                             GND
\end{lstlisting}
\caption{74HC595 Pinout und Daisy-Chain}
\label{fig:74hc595-detail}
\end{figure}

\begin{warnbox}[CLR-Pin nicht floaten!]
Der \pin{CLR}-Pin (Pin 10) muss auf \SI{3.3}{\volt} gelegt werden. Ein floatender
CLR-Pin führt zu sporadischem Löschen des Schieberegisters.
\end{warnbox}

\subsubsection{LED-Beschaltung (Active-High)}

\begin{figure}[H]
\centering
\begin{lstlisting}[basicstyle=\ttfamily\small, frame=single, numbers=none]
    Qx (74HC595)
     |
    +-+
    | | 330 - 3k Ohm
    +-+
     |
     v LED
     |
    GND
\end{lstlisting}
\caption{LED-Vorwiderstand-Beschaltung}
\label{fig:led-beschaltung}
\end{figure}

Die Stromberechnung für verschiedene Widerstandswerte bei einer LED-Flussspannung
von ca.\ \SI{2.0}{\volt}:

\begin{table}[H]
\centering
\caption{LED-Strom in Abhängigkeit vom Vorwiderstand}
\label{tab:led-strom}
\begin{tabular}{@{}r r l@{}}
\toprule
\textbf{Widerstand} & \textbf{Strom} & \textbf{Helligkeit} \\
\midrule
\SI{330}{\ohm}  & $\approx$\,\SI{4}{\milli\ampere}   & hell \\
\SI{1}{\kilo\ohm}   & $\approx$\,\SI{1.3}{\milli\ampere} & gedimmt \\
\SI{3}{\kilo\ohm}   & $\approx$\,\SI{0.4}{\milli\ampere} & schwach \\
\bottomrule
\end{tabular}
\end{table}

% -----------------------------------------------------------------------------
\subsection{CD4021B Detailschaltplan (Taster-Eingang)}
\label{subsec:hw-cd4021b}

Der CD4021B ist ein Parallel-In/Serial-Out Schieberegister. Er liest acht
parallele Eingänge ein und gibt sie seriell aus.

\subsubsection{Pinout}

\begin{figure}[H]
\centering
\begin{lstlisting}[basicstyle=\ttfamily\small, frame=single, numbers=none]
              +------------+
    PI-8  ----|  1  *  16  |---- VDD
      Q6  ----|  2     15  |---- PI-7
      Q8  ----|  3     14  |---- PI-6
    PI-4  ----|  4     13  |---- PI-5
    PI-3  ----|  5     12  |---- Q7
    PI-2  ----|  6     11  |---- SERIAL IN (DS)
    PI-1  ----|  7     10  |---- CLOCK
     VSS  ----|  8      9  |---- P/S CONTROL
              +------------+
\end{lstlisting}
\caption{CD4021B Pinout (DIP-16)}
\label{fig:cd4021b-pinout}
\end{figure}

\subsubsection{Beschaltung für 10 Taster}

\begin{figure}[H]
\centering
\begin{lstlisting}[basicstyle=\ttfamily\scriptsize, frame=single, numbers=none]
                     CD4021B (#0)                        CD4021B (#1)
                  +--------------+                    +--------------+
    BTN 1  ------|  1 PI-8 VDD 16|--- 3V3   BTN 9 ---|  1 PI-8 VDD 16|--- 3V3
                 |  2 Q6  PI-7 15|<-- BTN 2          |  2 Q6  PI-7 15|<-- BTN 10
   D9 (MISO) <---|  3 Q8  PI-6 14|<-- BTN 3          |  3 Q8  PI-6 14|    +3V3
    BTN 5  ------|  4 PI-4 PI-5 13|<-- BTN 4          |  4 PI-4 PI-5 13|    +3V3
    BTN 6  ------|  5 PI-3  Q7  12|    (n.c.)         |  5 PI-3  Q7  12|    (n.c.)
    BTN 7  ------|  6 PI-2  DS  11|<-- Q8 von #1      |  6 PI-2  DS  11|--- 3V3
    BTN 8  ------|  7 PI-1 CLK  10|<-- D8 (SCK)       |  7 PI-1 CLK  10|<-- D8 (SCK)
          GND ---|  8 VSS  P/S   9|<-- D1 (PS)        |  8 VSS  P/S   9|<-- D1 (PS)
                 +-------+-------+                   +-------+-------+
                         |       <----- Q8 (Pin 3) ------+
                        === C3                          === C4
                        100nF                           100nF
                         |                               |
                        GND                             GND
\end{lstlisting}
\caption{CD4021B Beschaltung mit Daisy-Chain}
\label{fig:cd4021b-beschaltung}
\end{figure}

\subsubsection{Pin-Zuordnung Taster → CD4021B}

Der CD4021B gibt Daten \highlight{MSB-first} aus: PI-8 erscheint zuerst (Bit 7),
PI-1 zuletzt (Bit 0).

\begin{table}[H]
\centering
\caption{Taster-zu-Bit-Zuordnung}
\label{tab:btn-mapping}
\begin{tabular}{@{}c c c c c@{}}
\toprule
\textbf{Taster} & \textbf{Chip} & \textbf{Pin-Name} & \textbf{Pin-Nr} & \textbf{Bit im Byte} \\
\midrule
BTN 1  & \#0 & PI-8 &  1 & Bit 7 \\
BTN 2  & \#0 & PI-7 & 15 & Bit 6 \\
BTN 3  & \#0 & PI-6 & 14 & Bit 5 \\
BTN 4  & \#0 & PI-5 & 13 & Bit 4 \\
BTN 5  & \#0 & PI-4 &  4 & Bit 3 \\
BTN 6  & \#0 & PI-3 &  5 & Bit 2 \\
BTN 7  & \#0 & PI-2 &  6 & Bit 1 \\
BTN 8  & \#0 & PI-1 &  7 & Bit 0 \\
\midrule
BTN 9  & \#1 & PI-8 &  1 & Bit 7 (Byte 1) \\
BTN 10 & \#1 & PI-7 & 15 & Bit 6 (Byte 1) \\
\bottomrule
\end{tabular}
\end{table}

\subsubsection{Daisy-Chain Datenfluss}

\begin{figure}[H]
\centering
\begin{lstlisting}[basicstyle=\ttfamily\small, frame=single, numbers=none]
3V3 --> DS [CD4021B #1] --> Q8 --> DS [CD4021B #0] --> Q8 --> D9 (MISO) --> ESP32
              |                          |
              |  BTN 9-10                |  BTN 1-8
              |  (Byte 1)                |  (Byte 0)
              |                          |
              +-- Spaeter gelesen -------+-- Zuerst gelesen -->
\end{lstlisting}
\caption{Datenfluss in der CD4021B Daisy-Chain}
\label{fig:daisy-chain-flow}
\end{figure}

\begin{warnbox}[DS-Pin des letzten Chips]
Der \pin{DS}-Pin (Pin 11) des \textbf{letzten} CD4021B muss auf \SI{+3.3}{\volt}
gelegt werden. Bei GND würden Nullen nachgeschoben, die als \enquote{gedrückt}
fehlinterpretiert werden. Ungenutzte PI-x Pins ebenfalls auf \SI{+3.3}{\volt}
legen -- interne Pull-ups reichen nicht zuverlässig.
\end{warnbox}

\subsubsection{Taster-Beschaltung (Active-Low)}

\begin{figure}[H]
\centering
\begin{lstlisting}[basicstyle=\ttfamily\small, frame=single, numbers=none]
    3V3
     |
    +-+
    | | 10 kOhm (Pull-up)
    +-+
     +--------- PIx (CD4021B)
     |
    +-+
    | | Taster
    +-+
     |
    GND

Zustand:
  - Losgelassen: PIx = HIGH (1) -> in Firmware: nicht gedrueckt
  - Gedrueckt:   PIx = LOW  (0) -> in Firmware: gedrueckt
\end{lstlisting}
\caption{Taster mit Pull-up-Widerstand}
\label{fig:taster-beschaltung}
\end{figure}

% -----------------------------------------------------------------------------
\subsection{SPI-Bus Verkabelung}
\label{subsec:hw-spi}

\begin{table}[H]
\centering
\caption{SPI-Signalführung}
\label{tab:spi-verkabelung}
\begin{tabularx}{\textwidth}{@{}l l X@{}}
\toprule
\textbf{ESP32-S3} & \textbf{74HC595} & \textbf{CD4021B} \\
\midrule
\pin{D10} (MOSI) & SER (Pin 14, \#0) $\rightarrow$ QH' $\rightarrow$ SER (\#1) & -- \\
\pin{D8} (SCK)   & SCK (Pin 11, beide) & CLK (Pin 10, beide) \\
\pin{D9} (MISO)  & -- & Q8 (Pin 3, \#0) $\leftarrow$ DS $\leftarrow$ Q8 (\#1) \\
\pin{D0} (RCK)   & RCK (Pin 12, beide) & -- \\
\pin{D1} (PS)    & -- & P/S (Pin 9, beide) \\
\pin{D2} (OE)    & OE (Pin 13, beide) & -- \\
\bottomrule
\end{tabularx}
\end{table}

% -----------------------------------------------------------------------------
\subsection{Timing-Diagramme}
\label{subsec:hw-timing}

\subsubsection{CD4021B Lesevorgang}

\begin{figure}[H]
\centering
\begin{lstlisting}[basicstyle=\ttfamily\small, frame=single, numbers=none]
    P/S -----+     +-------------------------
             |     |
             +-----+
                   ^
                   Parallel Load (HIGH->LOW->HIGH)

    CLK -------------+  +--+  +--+  +--+  +--
                     |  |  |  |  |  |  |  |
                     +--+  +--+  +--+  +--+
                        ^     ^     ^     ^
                       Bit6  Bit5  Bit4  Bit3 ...

    Q8  ============##=====##=====##=====##====
                    || Bit7|| Bit6|| Bit5|| ...
                    ##=====##=====##=====##====
                    ^
                    First-Bit (PI-8) liegt VOR erstem Clock an!
\end{lstlisting}
\caption{Timing des CD4021B Lesevorgangs}
\label{fig:timing-cd4021b}
\end{figure}

\subsubsection{74HC595 Schreibvorgang}

\begin{figure}[H]
\centering
\begin{lstlisting}[basicstyle=\ttfamily\small, frame=single, numbers=none]
    SER ----##=====##=====##=====##=====------
            || Bit7|| Bit6|| Bit5|| ...
            ##=====##=====##=====##=====------

    SCK ---------+  +--+  +--+  +--+  +-----
                 |  |  |  |  |  |  |  |
                 +--+  +--+  +--+  +--+
                    ^     ^     ^     ^
                   Daten bei steigender Flanke uebernommen

    RCK ------------------------------------+  +--
                                            |  |
                                            +--+
                                               ^
                                              Latch: Shift -> Output
\end{lstlisting}
\caption{Timing des 74HC595 Schreibvorgangs}
\label{fig:timing-74hc595}
\end{figure}

% -----------------------------------------------------------------------------
\subsection{Stromlimits}
\label{subsec:hw-stromlimits}

\begin{table}[H]
\centering
\caption{Elektrische Grenzwerte der ICs}
\label{tab:stromlimits}
\begin{tabular}{@{}l l r l@{}}
\toprule
\textbf{Komponente} & \textbf{Parameter} & \textbf{Wert} & \textbf{Bemerkung} \\
\midrule
\textbf{ESP32-S3} & GPIO Output (max)     & \SI{40}{\milli\ampere}   & Drive Strength 3 \\
                  & GPIO Output (default) & \SI{20}{\milli\ampere}   & Drive Strength 2 \\
                  & Gesamt alle GPIOs     & \SI{1200}{\milli\ampere} & Summe \\
\midrule
\textbf{74HC595}  & Output pro Pin (max)      & $\pm$\SI{35}{\milli\ampere} & Absolutes Maximum \\
                  & Output pro Pin (empfohlen)& $\pm$\SI{6}{\milli\ampere}  & Dauerbetrieb \\
                  & VCC/GND gesamt            & \SIrange{70}{75}{\milli\ampere} & \critical{Package-Limit!} \\
\midrule
\textbf{CD4021B}  & Input (pro Pin)  & $<$\,\SI{1}{\micro\ampere} & CMOS-Eingang \\
                  & Output Q8        & \SIrange{1}{3}{\milli\ampere} & Für SPI ausreichend \\
\bottomrule
\end{tabular}
\end{table}

\subsubsection{Stromversorgung}

\begin{table}[H]
\centering
\caption{Stromaufnahme des Systems}
\label{tab:stromversorgung}
\begin{tabular}{@{}l r r@{}}
\toprule
\textbf{Komponente} & \textbf{Typisch} & \textbf{Maximum} \\
\midrule
ESP32-S3            & \SI{80}{\milli\ampere}  & \SI{500}{\milli\ampere} (WiFi aktiv) \\
CD4021B ($\times$2) & $<$\,\SI{1}{\milli\ampere} & \SI{1}{\milli\ampere} \\
74HC595 ($\times$2) & $<$\,\SI{1}{\milli\ampere} & \SI{70}{\milli\ampere} (alle Ausgänge) \\
LEDs ($\times$10 @ \SI{4}{\milli\ampere}) & \SI{40}{\milli\ampere} & \SI{40}{\milli\ampere} \\
\midrule
\textbf{Gesamt}     & \textbf{$\approx$\,\SI{130}{\milli\ampere}} & \textbf{$\approx$\,\SI{620}{\milli\ampere}} \\
\bottomrule
\end{tabular}
\end{table}

\begin{tipbox}[USB-Stromversorgung]
Die USB-CDC-Versorgung vom Pi liefert bis zu \SI{500}{\milli\ampere}. Bei mehr als
8 LEDs gleichzeitig oder höheren Strömen: Helligkeit per PWM reduzieren oder externe
\SI{5}{\volt}-Versorgung verwenden.
\end{tipbox}

% -----------------------------------------------------------------------------
\subsection{Hardware-Eigenheiten}
\label{subsec:hw-eigenheiten}

\subsubsection{First-Bit-Problem (CD4021B)}

Nach dem Parallel-Load liegt das erste Bit (PI-8 $\rightarrow$ Q8) sofort am Ausgang,
\textbf{bevor} der erste Clock kommt. Der ESP32 samplet aber erst \textbf{nach} der
ersten Clock-Flanke. Die Firmware löst dies durch einen \lstinline[style=arduino]|digitalRead()|
vor dem SPI-Transfer.

\subsubsection{DS-Pin des letzten CD4021B}

Der DS-Pin (Serial Data Input, \textbf{Pin 11}) des \textbf{letzten} CD4021B in der
Kette muss auf \SI{+3.3}{\volt} gelegt werden, nicht auf GND. Bei GND würden beim
Einlesen Nullen nachgeschoben, die als \enquote{gedrückt} fehlinterpretiert werden.

\subsubsection{SPI-Bus Crosstalk}

Wenn der ESP32 vom CD4021B liest, taktet er dabei Nullen durch den 74HC595. Die
Ausgänge ändern sich erst beim Latch-Impuls, aber ein glitchender RCK-Pin könnte
LEDs kurz ausschalten. Die Firmware kompensiert dies durch
\lstinline[style=arduino]|LED_REFRESH_EVERY_CYCLE = true|.

\subsubsection{Active-Low Taster}

Die Taster sind Active-Low beschaltet (gedrückt = 0, losgelassen = 1). Die Firmware
invertiert dies in \filep{bitops.h}, sodass die Logik-Schicht mit
\lstinline[style=arduino]|pressed = true| arbeitet.

\subsubsection{Stützkondensatoren}

Jeder IC benötigt einen \SI{100}{\nano\farad} Keramikkondensator zwischen VCC und
GND, möglichst nah am Chip. Ohne diese Kondensatoren können Störungen auf der
Versorgung zu Fehlfunktionen führen.

% -----------------------------------------------------------------------------
\subsection{Skalierung auf 100 Buttons}
\label{subsec:hw-skalierung}

Für das 100-Button-System verlängern wir die Daisy-Chain auf jeweils 13
Schieberegister:

\begin{table}[H]
\centering
\caption{Komponentenvergleich 10 vs.\ 100 Buttons}
\label{tab:skalierung-hardware}
\begin{tabular}{@{}l r r@{}}
\toprule
\textbf{Komponente} & \textbf{10-Button} & \textbf{100-Button} \\
\midrule
CD4021B       & 2   & 13 \\
74HC595       & 2   & 13 \\
Taster        & 10  & 100 \\
LEDs          & 10  & 100 \\
R (LED)       & 10  & 100 \\
R (Pull-up)   & 10  & 100 \\
C (\SI{100}{\nano\farad}) & 4 & 26 \\
\bottomrule
\end{tabular}
\end{table}

Die SPI-Transferzeit steigt von ca.\ \SI{20}{\micro\second} auf ca.\
\SI{260}{\micro\second} -- weit unter dem \SI{5}{\milli\second}-Zyklus der
Firmware.

% Version: 2.5.2 | Stand: 2026-01-11
% =============================================================================

\section{Hardware}
\label{sec:hardware}

Dieses Kapitel dokumentiert den 10-Button-Prototyp des Selection Panels. Die Schaltung
lässt sich auf 100 Taster skalieren, indem wir die Daisy-Chain der Schieberegister
verlängern.

% -----------------------------------------------------------------------------
\subsection{Komponentenübersicht}
\label{subsec:hw-komponenten}

\begin{table}[H]
\centering
\caption{Stückliste des 10-Button-Prototyps}
\label{tab:hw-komponenten}
\begin{tabularx}{\textwidth}{@{}l l r X@{}}
\toprule
\textbf{Komponente} & \textbf{Typ} & \textbf{Anzahl} & \textbf{Funktion} \\
\midrule
XIAO ESP32-S3       & Mikrocontroller         & 1  & Steuerlogik, USB-CDC \\
Raspberry Pi 5      & SBC + Netzteil + microSD & 1 & Server, Dashboard, Medien \\
CD4021B             & 8-Bit PISO Schieberegister & 2 & Taster einlesen \\
74HC595             & 8-Bit SIPO Schieberegister & 2 & LEDs ansteuern \\
Taster              & $6 \times 6$\,\si{\milli\meter} Tactile Switch & 10 & Benutzereingabe \\
LED                 & \SI{5}{\milli\meter}, versch.\ Farben & 10 & Statusanzeige \\
Widerstand          & \SIrange{330}{3000}{\ohm} & 10 & LED-Strombegrenzung \\
Widerstand          & \SI{10}{\kilo\ohm}       & 10 & Pull-up für Taster \\
Kondensator         & \SI{100}{\nano\farad} (Keramik) & 4 & Stützkondensatoren \\
\bottomrule
\end{tabularx}
\end{table}

% -----------------------------------------------------------------------------
\subsection{Pin-Belegung XIAO ESP32-S3}
\label{subsec:hw-pinout}

\begin{figure}[H]
\centering
\begin{lstlisting}[basicstyle=\ttfamily\footnotesize, frame=single, numbers=none]
                  +-------------+
                  |    USB-C    |
                  |             |
 LED_RCK    D0 ---|  1       14 |--- 5V
 BTN_PS     D1 ---|  2       13 |--- GND
 LED_OE     D2 ---|  3       12 |--- 3V3
            D3 ---|  4       11 |--- D10  LED_MOSI
            D4 ---|  5       10 |--- D9   BTN_MISO
            D5 ---|  6        9 |--- D8   SCK
            D6 ---|  7        8 |--- D7
                  |             |
                  +-------------+
\end{lstlisting}
\caption{Pin-Belegung des XIAO ESP32-S3}
\label{fig:xiao-pinout}
\end{figure}

\begin{table}[H]
\centering
\caption{GPIO-Zuordnung für SPI und Steuerung}
\label{tab:gpio-zuordnung}
\begin{tabularx}{\textwidth}{@{}l l l X@{}}
\toprule
\textbf{Pin} & \textbf{Signal} & \textbf{Funktion} & \textbf{Ziel} \\
\midrule
\pin{D0}  & \texttt{LED\_RCK}  & Latch (STCP)              & 74HC595 Pin 12 \\
\pin{D1}  & \texttt{BTN\_PS}   & Parallel/Serial Control   & CD4021B Pin 9 \\
\pin{D2}  & \texttt{LED\_OE}   & Output Enable (PWM, active-low) & 74HC595 Pin 13 \\
\pin{D8}  & \texttt{SCK}       & SPI Clock                 & Beide Chip-Typen \\
\pin{D9}  & \texttt{BTN\_MISO} & Daten von CD4021B         & CD4021B \#0 Pin 3 (Q8) \\
\pin{D10} & \texttt{LED\_MOSI} & Daten zu 74HC595          & 74HC595 \#0 Pin 14 (SER) \\
\bottomrule
\end{tabularx}
\end{table}

% -----------------------------------------------------------------------------
\subsection{Byte-Mapping (Firmware \texorpdfstring{$\leftrightarrow$}{<->} Hardware)}
\label{subsec:hw-byte-mapping}

Die Asymmetrie zwischen Ein- und Ausgabe ergibt sich aus der Hardware: Der CD4021B
schiebt sein erstes Sample (PI-8) als erstes Bit heraus, während der 74HC595 das
zuletzt empfangene Bit auf QA (Ausgang 0) legt.

\begin{table}[H]
\centering
\caption{Bit-Zuordnung der Schieberegister}
\label{tab:byte-mapping}
\begin{tabular}{@{}l l l@{}}
\toprule
\textbf{IC} & \textbf{Byte 0} & \textbf{Byte 1} \\
\midrule
74HC595 (LEDs)    & LED 1--8 = Bit 0--7 & LED 9--10 = Bit 0--1 \\
CD4021 (Taster)   & BTN 1--8 = Bit 7--0 & BTN 9--10 = Bit 7--6 \\
\bottomrule
\end{tabular}
\end{table}

\begin{infobox}[Firmware-Abstraktion]
Die Firmware verwendet in \filep{bitops.h} die Formel \lstinline[style=arduino]|btn_bit(id) = 7 - ((id - 1) % 8)|,
um diese Zuordnung zu abstrahieren. Damit arbeitet die Logik-Schicht einheitlich mit
Button-IDs von 1--100.
\end{infobox}

% -----------------------------------------------------------------------------
\subsection{Schaltplan-Übersicht}
\label{subsec:hw-schaltplan}

\begin{figure}[H]
\centering
\begin{lstlisting}[basicstyle=\ttfamily\scriptsize, frame=single, numbers=none]
                        +--------------------------------------------------+
                        |                  OUTPUT (LEDs)                   |
    +---------+         |                                                  |
    | Pi 5    |         |   +----------+  QH'     +----------+             |
    |         |  USB-C  |   | 74HC595  |--------->| 74HC595  |             |
    |         |<------->|   |   #0     |   SER    |   #1     |             |
    +---------+         |   |          |          |          |             |
                        |   | QA-QH    |          | QA-QB    |             |
        +---------+     |   |   |      |          |   |      |             |
        |  XIAO   |     |   | LED 1-8  |          | LED 9-10 |             |
        | ESP32S3 |     |   +----------+          +----------+             |
        |         |     +--------------------------------------------------+
        | D10 ----+----------> SER #0 (MOSI)
        | D8  ----+----------> SCK (beide)
        | D0  ----+----------> RCK (beide)
        | D2  ----+----------> OE (beide)
        |         |
        | D9  <---+---------- Q8 #0 (MISO)
        | D8  ----+----------> CLK (beide)
        | D1  ----+----------> P/S (beide)
        |         |     +--------------------------------------------------+
        +---------+     |                  INPUT (Taster)                  |
                        |                                                  |
                        |   +----------+          +----------+             |
                        |   | CD4021B  |<-- DS ---| CD4021B  |<-- DS -- 3V3|
                        |   |   #0     |    Q8    |   #1     |             |
                        |   |          |<---------|          |             |
                        |   | PI 8-1   |          | PI 8-7   |             |
                        |   |   |      |          |   |      |             |
                        |   | BTN 1-8  |          | BTN 9-10 |             |
                        |   +----------+          +----------+             |
                        |        |                                         |
                        |        v                                         |
                        |      Q8 --> D9 (MISO)                            |
                        +--------------------------------------------------+
\end{lstlisting}
\caption{Gesamtübersicht der Schaltung}
\label{fig:schaltplan-uebersicht}
\end{figure}

% -----------------------------------------------------------------------------
\subsection{74HC595 Detailschaltplan (LED-Ausgang)}
\label{subsec:hw-74hc595}

Der 74HC595 ist ein Serial-In/Parallel-Out Schieberegister mit Latch. Die Daten
werden seriell eingetaktet und erst beim Latch-Impuls (RCK) an die Ausgänge
übernommen.

\begin{figure}[H]
\centering
\begin{lstlisting}[basicstyle=\ttfamily\scriptsize, frame=single, numbers=none]
                     74HC595 (#0)                    74HC595 (#1)
                  +-------------+                 +-------------+
    LED 2  <------|  1 QB  VCC 16|--- 3V3 LED 10 <|  1 QB  VCC 16|--- 3V3
    LED 3  <------|  2 QC  QA  15|--> LED 1       |  2 QC  QA  15|--> LED 9
    LED 4  <------|  3 QD  SER 14|<-- D10 (MOSI)  |  3 QD  SER 14|<-- #0.QH'
    LED 5  <------|  4 QE  OE  13|<-- D2 (PWM)    |  4 QE  OE  13|<-- D2 (PWM)
    LED 6  <------|  5 QF  RCK 12|<-- D0 (RCK)    |  5 QF  RCK 12|<-- D0 (RCK)
    LED 7  <------|  6 QG  SCK 11|<-- D8 (SCK)    |  6 QG  SCK 11|<-- D8 (SCK)
    LED 8  <------|  7 QH  CLR 10|--- 3V3         |  7 QH  CLR 10|--- 3V3
          GND ----|  8 GND QH'  9|---> #1.SER     |  8 GND QH'  9|    (n.c.)
                  +------+------+                 +------+------+
                         |                               |
                        === C1                          === C2
                        100nF                           100nF
                         |                               |
                        GND                             GND
\end{lstlisting}
\caption{74HC595 Pinout und Daisy-Chain}
\label{fig:74hc595-detail}
\end{figure}

\begin{warnbox}[CLR-Pin nicht floaten!]
Der \pin{CLR}-Pin (Pin 10) muss auf \SI{3.3}{\volt} gelegt werden. Ein floatender
CLR-Pin führt zu sporadischem Löschen des Schieberegisters.
\end{warnbox}

\subsubsection{LED-Beschaltung (Active-High)}

\begin{figure}[H]
\centering
\begin{lstlisting}[basicstyle=\ttfamily\small, frame=single, numbers=none]
    Qx (74HC595)
     |
    +-+
    | | 330 - 3k Ohm
    +-+
     |
     v LED
     |
    GND
\end{lstlisting}
\caption{LED-Vorwiderstand-Beschaltung}
\label{fig:led-beschaltung}
\end{figure}

Die Stromberechnung für verschiedene Widerstandswerte bei einer LED-Flussspannung
von ca.\ \SI{2.0}{\volt}:

\begin{table}[H]
\centering
\caption{LED-Strom in Abhängigkeit vom Vorwiderstand}
\label{tab:led-strom}
\begin{tabular}{@{}r r l@{}}
\toprule
\textbf{Widerstand} & \textbf{Strom} & \textbf{Helligkeit} \\
\midrule
\SI{330}{\ohm}  & $\approx$\,\SI{4}{\milli\ampere}   & hell \\
\SI{1}{\kilo\ohm}   & $\approx$\,\SI{1.3}{\milli\ampere} & gedimmt \\
\SI{3}{\kilo\ohm}   & $\approx$\,\SI{0.4}{\milli\ampere} & schwach \\
\bottomrule
\end{tabular}
\end{table}

% -----------------------------------------------------------------------------
\subsection{CD4021B Detailschaltplan (Taster-Eingang)}
\label{subsec:hw-cd4021b}

Der CD4021B ist ein Parallel-In/Serial-Out Schieberegister. Er liest acht
parallele Eingänge ein und gibt sie seriell aus.

\subsubsection{Pinout}

\begin{figure}[H]
\centering
\begin{lstlisting}[basicstyle=\ttfamily\small, frame=single, numbers=none]
              +------------+
    PI-8  ----|  1  *  16  |---- VDD
      Q6  ----|  2     15  |---- PI-7
      Q8  ----|  3     14  |---- PI-6
    PI-4  ----|  4     13  |---- PI-5
    PI-3  ----|  5     12  |---- Q7
    PI-2  ----|  6     11  |---- SERIAL IN (DS)
    PI-1  ----|  7     10  |---- CLOCK
     VSS  ----|  8      9  |---- P/S CONTROL
              +------------+
\end{lstlisting}
\caption{CD4021B Pinout (DIP-16)}
\label{fig:cd4021b-pinout}
\end{figure}

\subsubsection{Beschaltung für 10 Taster}

\begin{figure}[H]
\centering
\begin{lstlisting}[basicstyle=\ttfamily\scriptsize, frame=single, numbers=none]
                     CD4021B (#0)                        CD4021B (#1)
                  +--------------+                    +--------------+
    BTN 1  ------|  1 PI-8 VDD 16|--- 3V3   BTN 9 ---|  1 PI-8 VDD 16|--- 3V3
                 |  2 Q6  PI-7 15|<-- BTN 2          |  2 Q6  PI-7 15|<-- BTN 10
   D9 (MISO) <---|  3 Q8  PI-6 14|<-- BTN 3          |  3 Q8  PI-6 14|    +3V3
    BTN 5  ------|  4 PI-4 PI-5 13|<-- BTN 4          |  4 PI-4 PI-5 13|    +3V3
    BTN 6  ------|  5 PI-3  Q7  12|    (n.c.)         |  5 PI-3  Q7  12|    (n.c.)
    BTN 7  ------|  6 PI-2  DS  11|<-- Q8 von #1      |  6 PI-2  DS  11|--- 3V3
    BTN 8  ------|  7 PI-1 CLK  10|<-- D8 (SCK)       |  7 PI-1 CLK  10|<-- D8 (SCK)
          GND ---|  8 VSS  P/S   9|<-- D1 (PS)        |  8 VSS  P/S   9|<-- D1 (PS)
                 +-------+-------+                   +-------+-------+
                         |       <----- Q8 (Pin 3) ------+
                        === C3                          === C4
                        100nF                           100nF
                         |                               |
                        GND                             GND
\end{lstlisting}
\caption{CD4021B Beschaltung mit Daisy-Chain}
\label{fig:cd4021b-beschaltung}
\end{figure}

\subsubsection{Pin-Zuordnung Taster → CD4021B}

Der CD4021B gibt Daten \highlight{MSB-first} aus: PI-8 erscheint zuerst (Bit 7),
PI-1 zuletzt (Bit 0).

\begin{table}[H]
\centering
\caption{Taster-zu-Bit-Zuordnung}
\label{tab:btn-mapping}
\begin{tabular}{@{}c c c c c@{}}
\toprule
\textbf{Taster} & \textbf{Chip} & \textbf{Pin-Name} & \textbf{Pin-Nr} & \textbf{Bit im Byte} \\
\midrule
BTN 1  & \#0 & PI-8 &  1 & Bit 7 \\
BTN 2  & \#0 & PI-7 & 15 & Bit 6 \\
BTN 3  & \#0 & PI-6 & 14 & Bit 5 \\
BTN 4  & \#0 & PI-5 & 13 & Bit 4 \\
BTN 5  & \#0 & PI-4 &  4 & Bit 3 \\
BTN 6  & \#0 & PI-3 &  5 & Bit 2 \\
BTN 7  & \#0 & PI-2 &  6 & Bit 1 \\
BTN 8  & \#0 & PI-1 &  7 & Bit 0 \\
\midrule
BTN 9  & \#1 & PI-8 &  1 & Bit 7 (Byte 1) \\
BTN 10 & \#1 & PI-7 & 15 & Bit 6 (Byte 1) \\
\bottomrule
\end{tabular}
\end{table}

\subsubsection{Daisy-Chain Datenfluss}

\begin{figure}[H]
\centering
\begin{lstlisting}[basicstyle=\ttfamily\small, frame=single, numbers=none]
3V3 --> DS [CD4021B #1] --> Q8 --> DS [CD4021B #0] --> Q8 --> D9 (MISO) --> ESP32
              |                          |
              |  BTN 9-10                |  BTN 1-8
              |  (Byte 1)                |  (Byte 0)
              |                          |
              +-- Spaeter gelesen -------+-- Zuerst gelesen -->
\end{lstlisting}
\caption{Datenfluss in der CD4021B Daisy-Chain}
\label{fig:daisy-chain-flow}
\end{figure}

\begin{warnbox}[DS-Pin des letzten Chips]
Der \pin{DS}-Pin (Pin 11) des \textbf{letzten} CD4021B muss auf \SI{+3.3}{\volt}
gelegt werden. Bei GND würden Nullen nachgeschoben, die als \enquote{gedrückt}
fehlinterpretiert werden. Ungenutzte PI-x Pins ebenfalls auf \SI{+3.3}{\volt}
legen -- interne Pull-ups reichen nicht zuverlässig.
\end{warnbox}

\subsubsection{Taster-Beschaltung (Active-Low)}

\begin{figure}[H]
\centering
\begin{lstlisting}[basicstyle=\ttfamily\small, frame=single, numbers=none]
    3V3
     |
    +-+
    | | 10 kOhm (Pull-up)
    +-+
     +--------- PIx (CD4021B)
     |
    +-+
    | | Taster
    +-+
     |
    GND

Zustand:
  - Losgelassen: PIx = HIGH (1) -> in Firmware: nicht gedrueckt
  - Gedrueckt:   PIx = LOW  (0) -> in Firmware: gedrueckt
\end{lstlisting}
\caption{Taster mit Pull-up-Widerstand}
\label{fig:taster-beschaltung}
\end{figure}

% -----------------------------------------------------------------------------
\subsection{SPI-Bus Verkabelung}
\label{subsec:hw-spi}

\begin{table}[H]
\centering
\caption{SPI-Signalführung}
\label{tab:spi-verkabelung}
\begin{tabularx}{\textwidth}{@{}l l X@{}}
\toprule
\textbf{ESP32-S3} & \textbf{74HC595} & \textbf{CD4021B} \\
\midrule
\pin{D10} (MOSI) & SER (Pin 14, \#0) $\rightarrow$ QH' $\rightarrow$ SER (\#1) & -- \\
\pin{D8} (SCK)   & SCK (Pin 11, beide) & CLK (Pin 10, beide) \\
\pin{D9} (MISO)  & -- & Q8 (Pin 3, \#0) $\leftarrow$ DS $\leftarrow$ Q8 (\#1) \\
\pin{D0} (RCK)   & RCK (Pin 12, beide) & -- \\
\pin{D1} (PS)    & -- & P/S (Pin 9, beide) \\
\pin{D2} (OE)    & OE (Pin 13, beide) & -- \\
\bottomrule
\end{tabularx}
\end{table}

% -----------------------------------------------------------------------------
\subsection{Timing-Diagramme}
\label{subsec:hw-timing}

\subsubsection{CD4021B Lesevorgang}

\begin{figure}[H]
\centering
\begin{lstlisting}[basicstyle=\ttfamily\small, frame=single, numbers=none]
    P/S -----+     +-------------------------
             |     |
             +-----+
                   ^
                   Parallel Load (HIGH->LOW->HIGH)

    CLK -------------+  +--+  +--+  +--+  +--
                     |  |  |  |  |  |  |  |
                     +--+  +--+  +--+  +--+
                        ^     ^     ^     ^
                       Bit6  Bit5  Bit4  Bit3 ...

    Q8  ============##=====##=====##=====##====
                    || Bit7|| Bit6|| Bit5|| ...
                    ##=====##=====##=====##====
                    ^
                    First-Bit (PI-8) liegt VOR erstem Clock an!
\end{lstlisting}
\caption{Timing des CD4021B Lesevorgangs}
\label{fig:timing-cd4021b}
\end{figure}

\subsubsection{74HC595 Schreibvorgang}

\begin{figure}[H]
\centering
\begin{lstlisting}[basicstyle=\ttfamily\small, frame=single, numbers=none]
    SER ----##=====##=====##=====##=====------
            || Bit7|| Bit6|| Bit5|| ...
            ##=====##=====##=====##=====------

    SCK ---------+  +--+  +--+  +--+  +-----
                 |  |  |  |  |  |  |  |
                 +--+  +--+  +--+  +--+
                    ^     ^     ^     ^
                   Daten bei steigender Flanke uebernommen

    RCK ------------------------------------+  +--
                                            |  |
                                            +--+
                                               ^
                                              Latch: Shift -> Output
\end{lstlisting}
\caption{Timing des 74HC595 Schreibvorgangs}
\label{fig:timing-74hc595}
\end{figure}

% -----------------------------------------------------------------------------
\subsection{Stromlimits}
\label{subsec:hw-stromlimits}

\begin{table}[H]
\centering
\caption{Elektrische Grenzwerte der ICs}
\label{tab:stromlimits}
\begin{tabular}{@{}l l r l@{}}
\toprule
\textbf{Komponente} & \textbf{Parameter} & \textbf{Wert} & \textbf{Bemerkung} \\
\midrule
\textbf{ESP32-S3} & GPIO Output (max)     & \SI{40}{\milli\ampere}   & Drive Strength 3 \\
                  & GPIO Output (default) & \SI{20}{\milli\ampere}   & Drive Strength 2 \\
                  & Gesamt alle GPIOs     & \SI{1200}{\milli\ampere} & Summe \\
\midrule
\textbf{74HC595}  & Output pro Pin (max)      & $\pm$\SI{35}{\milli\ampere} & Absolutes Maximum \\
                  & Output pro Pin (empfohlen)& $\pm$\SI{6}{\milli\ampere}  & Dauerbetrieb \\
                  & VCC/GND gesamt            & \SIrange{70}{75}{\milli\ampere} & \critical{Package-Limit!} \\
\midrule
\textbf{CD4021B}  & Input (pro Pin)  & $<$\,\SI{1}{\micro\ampere} & CMOS-Eingang \\
                  & Output Q8        & \SIrange{1}{3}{\milli\ampere} & Für SPI ausreichend \\
\bottomrule
\end{tabular}
\end{table}

\subsubsection{Stromversorgung}

\begin{table}[H]
\centering
\caption{Stromaufnahme des Systems}
\label{tab:stromversorgung}
\begin{tabular}{@{}l r r@{}}
\toprule
\textbf{Komponente} & \textbf{Typisch} & \textbf{Maximum} \\
\midrule
ESP32-S3            & \SI{80}{\milli\ampere}  & \SI{500}{\milli\ampere} (WiFi aktiv) \\
CD4021B ($\times$2) & $<$\,\SI{1}{\milli\ampere} & \SI{1}{\milli\ampere} \\
74HC595 ($\times$2) & $<$\,\SI{1}{\milli\ampere} & \SI{70}{\milli\ampere} (alle Ausgänge) \\
LEDs ($\times$10 @ \SI{4}{\milli\ampere}) & \SI{40}{\milli\ampere} & \SI{40}{\milli\ampere} \\
\midrule
\textbf{Gesamt}     & \textbf{$\approx$\,\SI{130}{\milli\ampere}} & \textbf{$\approx$\,\SI{620}{\milli\ampere}} \\
\bottomrule
\end{tabular}
\end{table}

\begin{tipbox}[USB-Stromversorgung]
Die USB-CDC-Versorgung vom Pi liefert bis zu \SI{500}{\milli\ampere}. Bei mehr als
8 LEDs gleichzeitig oder höheren Strömen: Helligkeit per PWM reduzieren oder externe
\SI{5}{\volt}-Versorgung verwenden.
\end{tipbox}

% -----------------------------------------------------------------------------
\subsection{Hardware-Eigenheiten}
\label{subsec:hw-eigenheiten}

\subsubsection{First-Bit-Problem (CD4021B)}

Nach dem Parallel-Load liegt das erste Bit (PI-8 $\rightarrow$ Q8) sofort am Ausgang,
\textbf{bevor} der erste Clock kommt. Der ESP32 samplet aber erst \textbf{nach} der
ersten Clock-Flanke. Die Firmware löst dies durch einen \lstinline[style=arduino]|digitalRead()|
vor dem SPI-Transfer.

\subsubsection{DS-Pin des letzten CD4021B}

Der DS-Pin (Serial Data Input, \textbf{Pin 11}) des \textbf{letzten} CD4021B in der
Kette muss auf \SI{+3.3}{\volt} gelegt werden, nicht auf GND. Bei GND würden beim
Einlesen Nullen nachgeschoben, die als \enquote{gedrückt} fehlinterpretiert werden.

\subsubsection{SPI-Bus Crosstalk}

Wenn der ESP32 vom CD4021B liest, taktet er dabei Nullen durch den 74HC595. Die
Ausgänge ändern sich erst beim Latch-Impuls, aber ein glitchender RCK-Pin könnte
LEDs kurz ausschalten. Die Firmware kompensiert dies durch
\lstinline[style=arduino]|LED_REFRESH_EVERY_CYCLE = true|.

\subsubsection{Active-Low Taster}

Die Taster sind Active-Low beschaltet (gedrückt = 0, losgelassen = 1). Die Firmware
invertiert dies in \filep{bitops.h}, sodass die Logik-Schicht mit
\lstinline[style=arduino]|pressed = true| arbeitet.

\subsubsection{Stützkondensatoren}

Jeder IC benötigt einen \SI{100}{\nano\farad} Keramikkondensator zwischen VCC und
GND, möglichst nah am Chip. Ohne diese Kondensatoren können Störungen auf der
Versorgung zu Fehlfunktionen führen.

% -----------------------------------------------------------------------------
\subsection{Skalierung auf 100 Buttons}
\label{subsec:hw-skalierung}

Für das 100-Button-System verlängern wir die Daisy-Chain auf jeweils 13
Schieberegister:

\begin{table}[H]
\centering
\caption{Komponentenvergleich 10 vs.\ 100 Buttons}
\label{tab:skalierung-hardware}
\begin{tabular}{@{}l r r@{}}
\toprule
\textbf{Komponente} & \textbf{10-Button} & \textbf{100-Button} \\
\midrule
CD4021B       & 2   & 13 \\
74HC595       & 2   & 13 \\
Taster        & 10  & 100 \\
LEDs          & 10  & 100 \\
R (LED)       & 10  & 100 \\
R (Pull-up)   & 10  & 100 \\
C (\SI{100}{\nano\farad}) & 4 & 26 \\
\bottomrule
\end{tabular}
\end{table}

Die SPI-Transferzeit steigt von ca.\ \SI{20}{\micro\second} auf ca.\
\SI{260}{\micro\second} -- weit unter dem \SI{5}{\milli\second}-Zyklus der
Firmware.

% Version: 2.5.2 | Stand: 2026-01-11
% =============================================================================

\section{Hardware}
\label{sec:hardware}

Dieses Kapitel dokumentiert den 10-Button-Prototyp des Selection Panels. Die Schaltung
lässt sich auf 100 Taster skalieren, indem wir die Daisy-Chain der Schieberegister
verlängern.

% -----------------------------------------------------------------------------
\subsection{Komponentenübersicht}
\label{subsec:hw-komponenten}

\begin{table}[H]
\centering
\caption{Stückliste des 10-Button-Prototyps}
\label{tab:hw-komponenten}
\begin{tabularx}{\textwidth}{@{}l l r X@{}}
\toprule
\textbf{Komponente} & \textbf{Typ} & \textbf{Anzahl} & \textbf{Funktion} \\
\midrule
XIAO ESP32-S3       & Mikrocontroller         & 1  & Steuerlogik, USB-CDC \\
Raspberry Pi 5      & SBC + Netzteil + microSD & 1 & Server, Dashboard, Medien \\
CD4021B             & 8-Bit PISO Schieberegister & 2 & Taster einlesen \\
74HC595             & 8-Bit SIPO Schieberegister & 2 & LEDs ansteuern \\
Taster              & $6 \times 6$\,\si{\milli\meter} Tactile Switch & 10 & Benutzereingabe \\
LED                 & \SI{5}{\milli\meter}, versch.\ Farben & 10 & Statusanzeige \\
Widerstand          & \SIrange{330}{3000}{\ohm} & 10 & LED-Strombegrenzung \\
Widerstand          & \SI{10}{\kilo\ohm}       & 10 & Pull-up für Taster \\
Kondensator         & \SI{100}{\nano\farad} (Keramik) & 4 & Stützkondensatoren \\
\bottomrule
\end{tabularx}
\end{table}

% -----------------------------------------------------------------------------
\subsection{Pin-Belegung XIAO ESP32-S3}
\label{subsec:hw-pinout}

\begin{figure}[H]
\centering
\begin{lstlisting}[basicstyle=\ttfamily\footnotesize, frame=single, numbers=none]
                  +-------------+
                  |    USB-C    |
                  |             |
 LED_RCK    D0 ---|  1       14 |--- 5V
 BTN_PS     D1 ---|  2       13 |--- GND
 LED_OE     D2 ---|  3       12 |--- 3V3
            D3 ---|  4       11 |--- D10  LED_MOSI
            D4 ---|  5       10 |--- D9   BTN_MISO
            D5 ---|  6        9 |--- D8   SCK
            D6 ---|  7        8 |--- D7
                  |             |
                  +-------------+
\end{lstlisting}
\caption{Pin-Belegung des XIAO ESP32-S3}
\label{fig:xiao-pinout}
\end{figure}

\begin{table}[H]
\centering
\caption{GPIO-Zuordnung für SPI und Steuerung}
\label{tab:gpio-zuordnung}
\begin{tabularx}{\textwidth}{@{}l l l X@{}}
\toprule
\textbf{Pin} & \textbf{Signal} & \textbf{Funktion} & \textbf{Ziel} \\
\midrule
\pin{D0}  & \texttt{LED\_RCK}  & Latch (STCP)              & 74HC595 Pin 12 \\
\pin{D1}  & \texttt{BTN\_PS}   & Parallel/Serial Control   & CD4021B Pin 9 \\
\pin{D2}  & \texttt{LED\_OE}   & Output Enable (PWM, active-low) & 74HC595 Pin 13 \\
\pin{D8}  & \texttt{SCK}       & SPI Clock                 & Beide Chip-Typen \\
\pin{D9}  & \texttt{BTN\_MISO} & Daten von CD4021B         & CD4021B \#0 Pin 3 (Q8) \\
\pin{D10} & \texttt{LED\_MOSI} & Daten zu 74HC595          & 74HC595 \#0 Pin 14 (SER) \\
\bottomrule
\end{tabularx}
\end{table}

% -----------------------------------------------------------------------------
\subsection{Byte-Mapping (Firmware \texorpdfstring{$\leftrightarrow$}{<->} Hardware)}
\label{subsec:hw-byte-mapping}

Die Asymmetrie zwischen Ein- und Ausgabe ergibt sich aus der Hardware: Der CD4021B
schiebt sein erstes Sample (PI-8) als erstes Bit heraus, während der 74HC595 das
zuletzt empfangene Bit auf QA (Ausgang 0) legt.

\begin{table}[H]
\centering
\caption{Bit-Zuordnung der Schieberegister}
\label{tab:byte-mapping}
\begin{tabular}{@{}l l l@{}}
\toprule
\textbf{IC} & \textbf{Byte 0} & \textbf{Byte 1} \\
\midrule
74HC595 (LEDs)    & LED 1--8 = Bit 0--7 & LED 9--10 = Bit 0--1 \\
CD4021 (Taster)   & BTN 1--8 = Bit 7--0 & BTN 9--10 = Bit 7--6 \\
\bottomrule
\end{tabular}
\end{table}

\begin{infobox}[Firmware-Abstraktion]
Die Firmware verwendet in \filep{bitops.h} die Formel \lstinline[style=arduino]|btn_bit(id) = 7 - ((id - 1) % 8)|,
um diese Zuordnung zu abstrahieren. Damit arbeitet die Logik-Schicht einheitlich mit
Button-IDs von 1--100.
\end{infobox}

% -----------------------------------------------------------------------------
\subsection{Schaltplan-Übersicht}
\label{subsec:hw-schaltplan}

\begin{figure}[H]
\centering
\begin{lstlisting}[basicstyle=\ttfamily\scriptsize, frame=single, numbers=none]
                        +--------------------------------------------------+
                        |                  OUTPUT (LEDs)                   |
    +---------+         |                                                  |
    | Pi 5    |         |   +----------+  QH'     +----------+             |
    |         |  USB-C  |   | 74HC595  |--------->| 74HC595  |             |
    |         |<------->|   |   #0     |   SER    |   #1     |             |
    +---------+         |   |          |          |          |             |
                        |   | QA-QH    |          | QA-QB    |             |
        +---------+     |   |   |      |          |   |      |             |
        |  XIAO   |     |   | LED 1-8  |          | LED 9-10 |             |
        | ESP32S3 |     |   +----------+          +----------+             |
        |         |     +--------------------------------------------------+
        | D10 ----+----------> SER #0 (MOSI)
        | D8  ----+----------> SCK (beide)
        | D0  ----+----------> RCK (beide)
        | D2  ----+----------> OE (beide)
        |         |
        | D9  <---+---------- Q8 #0 (MISO)
        | D8  ----+----------> CLK (beide)
        | D1  ----+----------> P/S (beide)
        |         |     +--------------------------------------------------+
        +---------+     |                  INPUT (Taster)                  |
                        |                                                  |
                        |   +----------+          +----------+             |
                        |   | CD4021B  |<-- DS ---| CD4021B  |<-- DS -- 3V3|
                        |   |   #0     |    Q8    |   #1     |             |
                        |   |          |<---------|          |             |
                        |   | PI 8-1   |          | PI 8-7   |             |
                        |   |   |      |          |   |      |             |
                        |   | BTN 1-8  |          | BTN 9-10 |             |
                        |   +----------+          +----------+             |
                        |        |                                         |
                        |        v                                         |
                        |      Q8 --> D9 (MISO)                            |
                        +--------------------------------------------------+
\end{lstlisting}
\caption{Gesamtübersicht der Schaltung}
\label{fig:schaltplan-uebersicht}
\end{figure}

% -----------------------------------------------------------------------------
\subsection{74HC595 Detailschaltplan (LED-Ausgang)}
\label{subsec:hw-74hc595}

Der 74HC595 ist ein Serial-In/Parallel-Out Schieberegister mit Latch. Die Daten
werden seriell eingetaktet und erst beim Latch-Impuls (RCK) an die Ausgänge
übernommen.

\begin{figure}[H]
\centering
\begin{lstlisting}[basicstyle=\ttfamily\scriptsize, frame=single, numbers=none]
                     74HC595 (#0)                    74HC595 (#1)
                  +-------------+                 +-------------+
    LED 2  <------|  1 QB  VCC 16|--- 3V3 LED 10 <|  1 QB  VCC 16|--- 3V3
    LED 3  <------|  2 QC  QA  15|--> LED 1       |  2 QC  QA  15|--> LED 9
    LED 4  <------|  3 QD  SER 14|<-- D10 (MOSI)  |  3 QD  SER 14|<-- #0.QH'
    LED 5  <------|  4 QE  OE  13|<-- D2 (PWM)    |  4 QE  OE  13|<-- D2 (PWM)
    LED 6  <------|  5 QF  RCK 12|<-- D0 (RCK)    |  5 QF  RCK 12|<-- D0 (RCK)
    LED 7  <------|  6 QG  SCK 11|<-- D8 (SCK)    |  6 QG  SCK 11|<-- D8 (SCK)
    LED 8  <------|  7 QH  CLR 10|--- 3V3         |  7 QH  CLR 10|--- 3V3
          GND ----|  8 GND QH'  9|---> #1.SER     |  8 GND QH'  9|    (n.c.)
                  +------+------+                 +------+------+
                         |                               |
                        === C1                          === C2
                        100nF                           100nF
                         |                               |
                        GND                             GND
\end{lstlisting}
\caption{74HC595 Pinout und Daisy-Chain}
\label{fig:74hc595-detail}
\end{figure}

\begin{warnbox}[CLR-Pin nicht floaten!]
Der \pin{CLR}-Pin (Pin 10) muss auf \SI{3.3}{\volt} gelegt werden. Ein floatender
CLR-Pin führt zu sporadischem Löschen des Schieberegisters.
\end{warnbox}

\subsubsection{LED-Beschaltung (Active-High)}

\begin{figure}[H]
\centering
\begin{lstlisting}[basicstyle=\ttfamily\small, frame=single, numbers=none]
    Qx (74HC595)
     |
    +-+
    | | 330 - 3k Ohm
    +-+
     |
     v LED
     |
    GND
\end{lstlisting}
\caption{LED-Vorwiderstand-Beschaltung}
\label{fig:led-beschaltung}
\end{figure}

Die Stromberechnung für verschiedene Widerstandswerte bei einer LED-Flussspannung
von ca.\ \SI{2.0}{\volt}:

\begin{table}[H]
\centering
\caption{LED-Strom in Abhängigkeit vom Vorwiderstand}
\label{tab:led-strom}
\begin{tabular}{@{}r r l@{}}
\toprule
\textbf{Widerstand} & \textbf{Strom} & \textbf{Helligkeit} \\
\midrule
\SI{330}{\ohm}  & $\approx$\,\SI{4}{\milli\ampere}   & hell \\
\SI{1}{\kilo\ohm}   & $\approx$\,\SI{1.3}{\milli\ampere} & gedimmt \\
\SI{3}{\kilo\ohm}   & $\approx$\,\SI{0.4}{\milli\ampere} & schwach \\
\bottomrule
\end{tabular}
\end{table}

% -----------------------------------------------------------------------------
\subsection{CD4021B Detailschaltplan (Taster-Eingang)}
\label{subsec:hw-cd4021b}

Der CD4021B ist ein Parallel-In/Serial-Out Schieberegister. Er liest acht
parallele Eingänge ein und gibt sie seriell aus.

\subsubsection{Pinout}

\begin{figure}[H]
\centering
\begin{lstlisting}[basicstyle=\ttfamily\small, frame=single, numbers=none]
              +------------+
    PI-8  ----|  1  *  16  |---- VDD
      Q6  ----|  2     15  |---- PI-7
      Q8  ----|  3     14  |---- PI-6
    PI-4  ----|  4     13  |---- PI-5
    PI-3  ----|  5     12  |---- Q7
    PI-2  ----|  6     11  |---- SERIAL IN (DS)
    PI-1  ----|  7     10  |---- CLOCK
     VSS  ----|  8      9  |---- P/S CONTROL
              +------------+
\end{lstlisting}
\caption{CD4021B Pinout (DIP-16)}
\label{fig:cd4021b-pinout}
\end{figure}

\subsubsection{Beschaltung für 10 Taster}

\begin{figure}[H]
\centering
\begin{lstlisting}[basicstyle=\ttfamily\scriptsize, frame=single, numbers=none]
                     CD4021B (#0)                        CD4021B (#1)
                  +--------------+                    +--------------+
    BTN 1  ------|  1 PI-8 VDD 16|--- 3V3   BTN 9 ---|  1 PI-8 VDD 16|--- 3V3
                 |  2 Q6  PI-7 15|<-- BTN 2          |  2 Q6  PI-7 15|<-- BTN 10
   D9 (MISO) <---|  3 Q8  PI-6 14|<-- BTN 3          |  3 Q8  PI-6 14|    +3V3
    BTN 5  ------|  4 PI-4 PI-5 13|<-- BTN 4          |  4 PI-4 PI-5 13|    +3V3
    BTN 6  ------|  5 PI-3  Q7  12|    (n.c.)         |  5 PI-3  Q7  12|    (n.c.)
    BTN 7  ------|  6 PI-2  DS  11|<-- Q8 von #1      |  6 PI-2  DS  11|--- 3V3
    BTN 8  ------|  7 PI-1 CLK  10|<-- D8 (SCK)       |  7 PI-1 CLK  10|<-- D8 (SCK)
          GND ---|  8 VSS  P/S   9|<-- D1 (PS)        |  8 VSS  P/S   9|<-- D1 (PS)
                 +-------+-------+                   +-------+-------+
                         |       <----- Q8 (Pin 3) ------+
                        === C3                          === C4
                        100nF                           100nF
                         |                               |
                        GND                             GND
\end{lstlisting}
\caption{CD4021B Beschaltung mit Daisy-Chain}
\label{fig:cd4021b-beschaltung}
\end{figure}

\subsubsection{Pin-Zuordnung Taster → CD4021B}

Der CD4021B gibt Daten \highlight{MSB-first} aus: PI-8 erscheint zuerst (Bit 7),
PI-1 zuletzt (Bit 0).

\begin{table}[H]
\centering
\caption{Taster-zu-Bit-Zuordnung}
\label{tab:btn-mapping}
\begin{tabular}{@{}c c c c c@{}}
\toprule
\textbf{Taster} & \textbf{Chip} & \textbf{Pin-Name} & \textbf{Pin-Nr} & \textbf{Bit im Byte} \\
\midrule
BTN 1  & \#0 & PI-8 &  1 & Bit 7 \\
BTN 2  & \#0 & PI-7 & 15 & Bit 6 \\
BTN 3  & \#0 & PI-6 & 14 & Bit 5 \\
BTN 4  & \#0 & PI-5 & 13 & Bit 4 \\
BTN 5  & \#0 & PI-4 &  4 & Bit 3 \\
BTN 6  & \#0 & PI-3 &  5 & Bit 2 \\
BTN 7  & \#0 & PI-2 &  6 & Bit 1 \\
BTN 8  & \#0 & PI-1 &  7 & Bit 0 \\
\midrule
BTN 9  & \#1 & PI-8 &  1 & Bit 7 (Byte 1) \\
BTN 10 & \#1 & PI-7 & 15 & Bit 6 (Byte 1) \\
\bottomrule
\end{tabular}
\end{table}

\subsubsection{Daisy-Chain Datenfluss}

\begin{figure}[H]
\centering
\begin{lstlisting}[basicstyle=\ttfamily\small, frame=single, numbers=none]
3V3 --> DS [CD4021B #1] --> Q8 --> DS [CD4021B #0] --> Q8 --> D9 (MISO) --> ESP32
              |                          |
              |  BTN 9-10                |  BTN 1-8
              |  (Byte 1)                |  (Byte 0)
              |                          |
              +-- Spaeter gelesen -------+-- Zuerst gelesen -->
\end{lstlisting}
\caption{Datenfluss in der CD4021B Daisy-Chain}
\label{fig:daisy-chain-flow}
\end{figure}

\begin{warnbox}[DS-Pin des letzten Chips]
Der \pin{DS}-Pin (Pin 11) des \textbf{letzten} CD4021B muss auf \SI{+3.3}{\volt}
gelegt werden. Bei GND würden Nullen nachgeschoben, die als \enquote{gedrückt}
fehlinterpretiert werden. Ungenutzte PI-x Pins ebenfalls auf \SI{+3.3}{\volt}
legen -- interne Pull-ups reichen nicht zuverlässig.
\end{warnbox}

\subsubsection{Taster-Beschaltung (Active-Low)}

\begin{figure}[H]
\centering
\begin{lstlisting}[basicstyle=\ttfamily\small, frame=single, numbers=none]
    3V3
     |
    +-+
    | | 10 kOhm (Pull-up)
    +-+
     +--------- PIx (CD4021B)
     |
    +-+
    | | Taster
    +-+
     |
    GND

Zustand:
  - Losgelassen: PIx = HIGH (1) -> in Firmware: nicht gedrueckt
  - Gedrueckt:   PIx = LOW  (0) -> in Firmware: gedrueckt
\end{lstlisting}
\caption{Taster mit Pull-up-Widerstand}
\label{fig:taster-beschaltung}
\end{figure}

% -----------------------------------------------------------------------------
\subsection{SPI-Bus Verkabelung}
\label{subsec:hw-spi}

\begin{table}[H]
\centering
\caption{SPI-Signalführung}
\label{tab:spi-verkabelung}
\begin{tabularx}{\textwidth}{@{}l l X@{}}
\toprule
\textbf{ESP32-S3} & \textbf{74HC595} & \textbf{CD4021B} \\
\midrule
\pin{D10} (MOSI) & SER (Pin 14, \#0) $\rightarrow$ QH' $\rightarrow$ SER (\#1) & -- \\
\pin{D8} (SCK)   & SCK (Pin 11, beide) & CLK (Pin 10, beide) \\
\pin{D9} (MISO)  & -- & Q8 (Pin 3, \#0) $\leftarrow$ DS $\leftarrow$ Q8 (\#1) \\
\pin{D0} (RCK)   & RCK (Pin 12, beide) & -- \\
\pin{D1} (PS)    & -- & P/S (Pin 9, beide) \\
\pin{D2} (OE)    & OE (Pin 13, beide) & -- \\
\bottomrule
\end{tabularx}
\end{table}

% -----------------------------------------------------------------------------
\subsection{Timing-Diagramme}
\label{subsec:hw-timing}

\subsubsection{CD4021B Lesevorgang}

\begin{figure}[H]
\centering
\begin{lstlisting}[basicstyle=\ttfamily\small, frame=single, numbers=none]
    P/S -----+     +-------------------------
             |     |
             +-----+
                   ^
                   Parallel Load (HIGH->LOW->HIGH)

    CLK -------------+  +--+  +--+  +--+  +--
                     |  |  |  |  |  |  |  |
                     +--+  +--+  +--+  +--+
                        ^     ^     ^     ^
                       Bit6  Bit5  Bit4  Bit3 ...

    Q8  ============##=====##=====##=====##====
                    || Bit7|| Bit6|| Bit5|| ...
                    ##=====##=====##=====##====
                    ^
                    First-Bit (PI-8) liegt VOR erstem Clock an!
\end{lstlisting}
\caption{Timing des CD4021B Lesevorgangs}
\label{fig:timing-cd4021b}
\end{figure}

\subsubsection{74HC595 Schreibvorgang}

\begin{figure}[H]
\centering
\begin{lstlisting}[basicstyle=\ttfamily\small, frame=single, numbers=none]
    SER ----##=====##=====##=====##=====------
            || Bit7|| Bit6|| Bit5|| ...
            ##=====##=====##=====##=====------

    SCK ---------+  +--+  +--+  +--+  +-----
                 |  |  |  |  |  |  |  |
                 +--+  +--+  +--+  +--+
                    ^     ^     ^     ^
                   Daten bei steigender Flanke uebernommen

    RCK ------------------------------------+  +--
                                            |  |
                                            +--+
                                               ^
                                              Latch: Shift -> Output
\end{lstlisting}
\caption{Timing des 74HC595 Schreibvorgangs}
\label{fig:timing-74hc595}
\end{figure}

% -----------------------------------------------------------------------------
\subsection{Stromlimits}
\label{subsec:hw-stromlimits}

\begin{table}[H]
\centering
\caption{Elektrische Grenzwerte der ICs}
\label{tab:stromlimits}
\begin{tabular}{@{}l l r l@{}}
\toprule
\textbf{Komponente} & \textbf{Parameter} & \textbf{Wert} & \textbf{Bemerkung} \\
\midrule
\textbf{ESP32-S3} & GPIO Output (max)     & \SI{40}{\milli\ampere}   & Drive Strength 3 \\
                  & GPIO Output (default) & \SI{20}{\milli\ampere}   & Drive Strength 2 \\
                  & Gesamt alle GPIOs     & \SI{1200}{\milli\ampere} & Summe \\
\midrule
\textbf{74HC595}  & Output pro Pin (max)      & $\pm$\SI{35}{\milli\ampere} & Absolutes Maximum \\
                  & Output pro Pin (empfohlen)& $\pm$\SI{6}{\milli\ampere}  & Dauerbetrieb \\
                  & VCC/GND gesamt            & \SIrange{70}{75}{\milli\ampere} & \critical{Package-Limit!} \\
\midrule
\textbf{CD4021B}  & Input (pro Pin)  & $<$\,\SI{1}{\micro\ampere} & CMOS-Eingang \\
                  & Output Q8        & \SIrange{1}{3}{\milli\ampere} & Für SPI ausreichend \\
\bottomrule
\end{tabular}
\end{table}

\subsubsection{Stromversorgung}

\begin{table}[H]
\centering
\caption{Stromaufnahme des Systems}
\label{tab:stromversorgung}
\begin{tabular}{@{}l r r@{}}
\toprule
\textbf{Komponente} & \textbf{Typisch} & \textbf{Maximum} \\
\midrule
ESP32-S3            & \SI{80}{\milli\ampere}  & \SI{500}{\milli\ampere} (WiFi aktiv) \\
CD4021B ($\times$2) & $<$\,\SI{1}{\milli\ampere} & \SI{1}{\milli\ampere} \\
74HC595 ($\times$2) & $<$\,\SI{1}{\milli\ampere} & \SI{70}{\milli\ampere} (alle Ausgänge) \\
LEDs ($\times$10 @ \SI{4}{\milli\ampere}) & \SI{40}{\milli\ampere} & \SI{40}{\milli\ampere} \\
\midrule
\textbf{Gesamt}     & \textbf{$\approx$\,\SI{130}{\milli\ampere}} & \textbf{$\approx$\,\SI{620}{\milli\ampere}} \\
\bottomrule
\end{tabular}
\end{table}

\begin{tipbox}[USB-Stromversorgung]
Die USB-CDC-Versorgung vom Pi liefert bis zu \SI{500}{\milli\ampere}. Bei mehr als
8 LEDs gleichzeitig oder höheren Strömen: Helligkeit per PWM reduzieren oder externe
\SI{5}{\volt}-Versorgung verwenden.
\end{tipbox}

% -----------------------------------------------------------------------------
\subsection{Hardware-Eigenheiten}
\label{subsec:hw-eigenheiten}

\subsubsection{First-Bit-Problem (CD4021B)}

Nach dem Parallel-Load liegt das erste Bit (PI-8 $\rightarrow$ Q8) sofort am Ausgang,
\textbf{bevor} der erste Clock kommt. Der ESP32 samplet aber erst \textbf{nach} der
ersten Clock-Flanke. Die Firmware löst dies durch einen \lstinline[style=arduino]|digitalRead()|
vor dem SPI-Transfer.

\subsubsection{DS-Pin des letzten CD4021B}

Der DS-Pin (Serial Data Input, \textbf{Pin 11}) des \textbf{letzten} CD4021B in der
Kette muss auf \SI{+3.3}{\volt} gelegt werden, nicht auf GND. Bei GND würden beim
Einlesen Nullen nachgeschoben, die als \enquote{gedrückt} fehlinterpretiert werden.

\subsubsection{SPI-Bus Crosstalk}

Wenn der ESP32 vom CD4021B liest, taktet er dabei Nullen durch den 74HC595. Die
Ausgänge ändern sich erst beim Latch-Impuls, aber ein glitchender RCK-Pin könnte
LEDs kurz ausschalten. Die Firmware kompensiert dies durch
\lstinline[style=arduino]|LED_REFRESH_EVERY_CYCLE = true|.

\subsubsection{Active-Low Taster}

Die Taster sind Active-Low beschaltet (gedrückt = 0, losgelassen = 1). Die Firmware
invertiert dies in \filep{bitops.h}, sodass die Logik-Schicht mit
\lstinline[style=arduino]|pressed = true| arbeitet.

\subsubsection{Stützkondensatoren}

Jeder IC benötigt einen \SI{100}{\nano\farad} Keramikkondensator zwischen VCC und
GND, möglichst nah am Chip. Ohne diese Kondensatoren können Störungen auf der
Versorgung zu Fehlfunktionen führen.

% -----------------------------------------------------------------------------
\subsection{Skalierung auf 100 Buttons}
\label{subsec:hw-skalierung}

Für das 100-Button-System verlängern wir die Daisy-Chain auf jeweils 13
Schieberegister:

\begin{table}[H]
\centering
\caption{Komponentenvergleich 10 vs.\ 100 Buttons}
\label{tab:skalierung-hardware}
\begin{tabular}{@{}l r r@{}}
\toprule
\textbf{Komponente} & \textbf{10-Button} & \textbf{100-Button} \\
\midrule
CD4021B       & 2   & 13 \\
74HC595       & 2   & 13 \\
Taster        & 10  & 100 \\
LEDs          & 10  & 100 \\
R (LED)       & 10  & 100 \\
R (Pull-up)   & 10  & 100 \\
C (\SI{100}{\nano\farad}) & 4 & 26 \\
\bottomrule
\end{tabular}
\end{table}

Die SPI-Transferzeit steigt von ca.\ \SI{20}{\micro\second} auf ca.\
\SI{260}{\micro\second} -- weit unter dem \SI{5}{\milli\second}-Zyklus der
Firmware.

% Version: 2.5.2 | Stand: 2026-01-11
% =============================================================================

\section{Hardware}
\label{sec:hardware}

Dieses Kapitel dokumentiert den 10-Button-Prototyp des Selection Panels. Die Schaltung
lässt sich auf 100 Taster skalieren, indem wir die Daisy-Chain der Schieberegister
verlängern.

% -----------------------------------------------------------------------------
\subsection{Komponentenübersicht}
\label{subsec:hw-komponenten}

\begin{table}[H]
\centering
\caption{Stückliste des 10-Button-Prototyps}
\label{tab:hw-komponenten}
\begin{tabularx}{\textwidth}{@{}l l r X@{}}
\toprule
\textbf{Komponente} & \textbf{Typ} & \textbf{Anzahl} & \textbf{Funktion} \\
\midrule
XIAO ESP32-S3       & Mikrocontroller         & 1  & Steuerlogik, USB-CDC \\
Raspberry Pi 5      & SBC + Netzteil + microSD & 1 & Server, Dashboard, Medien \\
CD4021B             & 8-Bit PISO Schieberegister & 2 & Taster einlesen \\
74HC595             & 8-Bit SIPO Schieberegister & 2 & LEDs ansteuern \\
Taster              & $6 \times 6$\,\si{\milli\meter} Tactile Switch & 10 & Benutzereingabe \\
LED                 & \SI{5}{\milli\meter}, versch.\ Farben & 10 & Statusanzeige \\
Widerstand          & \SIrange{330}{3000}{\ohm} & 10 & LED-Strombegrenzung \\
Widerstand          & \SI{10}{\kilo\ohm}       & 10 & Pull-up für Taster \\
Kondensator         & \SI{100}{\nano\farad} (Keramik) & 4 & Stützkondensatoren \\
\bottomrule
\end{tabularx}
\end{table}

% -----------------------------------------------------------------------------
\subsection{Pin-Belegung XIAO ESP32-S3}
\label{subsec:hw-pinout}

\begin{figure}[H]
\centering
\begin{lstlisting}[basicstyle=\ttfamily\footnotesize, frame=single, numbers=none]
                  +-------------+
                  |    USB-C    |
                  |             |
 LED_RCK    D0 ---|  1       14 |--- 5V
 BTN_PS     D1 ---|  2       13 |--- GND
 LED_OE     D2 ---|  3       12 |--- 3V3
            D3 ---|  4       11 |--- D10  LED_MOSI
            D4 ---|  5       10 |--- D9   BTN_MISO
            D5 ---|  6        9 |--- D8   SCK
            D6 ---|  7        8 |--- D7
                  |             |
                  +-------------+
\end{lstlisting}
\caption{Pin-Belegung des XIAO ESP32-S3}
\label{fig:xiao-pinout}
\end{figure}

\begin{table}[H]
\centering
\caption{GPIO-Zuordnung für SPI und Steuerung}
\label{tab:gpio-zuordnung}
\begin{tabularx}{\textwidth}{@{}l l l X@{}}
\toprule
\textbf{Pin} & \textbf{Signal} & \textbf{Funktion} & \textbf{Ziel} \\
\midrule
\pin{D0}  & \texttt{LED\_RCK}  & Latch (STCP)              & 74HC595 Pin 12 \\
\pin{D1}  & \texttt{BTN\_PS}   & Parallel/Serial Control   & CD4021B Pin 9 \\
\pin{D2}  & \texttt{LED\_OE}   & Output Enable (PWM, active-low) & 74HC595 Pin 13 \\
\pin{D8}  & \texttt{SCK}       & SPI Clock                 & Beide Chip-Typen \\
\pin{D9}  & \texttt{BTN\_MISO} & Daten von CD4021B         & CD4021B \#0 Pin 3 (Q8) \\
\pin{D10} & \texttt{LED\_MOSI} & Daten zu 74HC595          & 74HC595 \#0 Pin 14 (SER) \\
\bottomrule
\end{tabularx}
\end{table}

% -----------------------------------------------------------------------------
\subsection{Byte-Mapping (Firmware \texorpdfstring{$\leftrightarrow$}{<->} Hardware)}
\label{subsec:hw-byte-mapping}

Die Asymmetrie zwischen Ein- und Ausgabe ergibt sich aus der Hardware: Der CD4021B
schiebt sein erstes Sample (PI-8) als erstes Bit heraus, während der 74HC595 das
zuletzt empfangene Bit auf QA (Ausgang 0) legt.

\begin{table}[H]
\centering
\caption{Bit-Zuordnung der Schieberegister}
\label{tab:byte-mapping}
\begin{tabular}{@{}l l l@{}}
\toprule
\textbf{IC} & \textbf{Byte 0} & \textbf{Byte 1} \\
\midrule
74HC595 (LEDs)    & LED 1--8 = Bit 0--7 & LED 9--10 = Bit 0--1 \\
CD4021 (Taster)   & BTN 1--8 = Bit 7--0 & BTN 9--10 = Bit 7--6 \\
\bottomrule
\end{tabular}
\end{table}

\begin{infobox}[Firmware-Abstraktion]
Die Firmware verwendet in \filep{bitops.h} die Formel \lstinline[style=arduino]|btn_bit(id) = 7 - ((id - 1) % 8)|,
um diese Zuordnung zu abstrahieren. Damit arbeitet die Logik-Schicht einheitlich mit
Button-IDs von 1--100.
\end{infobox}

% -----------------------------------------------------------------------------
\subsection{Schaltplan-Übersicht}
\label{subsec:hw-schaltplan}

\begin{figure}[H]
\centering
\begin{lstlisting}[basicstyle=\ttfamily\scriptsize, frame=single, numbers=none]
                        +--------------------------------------------------+
                        |                  OUTPUT (LEDs)                   |
    +---------+         |                                                  |
    | Pi 5    |         |   +----------+  QH'     +----------+             |
    |         |  USB-C  |   | 74HC595  |--------->| 74HC595  |             |
    |         |<------->|   |   #0     |   SER    |   #1     |             |
    +---------+         |   |          |          |          |             |
                        |   | QA-QH    |          | QA-QB    |             |
        +---------+     |   |   |      |          |   |      |             |
        |  XIAO   |     |   | LED 1-8  |          | LED 9-10 |             |
        | ESP32S3 |     |   +----------+          +----------+             |
        |         |     +--------------------------------------------------+
        | D10 ----+----------> SER #0 (MOSI)
        | D8  ----+----------> SCK (beide)
        | D0  ----+----------> RCK (beide)
        | D2  ----+----------> OE (beide)
        |         |
        | D9  <---+---------- Q8 #0 (MISO)
        | D8  ----+----------> CLK (beide)
        | D1  ----+----------> P/S (beide)
        |         |     +--------------------------------------------------+
        +---------+     |                  INPUT (Taster)                  |
                        |                                                  |
                        |   +----------+          +----------+             |
                        |   | CD4021B  |<-- DS ---| CD4021B  |<-- DS -- 3V3|
                        |   |   #0     |    Q8    |   #1     |             |
                        |   |          |<---------|          |             |
                        |   | PI 8-1   |          | PI 8-7   |             |
                        |   |   |      |          |   |      |             |
                        |   | BTN 1-8  |          | BTN 9-10 |             |
                        |   +----------+          +----------+             |
                        |        |                                         |
                        |        v                                         |
                        |      Q8 --> D9 (MISO)                            |
                        +--------------------------------------------------+
\end{lstlisting}
\caption{Gesamtübersicht der Schaltung}
\label{fig:schaltplan-uebersicht}
\end{figure}

% -----------------------------------------------------------------------------
\subsection{74HC595 Detailschaltplan (LED-Ausgang)}
\label{subsec:hw-74hc595}

Der 74HC595 ist ein Serial-In/Parallel-Out Schieberegister mit Latch. Die Daten
werden seriell eingetaktet und erst beim Latch-Impuls (RCK) an die Ausgänge
übernommen.

\begin{figure}[H]
\centering
\begin{lstlisting}[basicstyle=\ttfamily\scriptsize, frame=single, numbers=none]
                     74HC595 (#0)                    74HC595 (#1)
                  +-------------+                 +-------------+
    LED 2  <------|  1 QB  VCC 16|--- 3V3 LED 10 <|  1 QB  VCC 16|--- 3V3
    LED 3  <------|  2 QC  QA  15|--> LED 1       |  2 QC  QA  15|--> LED 9
    LED 4  <------|  3 QD  SER 14|<-- D10 (MOSI)  |  3 QD  SER 14|<-- #0.QH'
    LED 5  <------|  4 QE  OE  13|<-- D2 (PWM)    |  4 QE  OE  13|<-- D2 (PWM)
    LED 6  <------|  5 QF  RCK 12|<-- D0 (RCK)    |  5 QF  RCK 12|<-- D0 (RCK)
    LED 7  <------|  6 QG  SCK 11|<-- D8 (SCK)    |  6 QG  SCK 11|<-- D8 (SCK)
    LED 8  <------|  7 QH  CLR 10|--- 3V3         |  7 QH  CLR 10|--- 3V3
          GND ----|  8 GND QH'  9|---> #1.SER     |  8 GND QH'  9|    (n.c.)
                  +------+------+                 +------+------+
                         |                               |
                        === C1                          === C2
                        100nF                           100nF
                         |                               |
                        GND                             GND
\end{lstlisting}
\caption{74HC595 Pinout und Daisy-Chain}
\label{fig:74hc595-detail}
\end{figure}

\begin{warnbox}[CLR-Pin nicht floaten!]
Der \pin{CLR}-Pin (Pin 10) muss auf \SI{3.3}{\volt} gelegt werden. Ein floatender
CLR-Pin führt zu sporadischem Löschen des Schieberegisters.
\end{warnbox}

\subsubsection{LED-Beschaltung (Active-High)}

\begin{figure}[H]
\centering
\begin{lstlisting}[basicstyle=\ttfamily\small, frame=single, numbers=none]
    Qx (74HC595)
     |
    +-+
    | | 330 - 3k Ohm
    +-+
     |
     v LED
     |
    GND
\end{lstlisting}
\caption{LED-Vorwiderstand-Beschaltung}
\label{fig:led-beschaltung}
\end{figure}

Die Stromberechnung für verschiedene Widerstandswerte bei einer LED-Flussspannung
von ca.\ \SI{2.0}{\volt}:

\begin{table}[H]
\centering
\caption{LED-Strom in Abhängigkeit vom Vorwiderstand}
\label{tab:led-strom}
\begin{tabular}{@{}r r l@{}}
\toprule
\textbf{Widerstand} & \textbf{Strom} & \textbf{Helligkeit} \\
\midrule
\SI{330}{\ohm}  & $\approx$\,\SI{4}{\milli\ampere}   & hell \\
\SI{1}{\kilo\ohm}   & $\approx$\,\SI{1.3}{\milli\ampere} & gedimmt \\
\SI{3}{\kilo\ohm}   & $\approx$\,\SI{0.4}{\milli\ampere} & schwach \\
\bottomrule
\end{tabular}
\end{table}

% -----------------------------------------------------------------------------
\subsection{CD4021B Detailschaltplan (Taster-Eingang)}
\label{subsec:hw-cd4021b}

Der CD4021B ist ein Parallel-In/Serial-Out Schieberegister. Er liest acht
parallele Eingänge ein und gibt sie seriell aus.

\subsubsection{Pinout}

\begin{figure}[H]
\centering
\begin{lstlisting}[basicstyle=\ttfamily\small, frame=single, numbers=none]
              +------------+
    PI-8  ----|  1  *  16  |---- VDD
      Q6  ----|  2     15  |---- PI-7
      Q8  ----|  3     14  |---- PI-6
    PI-4  ----|  4     13  |---- PI-5
    PI-3  ----|  5     12  |---- Q7
    PI-2  ----|  6     11  |---- SERIAL IN (DS)
    PI-1  ----|  7     10  |---- CLOCK
     VSS  ----|  8      9  |---- P/S CONTROL
              +------------+
\end{lstlisting}
\caption{CD4021B Pinout (DIP-16)}
\label{fig:cd4021b-pinout}
\end{figure}

\subsubsection{Beschaltung für 10 Taster}

\begin{figure}[H]
\centering
\begin{lstlisting}[basicstyle=\ttfamily\scriptsize, frame=single, numbers=none]
                     CD4021B (#0)                        CD4021B (#1)
                  +--------------+                    +--------------+
    BTN 1  ------|  1 PI-8 VDD 16|--- 3V3   BTN 9 ---|  1 PI-8 VDD 16|--- 3V3
                 |  2 Q6  PI-7 15|<-- BTN 2          |  2 Q6  PI-7 15|<-- BTN 10
   D9 (MISO) <---|  3 Q8  PI-6 14|<-- BTN 3          |  3 Q8  PI-6 14|    +3V3
    BTN 5  ------|  4 PI-4 PI-5 13|<-- BTN 4          |  4 PI-4 PI-5 13|    +3V3
    BTN 6  ------|  5 PI-3  Q7  12|    (n.c.)         |  5 PI-3  Q7  12|    (n.c.)
    BTN 7  ------|  6 PI-2  DS  11|<-- Q8 von #1      |  6 PI-2  DS  11|--- 3V3
    BTN 8  ------|  7 PI-1 CLK  10|<-- D8 (SCK)       |  7 PI-1 CLK  10|<-- D8 (SCK)
          GND ---|  8 VSS  P/S   9|<-- D1 (PS)        |  8 VSS  P/S   9|<-- D1 (PS)
                 +-------+-------+                   +-------+-------+
                         |       <----- Q8 (Pin 3) ------+
                        === C3                          === C4
                        100nF                           100nF
                         |                               |
                        GND                             GND
\end{lstlisting}
\caption{CD4021B Beschaltung mit Daisy-Chain}
\label{fig:cd4021b-beschaltung}
\end{figure}

\subsubsection{Pin-Zuordnung Taster → CD4021B}

Der CD4021B gibt Daten \highlight{MSB-first} aus: PI-8 erscheint zuerst (Bit 7),
PI-1 zuletzt (Bit 0).

\begin{table}[H]
\centering
\caption{Taster-zu-Bit-Zuordnung}
\label{tab:btn-mapping}
\begin{tabular}{@{}c c c c c@{}}
\toprule
\textbf{Taster} & \textbf{Chip} & \textbf{Pin-Name} & \textbf{Pin-Nr} & \textbf{Bit im Byte} \\
\midrule
BTN 1  & \#0 & PI-8 &  1 & Bit 7 \\
BTN 2  & \#0 & PI-7 & 15 & Bit 6 \\
BTN 3  & \#0 & PI-6 & 14 & Bit 5 \\
BTN 4  & \#0 & PI-5 & 13 & Bit 4 \\
BTN 5  & \#0 & PI-4 &  4 & Bit 3 \\
BTN 6  & \#0 & PI-3 &  5 & Bit 2 \\
BTN 7  & \#0 & PI-2 &  6 & Bit 1 \\
BTN 8  & \#0 & PI-1 &  7 & Bit 0 \\
\midrule
BTN 9  & \#1 & PI-8 &  1 & Bit 7 (Byte 1) \\
BTN 10 & \#1 & PI-7 & 15 & Bit 6 (Byte 1) \\
\bottomrule
\end{tabular}
\end{table}

\subsubsection{Daisy-Chain Datenfluss}

\begin{figure}[H]
\centering
\begin{lstlisting}[basicstyle=\ttfamily\small, frame=single, numbers=none]
3V3 --> DS [CD4021B #1] --> Q8 --> DS [CD4021B #0] --> Q8 --> D9 (MISO) --> ESP32
              |                          |
              |  BTN 9-10                |  BTN 1-8
              |  (Byte 1)                |  (Byte 0)
              |                          |
              +-- Spaeter gelesen -------+-- Zuerst gelesen -->
\end{lstlisting}
\caption{Datenfluss in der CD4021B Daisy-Chain}
\label{fig:daisy-chain-flow}
\end{figure}

\begin{warnbox}[DS-Pin des letzten Chips]
Der \pin{DS}-Pin (Pin 11) des \textbf{letzten} CD4021B muss auf \SI{+3.3}{\volt}
gelegt werden. Bei GND würden Nullen nachgeschoben, die als \enquote{gedrückt}
fehlinterpretiert werden. Ungenutzte PI-x Pins ebenfalls auf \SI{+3.3}{\volt}
legen -- interne Pull-ups reichen nicht zuverlässig.
\end{warnbox}

\subsubsection{Taster-Beschaltung (Active-Low)}

\begin{figure}[H]
\centering
\begin{lstlisting}[basicstyle=\ttfamily\small, frame=single, numbers=none]
    3V3
     |
    +-+
    | | 10 kOhm (Pull-up)
    +-+
     +--------- PIx (CD4021B)
     |
    +-+
    | | Taster
    +-+
     |
    GND

Zustand:
  - Losgelassen: PIx = HIGH (1) -> in Firmware: nicht gedrueckt
  - Gedrueckt:   PIx = LOW  (0) -> in Firmware: gedrueckt
\end{lstlisting}
\caption{Taster mit Pull-up-Widerstand}
\label{fig:taster-beschaltung}
\end{figure}

% -----------------------------------------------------------------------------
\subsection{SPI-Bus Verkabelung}
\label{subsec:hw-spi}

\begin{table}[H]
\centering
\caption{SPI-Signalführung}
\label{tab:spi-verkabelung}
\begin{tabularx}{\textwidth}{@{}l l X@{}}
\toprule
\textbf{ESP32-S3} & \textbf{74HC595} & \textbf{CD4021B} \\
\midrule
\pin{D10} (MOSI) & SER (Pin 14, \#0) $\rightarrow$ QH' $\rightarrow$ SER (\#1) & -- \\
\pin{D8} (SCK)   & SCK (Pin 11, beide) & CLK (Pin 10, beide) \\
\pin{D9} (MISO)  & -- & Q8 (Pin 3, \#0) $\leftarrow$ DS $\leftarrow$ Q8 (\#1) \\
\pin{D0} (RCK)   & RCK (Pin 12, beide) & -- \\
\pin{D1} (PS)    & -- & P/S (Pin 9, beide) \\
\pin{D2} (OE)    & OE (Pin 13, beide) & -- \\
\bottomrule
\end{tabularx}
\end{table}

% -----------------------------------------------------------------------------
\subsection{Timing-Diagramme}
\label{subsec:hw-timing}

\subsubsection{CD4021B Lesevorgang}

\begin{figure}[H]
\centering
\begin{lstlisting}[basicstyle=\ttfamily\small, frame=single, numbers=none]
    P/S -----+     +-------------------------
             |     |
             +-----+
                   ^
                   Parallel Load (HIGH->LOW->HIGH)

    CLK -------------+  +--+  +--+  +--+  +--
                     |  |  |  |  |  |  |  |
                     +--+  +--+  +--+  +--+
                        ^     ^     ^     ^
                       Bit6  Bit5  Bit4  Bit3 ...

    Q8  ============##=====##=====##=====##====
                    || Bit7|| Bit6|| Bit5|| ...
                    ##=====##=====##=====##====
                    ^
                    First-Bit (PI-8) liegt VOR erstem Clock an!
\end{lstlisting}
\caption{Timing des CD4021B Lesevorgangs}
\label{fig:timing-cd4021b}
\end{figure}

\subsubsection{74HC595 Schreibvorgang}

\begin{figure}[H]
\centering
\begin{lstlisting}[basicstyle=\ttfamily\small, frame=single, numbers=none]
    SER ----##=====##=====##=====##=====------
            || Bit7|| Bit6|| Bit5|| ...
            ##=====##=====##=====##=====------

    SCK ---------+  +--+  +--+  +--+  +-----
                 |  |  |  |  |  |  |  |
                 +--+  +--+  +--+  +--+
                    ^     ^     ^     ^
                   Daten bei steigender Flanke uebernommen

    RCK ------------------------------------+  +--
                                            |  |
                                            +--+
                                               ^
                                              Latch: Shift -> Output
\end{lstlisting}
\caption{Timing des 74HC595 Schreibvorgangs}
\label{fig:timing-74hc595}
\end{figure}

% -----------------------------------------------------------------------------
\subsection{Stromlimits}
\label{subsec:hw-stromlimits}

\begin{table}[H]
\centering
\caption{Elektrische Grenzwerte der ICs}
\label{tab:stromlimits}
\begin{tabular}{@{}l l r l@{}}
\toprule
\textbf{Komponente} & \textbf{Parameter} & \textbf{Wert} & \textbf{Bemerkung} \\
\midrule
\textbf{ESP32-S3} & GPIO Output (max)     & \SI{40}{\milli\ampere}   & Drive Strength 3 \\
                  & GPIO Output (default) & \SI{20}{\milli\ampere}   & Drive Strength 2 \\
                  & Gesamt alle GPIOs     & \SI{1200}{\milli\ampere} & Summe \\
\midrule
\textbf{74HC595}  & Output pro Pin (max)      & $\pm$\SI{35}{\milli\ampere} & Absolutes Maximum \\
                  & Output pro Pin (empfohlen)& $\pm$\SI{6}{\milli\ampere}  & Dauerbetrieb \\
                  & VCC/GND gesamt            & \SIrange{70}{75}{\milli\ampere} & \critical{Package-Limit!} \\
\midrule
\textbf{CD4021B}  & Input (pro Pin)  & $<$\,\SI{1}{\micro\ampere} & CMOS-Eingang \\
                  & Output Q8        & \SIrange{1}{3}{\milli\ampere} & Für SPI ausreichend \\
\bottomrule
\end{tabular}
\end{table}

\subsubsection{Stromversorgung}

\begin{table}[H]
\centering
\caption{Stromaufnahme des Systems}
\label{tab:stromversorgung}
\begin{tabular}{@{}l r r@{}}
\toprule
\textbf{Komponente} & \textbf{Typisch} & \textbf{Maximum} \\
\midrule
ESP32-S3            & \SI{80}{\milli\ampere}  & \SI{500}{\milli\ampere} (WiFi aktiv) \\
CD4021B ($\times$2) & $<$\,\SI{1}{\milli\ampere} & \SI{1}{\milli\ampere} \\
74HC595 ($\times$2) & $<$\,\SI{1}{\milli\ampere} & \SI{70}{\milli\ampere} (alle Ausgänge) \\
LEDs ($\times$10 @ \SI{4}{\milli\ampere}) & \SI{40}{\milli\ampere} & \SI{40}{\milli\ampere} \\
\midrule
\textbf{Gesamt}     & \textbf{$\approx$\,\SI{130}{\milli\ampere}} & \textbf{$\approx$\,\SI{620}{\milli\ampere}} \\
\bottomrule
\end{tabular}
\end{table}

\begin{tipbox}[USB-Stromversorgung]
Die USB-CDC-Versorgung vom Pi liefert bis zu \SI{500}{\milli\ampere}. Bei mehr als
8 LEDs gleichzeitig oder höheren Strömen: Helligkeit per PWM reduzieren oder externe
\SI{5}{\volt}-Versorgung verwenden.
\end{tipbox}

% -----------------------------------------------------------------------------
\subsection{Hardware-Eigenheiten}
\label{subsec:hw-eigenheiten}

\subsubsection{First-Bit-Problem (CD4021B)}

Nach dem Parallel-Load liegt das erste Bit (PI-8 $\rightarrow$ Q8) sofort am Ausgang,
\textbf{bevor} der erste Clock kommt. Der ESP32 samplet aber erst \textbf{nach} der
ersten Clock-Flanke. Die Firmware löst dies durch einen \lstinline[style=arduino]|digitalRead()|
vor dem SPI-Transfer.

\subsubsection{DS-Pin des letzten CD4021B}

Der DS-Pin (Serial Data Input, \textbf{Pin 11}) des \textbf{letzten} CD4021B in der
Kette muss auf \SI{+3.3}{\volt} gelegt werden, nicht auf GND. Bei GND würden beim
Einlesen Nullen nachgeschoben, die als \enquote{gedrückt} fehlinterpretiert werden.

\subsubsection{SPI-Bus Crosstalk}

Wenn der ESP32 vom CD4021B liest, taktet er dabei Nullen durch den 74HC595. Die
Ausgänge ändern sich erst beim Latch-Impuls, aber ein glitchender RCK-Pin könnte
LEDs kurz ausschalten. Die Firmware kompensiert dies durch
\lstinline[style=arduino]|LED_REFRESH_EVERY_CYCLE = true|.

\subsubsection{Active-Low Taster}

Die Taster sind Active-Low beschaltet (gedrückt = 0, losgelassen = 1). Die Firmware
invertiert dies in \filep{bitops.h}, sodass die Logik-Schicht mit
\lstinline[style=arduino]|pressed = true| arbeitet.

\subsubsection{Stützkondensatoren}

Jeder IC benötigt einen \SI{100}{\nano\farad} Keramikkondensator zwischen VCC und
GND, möglichst nah am Chip. Ohne diese Kondensatoren können Störungen auf der
Versorgung zu Fehlfunktionen führen.

% -----------------------------------------------------------------------------
\subsection{Skalierung auf 100 Buttons}
\label{subsec:hw-skalierung}

Für das 100-Button-System verlängern wir die Daisy-Chain auf jeweils 13
Schieberegister:

\begin{table}[H]
\centering
\caption{Komponentenvergleich 10 vs.\ 100 Buttons}
\label{tab:skalierung-hardware}
\begin{tabular}{@{}l r r@{}}
\toprule
\textbf{Komponente} & \textbf{10-Button} & \textbf{100-Button} \\
\midrule
CD4021B       & 2   & 13 \\
74HC595       & 2   & 13 \\
Taster        & 10  & 100 \\
LEDs          & 10  & 100 \\
R (LED)       & 10  & 100 \\
R (Pull-up)   & 10  & 100 \\
C (\SI{100}{\nano\farad}) & 4 & 26 \\
\bottomrule
\end{tabular}
\end{table}

Die SPI-Transferzeit steigt von ca.\ \SI{20}{\micro\second} auf ca.\
\SI{260}{\micro\second} -- weit unter dem \SI{5}{\milli\second}-Zyklus der
Firmware.
