% =============================================================================
% LOETEN.tex – Bleifreies Löten mit SAC305
% Modulares Fragment (kein \documentclass, kein \begin{document})
% Stand: 2025-12 | Version: 1.0
% =============================================================================

\section{Bleifreies Löten mit SAC305}
\label{sec:loeten}

Handlöten von \textbf{ESP32-Breakout-Boards} und \textbf{XT60-Steckern} mit bleifreiem Lot.

% -----------------------------------------------------------------------------
\subsection{Ausrüstung}
\label{subsec:loeten-ausruestung}

\subsubsection{Lötzinn}

\begin{table}[H]
  \centering
  \caption{Empfohlenes Lötzinn}
  \label{tab:loeten-loetzinn}
  \begin{tabularx}{\textwidth}{@{}l l c l r@{}}
    \toprule
    \textbf{Produkt} & \textbf{Legierung} & \textbf{Durchmesser} & \textbf{Reichelt-Nr.} & \textbf{Preis} \\
    \midrule
    Felder ISO-Core \glqq Clear\grqq & SAC305 (Sn96,5Ag3Cu0,5) & \SI{0,5}{\milli\meter} & LZ FE CSA 0,5 25 & € 45,99 \\
    \bottomrule
  \end{tabularx}
\end{table}

\begin{table}[H]
  \centering
  \caption{Lötzinn-Parameter}
  \label{tab:loeten-parameter}
  \begin{tabularx}{0.5\textwidth}{@{}l X@{}}
    \toprule
    \textbf{Parameter} & \textbf{Wert} \\
    \midrule
    Schmelzbereich & \SI{217}{\celsius}–\SI{219}{\celsius} \\
    Flussmittelanteil & 3,5\,\% (No-Clean) \\
    \bottomrule
  \end{tabularx}
\end{table}

\begin{tipbox}[0,5 mm für alles]
  THT, SMD, XT60 – bei großen Kontakten 5–6× zuführen.
\end{tipbox}

\subsubsection{Lötstationen}

\begin{table}[H]
  \centering
  \caption{Lötstationen}
  \label{tab:loeten-stationen}
  \begin{tabularx}{\textwidth}{@{}l c l c l@{}}
    \toprule
    \textbf{Gerät} & \textbf{Leistung} & \textbf{Einsatz} & \textbf{Temperatur} & \textbf{Spitze} \\
    \midrule
    Ersa i-CON PICO & \SI{68}{\watt} & THT, SMD & \SI{360}{\celsius} & 0102CDLF16 (\SI{1,6}{\milli\meter} Meißel) \\
    Ersa 150 S & \SI{150}{\watt} & XT60 & \SI{400}{\celsius} & 0152KD (\SI{5,3}{\milli\meter} Meißel) \\
    \bottomrule
  \end{tabularx}
\end{table}

\subsubsection{Zubehör}

\begin{table}[H]
  \centering
  \caption{Lötzubehör}
  \label{tab:loeten-zubehoer}
  \begin{tabularx}{\textwidth}{@{}l l l r@{}}
    \toprule
    \textbf{Produkt} & \textbf{Funktion} & \textbf{Reichelt-Nr.} & \textbf{Preis} \\
    \midrule
    Stannol X32-10i & Flussmittelstift & STANNOL X32-10I & € 6,99 \\
    ERSA Tip-Reactivator & Spitzen regenerieren & ERSA 0TR01 & $\sim$€ 13 \\
    \bottomrule
  \end{tabularx}
\end{table}

% -----------------------------------------------------------------------------
\subsection{Temperatur}
\label{subsec:loeten-temperatur}

Die Spitzentemperatur ergibt sich aus der Schmelztemperatur plus Aufschlag:

\begin{equation}
  T_{\text{Spitze}} = T_{\text{Schmelz}} + \SI{120}{\celsius}
\end{equation}

\begin{table}[H]
  \centering
  \caption{Arbeitstemperaturen}
  \label{tab:loeten-temperaturen}
  \begin{tabularx}{0.6\textwidth}{@{}l c@{}}
    \toprule
    \textbf{Arbeitsgang} & \textbf{Temperatur} \\
    \midrule
    SMD, Fein-Pitch & \SI{350}{\celsius}–\SI{360}{\celsius} \\
    THT, Pin-Header & \SI{360}{\celsius}–\SI{380}{\celsius} \\
    XT60, Litzen $\geq$ \SI{1,5}{\milli\meter\squared} & \SI{400}{\celsius} \\
    \bottomrule
  \end{tabularx}
\end{table}

\begin{table}[H]
  \centering
  \caption{Temperaturanpassungen}
  \label{tab:loeten-anpassungen}
  \begin{tabularx}{0.6\textwidth}{@{}l X@{}}
    \toprule
    \textbf{Situation} & \textbf{Anpassung} \\
    \midrule
    Große Masseflächen & +\SI{20}{\celsius}–\SI{30}{\celsius} \\
    Empfindliche Bauteile & $-$\SI{20}{\celsius} \\
    \bottomrule
  \end{tabularx}
\end{table}

% -----------------------------------------------------------------------------
\subsection{Technik}
\label{subsec:loeten-technik}

\subsubsection{Lötstelle in 5 Schritten}

\begin{enumerate}
  \item \textbf{Spitze verzinnen} → Wärmeübertragung verbessern
  \item \textbf{Spitze ansetzen} → An Pad UND Bauteilbein
  \item \textbf{Warten} (1–2 s) → Wärme übertragen
  \item \textbf{Lot zuführen} → Von der Gegenseite
  \item \textbf{Abziehen} → Erst Lot, dann Spitze
\end{enumerate}

\textbf{Gesamtzeit:} 2–4 Sekunden

\subsubsection{Entlöten}

\begin{table}[H]
  \centering
  \caption{Entlötmethoden}
  \label{tab:loeten-entloeten}
  \begin{tabularx}{0.6\textwidth}{@{}l X@{}}
    \toprule
    \textbf{Methode} & \textbf{Anwendung} \\
    \midrule
    Entlötlitze + Flux & SMD, kleine Mengen \\
    Entlötpumpe & THT, große Mengen \\
    \bottomrule
  \end{tabularx}
\end{table}

% -----------------------------------------------------------------------------
\subsection{ESP32 Breakout löten}
\label{subsec:loeten-esp32}

\begin{itemize}
  \item[$\square$] i-CON PICO auf \textbf{\SI{360}{\celsius}}
  \item[$\square$] Spitze 0102CDLF16 (\SI{1,6}{\milli\meter} Meißel)
  \item[$\square$] Platine fixieren
\end{itemize}

\begin{lstlisting}[style=shell,numbers=none]
Spitze verzinnen -> An Pad+Pin -> 1-2 s -> Lot zufuehren -> Abziehen
\end{lstlisting}

% -----------------------------------------------------------------------------
\subsection{XT60 Stecker löten}
\label{subsec:loeten-xt60}

\begin{itemize}
  \item[$\square$] Ersa 150 S auf \textbf{\SI{400}{\celsius}} (5 Min vorheizen)
  \item[$\square$] Spitze 0152KD (\SI{5,3}{\milli\meter} Meißel)
  \item[$\square$] Litze abisolieren (8–10 mm), verdrillen
\end{itemize}

\begin{lstlisting}[style=shell,numbers=none]
Huelse vorverzinnen -> Litze vorverzinnen -> Einfuehren ->
Spitze seitlich -> Warten (3-5 s) -> Nicht bewegen bis erstarrt
\end{lstlisting}

% -----------------------------------------------------------------------------
\subsection{Spitzenpflege}
\label{subsec:loeten-spitzenpflege}

\begin{table}[H]
  \centering
  \caption{Spitzenpflege}
  \label{tab:loeten-spitzenpflege}
  \begin{tabularx}{\textwidth}{@{}l X@{}}
    \toprule
    \textbf{Wann} & \textbf{Aktion} \\
    \midrule
    Vor jedem Löten & Messingwolle → frisch verzinnen \\
    Vor dem Ausschalten & \textbf{Niemals blank} – Zinn drauf \\
    Oxidiert (Lot perlt ab) & Tip-Reactivator bei \SI{250}{\celsius}–\SI{300}{\celsius} \\
    \bottomrule
  \end{tabularx}
\end{table}

% -----------------------------------------------------------------------------
\subsection{Qualitätskontrolle}
\label{subsec:loeten-qualitaet}

\begin{table}[H]
  \centering
  \caption{Qualitätsmerkmale}
  \label{tab:loeten-qualitaet}
  \begin{tabularx}{0.7\textwidth}{@{}l X@{}}
    \toprule
    \textbf{Gut} & \textbf{Mangelhaft} \\
    \midrule
    Glänzend bis seidenmatt & Rissig, körnig \\
    Konkaver Meniskus & Kugelig \\
    Vollständige Benetzung & Lot perlt ab \\
    \bottomrule
  \end{tabularx}
\end{table}

\begin{infobox}[Hinweis]
  SAC305 glänzt weniger als verbleites Lot. Seidenmatt ist normal.
\end{infobox}

% -----------------------------------------------------------------------------
\subsection{Sicherheit}
\label{subsec:loeten-sicherheit}

\begin{itemize}
  \item Dämpfe nicht einatmen – Absaugung verwenden
  \item Lötkolben nie unbeaufsichtigt
  \item Bleihaltiges und bleifreies Lot \textbf{nie mischen}
\end{itemize}

% -----------------------------------------------------------------------------
\subsection{Einkaufsliste}
\label{subsec:loeten-einkauf}

\begin{table}[H]
  \centering
  \caption{Einkaufsliste (Reichelt)}
  \label{tab:loeten-einkauf}
  \begin{tabularx}{0.7\textwidth}{@{}l l r@{}}
    \toprule
    \textbf{Artikel} & \textbf{Art.-Nr.} & \textbf{Preis} \\
    \midrule
    Lötzinn bleifrei \SI{0,5}{\milli\meter}, \SI{250}{\gram} & LZ FE CSA 0,5 25 & € 45,99 \\
    Flussmittelstift No-Clean & STANNOL X32-10I & € 6,99 \\
    \midrule
    \textbf{Gesamt} & & \textbf{€ 52,98} \\
    \bottomrule
  \end{tabularx}
\end{table}
