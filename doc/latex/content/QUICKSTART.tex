% =============================================================================
% QUICKSTART.tex – Server + Dashboard in 5 Minuten
% Modulares Fragment (kein \documentclass, kein \begin{document})
% Stand: 2026-01-08 | Version: 2.5.2
% =============================================================================

\section{Quickstart}
\label{sec:quickstart}

In diesem Kapitel bringen wir das Selection Panel zum Laufen – Server und Dashboard in 5 Minuten. Wir gehen davon aus, dass SSH eingerichtet, das Repository geklont und der ESP32 geflasht ist.

\begin{table}[H]
  \centering
  \caption{Quickstart-Metadaten}
  \label{tab:quickstart-meta}
  \begin{tabularx}{0.5\textwidth}{@{}l X@{}}
    \toprule
    \textbf{Metadaten} & \textbf{Wert} \\
    \midrule
    Stand & 2026-01-08 \\
    Version & 2.5.2 \\
    Status & ✓ Prototyp funktionsfähig \\
    \bottomrule
  \end{tabularx}
\end{table}

% -----------------------------------------------------------------------------
\subsection{Voraussetzungen}
\label{subsec:quickstart-voraussetzungen}

Bevor wir starten, prüfen wir kurz die Voraussetzungen:

\begin{itemize}
  \item[$\square$] SSH eingerichtet → \Cref{sec:ssh}
  \item[$\square$] Repository geklont → \Cref{sec:git}
  \item[$\square$] ESP32 geflasht → \Cref{sec:firmware-code-guide}
\end{itemize}

% -----------------------------------------------------------------------------
\subsection{Setup (einmalig)}
\label{subsec:quickstart-setup}

Zunächst erstellen wir eine virtuelle Python-Umgebung und installieren die Abhängigkeiten:

\begin{lstlisting}[style=shell]
ssh rover
cd ~/selection-panel

python3 -m venv venv
venv/bin/pip install -r requirements.txt
\end{lstlisting}

\begin{tipbox}[Minimale Abhängigkeiten]
  Nur \texttt{aiohttp} wird benötigt – kein pyserial mehr. Der Server nutzt direkte Dateizugriffe auf den Serial-Port.
\end{tipbox}

% -----------------------------------------------------------------------------
\subsection{Server starten}
\label{subsec:quickstart-server}

Jetzt starten wir den Server:

\begin{lstlisting}[style=shell]
cd ~/selection-panel
source venv/bin/activate
python server.py
\end{lstlisting}

Bei erfolgreichem Start sehen wir folgende Ausgabe:

\begin{lstlisting}[style=shell,numbers=none]
==================================================
Auswahlpanel Server v2.5.2 (PROTOTYPE)
==================================================
Medien: 10 erwartet (IDs: 001-010)
Taster: 1-10 (1-basiert)
Serial: /dev/serial/by-id/usb-Espressif...
HTTP:   http://0.0.0.0:8080/
ESP32 lokale LED: aktiviert
==================================================
Medien-Validierung: 10/10 vollstaendig
==================================================
Serial verbinde: /dev/serial/by-id/usb-Espressif...
Serial verbunden
\end{lstlisting}

Das Dashboard erreichen wir unter \texttt{http://rover:8080/}.

% -----------------------------------------------------------------------------
\subsection{Dashboard nutzen}
\label{subsec:quickstart-dashboard}

Das Dashboard führt uns durch den Startvorgang:

\begin{enumerate}
  \item Dashboard öffnen: \texttt{http://rover:8080/}
  \item \textbf{„Sound aktivieren"} Button klicken – wichtig für die Audio-Wiedergabe!
  \item Warten auf Preload: „Lade Medien... 5/10" → „Warte auf Tastendruck..."
  \item Taster drücken → Bild und Ton werden sofort abgespielt
\end{enumerate}

\begin{infobox}[Latenz-Verhalten]
  Die LED leuchtet sofort (< \SI{1}{\milli\second}), die Wiedergabe erfolgt aus dem Cache (< \SI{50}{\milli\second}). Das System fühlt sich instantan an.
\end{infobox}

% -----------------------------------------------------------------------------
\subsection{Testen ohne Hardware}
\label{subsec:quickstart-test}

Auch ohne angeschlossene Taster können wir das System testen. Die HTTP-API simuliert Tastendrücke:

\begin{lstlisting}[style=shell]
# Wiedergabe simulieren (1-basiert!)
curl http://rover:8080/test/play/1
curl http://rover:8080/test/play/5
curl http://rover:8080/test/play/10

# Status abfragen
curl http://rover:8080/status | jq

# Health-Check
curl http://rover:8080/health | jq

# Wiedergabe stoppen
curl http://rover:8080/test/stop
\end{lstlisting}

% -----------------------------------------------------------------------------
\subsection{Serial direkt testen}
\label{subsec:quickstart-serial}

Für die Fehlersuche ist es hilfreich, die Serial-Kommunikation direkt zu beobachten:

\begin{lstlisting}[style=shell]
# Stabilen Port ermitteln
SERIAL_PORT=$(ls /dev/serial/by-id/usb-Espressif* 2>/dev/null | head -1)
echo "Port: $SERIAL_PORT"

# Port konfigurieren
stty -F $SERIAL_PORT 115200 raw -echo

# Daten empfangen (Ctrl+C zum Beenden)
cat $SERIAL_PORT
\end{lstlisting}

Befehle können wir direkt an den ESP32 senden:

\begin{lstlisting}[style=shell]
echo "PING" > $SERIAL_PORT
echo "STATUS" > $SERIAL_PORT
echo "LEDSET 001" > $SERIAL_PORT
echo "LEDCLR" > $SERIAL_PORT
\end{lstlisting}

% -----------------------------------------------------------------------------
\subsection{Autostart einrichten}
\label{subsec:quickstart-autostart}

Damit der Server nach einem Reboot automatisch startet, installieren wir den systemd-Service:

\begin{lstlisting}[style=shell]
sudo cp selection-panel.service /etc/systemd/system/
sudo systemctl daemon-reload
sudo systemctl enable --now selection-panel.service
\end{lstlisting}

Status und Logs prüfen wir mit:

\begin{lstlisting}[style=shell]
sudo systemctl status selection-panel
journalctl -u selection-panel -f
\end{lstlisting}

% -----------------------------------------------------------------------------
\subsection{Troubleshooting}
\label{subsec:quickstart-troubleshooting}

\Cref{tab:quickstart-troubleshooting} listet die häufigsten Probleme und deren Lösungen.

\begin{table}[H]
  \centering
  \caption{Häufige Probleme und Lösungen}
  \label{tab:quickstart-troubleshooting}
  \begin{tabularx}{\textwidth}{@{}l X@{}}
    \toprule
    \textbf{Problem} & \textbf{Lösung} \\
    \midrule
    \texttt{ModuleNotFoundError} & \texttt{venv/bin/pip install aiohttp} \\
    \texttt{Permission denied: /dev/ttyACM0} & \texttt{sudo usermod -aG dialout \$USER} → Neu einloggen \\
    Port blockiert & \texttt{sudo fuser /dev/ttyACM0} → Prozess beenden \\
    Kein Ton & „Sound aktivieren"-Button im Browser klicken \\
    Server startet nicht & \texttt{journalctl -u selection-panel -f} \\
    Taster nicht erkannt & Serial testen: \texttt{cat /dev/serial/by-id/usb-Espressif*} \\
    Falsche Medien & Prüfen: \texttt{ls media/} (001.jpg bis 010.jpg) \\
    Preload dauert lange & Medien komprimieren oder Concurrency erhöhen \\
    \bottomrule
  \end{tabularx}
\end{table}

% -----------------------------------------------------------------------------
\subsection{Medien-Struktur}
\label{subsec:quickstart-medien}

Die Medien folgen der 1-basierten Nummerierung mit Zero-Padding auf 3 Stellen:

\begin{lstlisting}[style=shell,numbers=none]
media/
|-- 001.jpg  001.mp3
|-- 002.jpg  002.mp3
|-- ...
+-- 010.jpg  010.mp3
\end{lstlisting}

Für Tests generieren wir Platzhalter-Medien mit:

\begin{lstlisting}[style=shell]
./scripts/generate_test_media.sh 10
\end{lstlisting}

% -----------------------------------------------------------------------------
\subsection{Latenz-Budget}
\label{subsec:quickstart-latenz}

\begin{table}[H]
  \centering
  \caption{Latenz-Budget vom Tastendruck bis zur Wiedergabe}
  \label{tab:quickstart-latenz}
  \begin{tabularx}{0.6\textwidth}{@{}l r@{}}
    \toprule
    \textbf{Komponente} & \textbf{Latenz} \\
    \midrule
    ESP32 LED & < \SI{1}{\milli\second} \\
    Serial + Server & $\sim$\SI{10}{\milli\second} \\
    Dashboard (aus Cache) & < \SI{50}{\milli\second} \\
    \midrule
    \textbf{Gesamt} & \textbf{< \SI{70}{\milli\second}} \\
    \bottomrule
  \end{tabularx}
\end{table}

% -----------------------------------------------------------------------------
\subsection{Referenz-System}
\label{subsec:quickstart-referenz}

\begin{table}[H]
  \centering
  \caption{Referenz-System für diese Dokumentation}
  \label{tab:quickstart-referenz}
  \begin{tabularx}{0.6\textwidth}{@{}l X@{}}
    \toprule
    \textbf{Komponente} & \textbf{Version} \\
    \midrule
    Raspberry Pi 5 & \SI{4}{\giga\byte} RAM \\
    Pi OS & Debian 13 (trixie) \\
    Python & 3.13+ \\
    aiohttp & 3.9+ \\
    ESP32 Firmware & 2.5.2 \\
    Server & 2.5.2 \\
    Dashboard & 2.5.1 \\
    \bottomrule
  \end{tabularx}
\end{table}

% -----------------------------------------------------------------------------
\subsection{Schnellreferenz}
\label{subsec:quickstart-schnellreferenz}

\Cref{tab:quickstart-schnellreferenz} fasst die wichtigsten Befehle zusammen.

\begin{table}[H]
  \centering
  \caption{Befehls-Schnellreferenz}
  \label{tab:quickstart-schnellreferenz}
  \begin{tabularx}{\textwidth}{@{}l X@{}}
    \toprule
    \textbf{Aktion} & \textbf{Befehl} \\
    \midrule
    Server starten & \texttt{python server.py} \\
    Dashboard & \texttt{http://rover:8080/} \\
    Status & \texttt{curl http://rover:8080/status} \\
    Health & \texttt{curl http://rover:8080/health} \\
    Test Play & \texttt{curl http://rover:8080/test/play/5} \\
    Serial Monitor & \texttt{cat /dev/serial/by-id/usb-Espressif*} \\
    Service Status & \texttt{sudo systemctl status selection-panel} \\
    Service Logs & \texttt{journalctl -u selection-panel -f} \\
    \bottomrule
  \end{tabularx}
\end{table}
