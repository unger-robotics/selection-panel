% =============================================================================
% ROADMAP.tex – Implementierungsplan
% Modulares Fragment (kein \documentclass, kein \begin{document})
% Stand: 2026-01-08 | Version: 2.5.2
% =============================================================================

\section{Roadmap – Implementierungsplan}
\label{sec:roadmap}

Phasen und Status. Details: siehe \Cref{sec:spec}, Tests: siehe \Cref{sec:runbook}.

\begin{table}[H]
  \centering
  \caption{Roadmap-Metadaten}
  \label{tab:roadmap-meta}
  \begin{tabularx}{0.4\textwidth}{@{}l X@{}}
    \toprule
    \textbf{Metadaten} & \textbf{Wert} \\
    \midrule
    Stand & 2026-01-08 \\
    Version & 2.5.2 \\
    \bottomrule
  \end{tabularx}
\end{table}

% -----------------------------------------------------------------------------
\subsection{Übersicht}
\label{subsec:roadmap-uebersicht}

\begin{lstlisting}[style=shell,numbers=none]
Phase 1: Infrastruktur        [x]
Phase 2: Hardware-Prototyp    [x]
Phase 3: ESP32 Firmware       [x]
Phase 4: Raspberry Pi Server  [x]
Phase 5: Web-Dashboard        [x]
Phase 6: Integration & Test   [x] Prototyp (10x)
Phase 7: Pi-Integration       [x] Serial + WebSocket
Phase 8: Skalierung (100x)    <-- naechste Phase
Phase 9: Produktivbetrieb     [ ]
\end{lstlisting}

% -----------------------------------------------------------------------------
\subsection{Phase 1–5: Abgeschlossen}
\label{subsec:roadmap-phase1-5}

\begin{table}[H]
  \centering
  \caption{Meilensteine Phase 1–5}
  \label{tab:roadmap-phase1-5}
  \begin{tabularx}{0.7\textwidth}{@{}c l c@{}}
    \toprule
    \textbf{Phase} & \textbf{Meilenstein} & \textbf{Status} \\
    \midrule
    1 & Toolchain funktioniert & ✓ \\
    2 & Taster → LED reagiert & ✓ \\
    3 & \texttt{PRESS} → \texttt{LEDSET} funktioniert & ✓ \\
    4 & WebSocket-Verbindung steht & ✓ \\
    5 & Browser zeigt Bild + Audio & ✓ \\
    \bottomrule
  \end{tabularx}
\end{table}

% -----------------------------------------------------------------------------
\subsection{Phase 6: Integration (Prototyp)}
\label{subsec:roadmap-phase6}

\subsubsection{Hardware Prototyp}

\begin{table}[H]
  \centering
  \caption{Hardware-Aufgaben Phase 6}
  \label{tab:roadmap-hw-phase6}
  \begin{tabularx}{0.6\textwidth}{@{}X c@{}}
    \toprule
    \textbf{Aufgabe} & \textbf{Status} \\
    \midrule
    2× CD4021B kaskadiert & ✓ \\
    2× 74HC595 kaskadiert & ✓ \\
    10 Taster verdrahtet & ✓ \\
    10 LEDs verdrahtet & ✓ \\
    Bit-Mapping implementiert & ✓ \\
    \bottomrule
  \end{tabularx}
\end{table}

\subsubsection{Software}

\begin{table}[H]
  \centering
  \caption{Software-Aufgaben Phase 6}
  \label{tab:roadmap-sw-phase6}
  \begin{tabularx}{0.7\textwidth}{@{}X c@{}}
    \toprule
    \textbf{Aufgabe} & \textbf{Status} \\
    \midrule
    Firmware v2.5.2 (FreeRTOS, Hardware-SPI) & ✓ \\
    Server v2.5.2 (asyncio, by-id Pfad) & ✓ \\
    Dashboard v2.5.1 (Preloading) & ✓ \\
    1-basierte Nummerierung & ✓ \\
    \bottomrule
  \end{tabularx}
\end{table}

\subsubsection{Tests}

\begin{table}[H]
  \centering
  \caption{Tests Phase 6}
  \label{tab:roadmap-tests-phase6}
  \begin{tabularx}{0.6\textwidth}{@{}X c@{}}
    \toprule
    \textbf{Test} & \textbf{Status} \\
    \midrule
    Alle 10 Taster erkannt & ✓ \\
    Zuordnung Taster → Medien & ✓ \\
    Preempt (curl) & ✓ \\
    Ende → LEDCLR & ✓ \\
    WebSocket Broadcast & ✓ \\
    LED-Latenz < \SI{1}{\milli\second} & ✓ \\
    Dashboard-Latenz < \SI{50}{\milli\second} & ✓ \\
    \bottomrule
  \end{tabularx}
\end{table}

\textbf{Meilenstein M6:} ✓ Prototyp mit 10 Tastern voll funktionsfähig.

% -----------------------------------------------------------------------------
\subsection{Phase 7: Pi-Integration}
\label{subsec:roadmap-phase7}

\subsubsection{Architektur}

\begin{table}[H]
  \centering
  \caption{Architektur-Aufgaben Phase 7}
  \label{tab:roadmap-arch-phase7}
  \begin{tabularx}{0.7\textwidth}{@{}X c@{}}
    \toprule
    \textbf{Aufgabe} & \textbf{Status} \\
    \midrule
    ESP32 als reiner I/O-Controller & ✓ \\
    Pi als Anwendungslogik-Host & ✓ \\
    Serial-Protokoll dokumentiert & ✓ \\
    WebSocket-Protokoll dokumentiert & ✓ \\
    by-id Serial-Pfad (stabil) & ✓ \\
    \bottomrule
  \end{tabularx}
\end{table}

\subsubsection{Firmware}

\begin{table}[H]
  \centering
  \caption{Firmware-Aufgaben Phase 7}
  \label{tab:roadmap-fw-phase7}
  \begin{tabularx}{0.7\textwidth}{@{}X c@{}}
    \toprule
    \textbf{Aufgabe} & \textbf{Status} \\
    \midrule
    FreeRTOS Dual-Core & ✓ \\
    Hardware-SPI (shared Bus) & ✓ \\
    First-Bit-Rescue (CD4021B) & ✓ \\
    SPI-Modi dokumentiert & ✓ \\
    Zeitbasiertes Debouncing & ✓ \\
    \bottomrule
  \end{tabularx}
\end{table}

\subsubsection{Server}

\begin{table}[H]
  \centering
  \caption{Server-Aufgaben Phase 7}
  \label{tab:roadmap-server-phase7}
  \begin{tabularx}{0.6\textwidth}{@{}X c@{}}
    \toprule
    \textbf{Aufgabe} & \textbf{Status} \\
    \midrule
    aiohttp WebSocket & ✓ \\
    Serial-Thread mit asyncio Bridge & ✓ \\
    Medien-Validierung & ✓ \\
    Health-Check Endpoint & ✓ \\
    systemd-Service & ✓ \\
    \bottomrule
  \end{tabularx}
\end{table}

\textbf{Meilenstein M7:} ✓ Pi-Integration mit Serial + WebSocket funktionsfähig.

% -----------------------------------------------------------------------------
\subsection{Phase 8: Skalierung (100×)}
\label{subsec:roadmap-phase8}

\begin{warnbox}[Nächste Phase]
  Phase 8 ist die nächste geplante Entwicklungsphase.
\end{warnbox}

\subsubsection{Hardware}

\begin{table}[H]
  \centering
  \caption{Hardware-Aufgaben Phase 8}
  \label{tab:roadmap-hw-phase8}
  \begin{tabularx}{0.6\textwidth}{@{}X c@{}}
    \toprule
    \textbf{Aufgabe} & \textbf{Status} \\
    \midrule
    13× CD4021B kaskadieren & ◯ \\
    13× 74HC595 kaskadieren & ◯ \\
    100 Taster verdrahten & ◯ \\
    100 LEDs verdrahten & ◯ \\
    PCB-Design (KiCad) & ◯ \\
    Lineare Verdrahtung & ◯ \\
    \bottomrule
  \end{tabularx}
\end{table}

\subsubsection{Software}

\begin{table}[H]
  \centering
  \caption{Software-Aufgaben Phase 8}
  \label{tab:roadmap-sw-phase8}
  \begin{tabularx}{0.7\textwidth}{@{}X c@{}}
    \toprule
    \textbf{Aufgabe} & \textbf{Status} \\
    \midrule
    \texttt{PROTOTYPE\_MODE = False} & ◯ \\
    \texttt{BTN\_COUNT = 100}, \texttt{LED\_COUNT = 100} & ◯ \\
    100 Medien-Sets & ◯ \\
    \bottomrule
  \end{tabularx}
\end{table}

\textbf{Meilenstein M8:} System mit 100 Tastern funktionsfähig.

% -----------------------------------------------------------------------------
\subsection{Phase 9: Produktivbetrieb}
\label{subsec:roadmap-phase9}

\begin{table}[H]
  \centering
  \caption{Aufgaben Phase 9}
  \label{tab:roadmap-phase9}
  \begin{tabularx}{0.6\textwidth}{@{}X c@{}}
    \toprule
    \textbf{Aufgabe} & \textbf{Status} \\
    \midrule
    systemd-Service aktiviert & ◯ \\
    Reboot-Test & ◯ \\
    Dauertest & ◯ \\
    Gehäuse & ◯ \\
    Dokumentation finalisiert & ◯ \\
    \bottomrule
  \end{tabularx}
\end{table}

\textbf{Meilenstein M9:} System läuft produktiv.

% -----------------------------------------------------------------------------
\subsection{Gelöste Probleme}
\label{subsec:roadmap-geloest}

\begin{table}[H]
  \centering
  \caption{Gelöste Probleme}
  \label{tab:roadmap-geloest}
  \begin{tabularx}{\textwidth}{@{}l X l@{}}
    \toprule
    \textbf{Problem} & \textbf{Lösung} & \textbf{Version} \\
    \midrule
    LED-Latenz durch Roundtrip & ESP32 setzt LED lokal (< \SI{1}{\milli\second}) & FW 2.5.2 \\
    Dashboard-Latenz & Medien-Preloading + Cache & Dashboard 2.5.1 \\
    Sequentielle Server-Aktionen & \texttt{asyncio.gather()} für Parallelität & Server 2.5.2 \\
    iOS Audio-Unlock fehlgeschlagen & AudioContext API + Fallback & Dashboard 2.5.1 \\
    ESP32-S3 USB-CDC fragmentiert & \texttt{Serial.flush()} nach jedem Event & FW 2.5.2 \\
    pyserial funktioniert nicht & \texttt{os.open} + \texttt{stty} statt pyserial & Server 2.5.2 \\
    CD4021B First-Bit-Problem & \texttt{digitalRead()} vor SPI & FW 2.5.2 \\
    CD4021B Timing-Probleme & Längere Load-Pulse (\SI{2}{\micro\second}) & FW 2.5.2 \\
    Serial-Pfad instabil & by-id Pfad verwenden & Server 2.5.2 \\
    \bottomrule
  \end{tabularx}
\end{table}

% -----------------------------------------------------------------------------
\subsection{Zeitschätzung}
\label{subsec:roadmap-zeit}

\begin{table}[H]
  \centering
  \caption{Zeitschätzung nach Phasen}
  \label{tab:roadmap-zeit}
  \begin{tabularx}{0.5\textwidth}{@{}c c c@{}}
    \toprule
    \textbf{Phase} & \textbf{Aufwand} & \textbf{Status} \\
    \midrule
    1–5 & 6–9 Tage & ✓ \\
    6 & 3–4 Tage & ✓ \\
    7 & 2–3 Tage & ✓ \\
    8 & 2–3 Tage & ◯ \\
    9 & 0,5 Tage & ◯ \\
    \bottomrule
  \end{tabularx}
\end{table}

% -----------------------------------------------------------------------------
\subsection{Nächste Schritte}
\label{subsec:roadmap-naechste}

\begin{enumerate}
  \item[$\square$] Schaltplan (KiCad)
  \item[$\square$] Lineare Verdrahtung (kein Bit-Mapping nötig)
  \item[$\square$] 100 Medien-Sets erstellen
  \item[$\square$] End-to-End Tests (100×)
  \item[$\square$] Gehäuse-Design
\end{enumerate}

% -----------------------------------------------------------------------------
\subsection{Versionen}
\label{subsec:roadmap-versionen}

\begin{table}[H]
  \centering
  \caption{Versionsübersicht}
  \label{tab:roadmap-versionen}
  \begin{tabularx}{0.7\textwidth}{@{}l c c@{}}
    \toprule
    \textbf{Komponente} & \textbf{Prototyp (aktuell)} & \textbf{Produktion (geplant)} \\
    \midrule
    Firmware & 2.5.2 & 3.0.0 \\
    Server & 2.5.2 & 3.0.0 \\
    Dashboard & 2.5.1 & 3.0.0 \\
    Hardware & 2× ICs & 26× ICs \\
    Taster & 10 & 100 \\
    Medien & 001–010 & 001–100 \\
    \bottomrule
  \end{tabularx}
\end{table}
