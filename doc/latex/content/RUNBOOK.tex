% =============================================================================
% RUNBOOK.tex – Betrieb und Troubleshooting
% Modulares Fragment (kein \documentclass, kein \begin{document})
% Stand: 2026-01-31 | Version: 2.5.3
% =============================================================================

\section{Runbook – Betrieb und Troubleshooting}
\label{sec:runbook}

Debugging-Checklisten und Testprozeduren. Befehle: siehe \Cref{sec:commands}.

\begin{table}[H]
  \centering
  \caption{Runbook-Metadaten}
  \label{tab:runbook-meta}
  \begin{tabularx}{0.4\textwidth}{@{}l X@{}}
    \toprule
    \textbf{Metadaten} & \textbf{Wert} \\
    \midrule
    Stand & 2026-01-31 \\
    Version & 2.5.3 \\
    \bottomrule
  \end{tabularx}
\end{table}

% -----------------------------------------------------------------------------
\subsection{Schnellstart}
\label{subsec:runbook-schnellstart}

\begin{lstlisting}[style=shell]
open http://rover:8080/              # Dashboard
curl http://rover:8080/status | jq   # Status
curl http://rover:8080/health        # Health (200/503)
curl http://rover:8080/test/play/5   # Test (1-basiert!)
\end{lstlisting}

% -----------------------------------------------------------------------------
\subsection{Hardware-Debugging}
\label{subsec:runbook-hardware}

\subsubsection{CD4021B (Buttons)}

\begin{table}[H]
  \centering
  \caption{CD4021B Troubleshooting}
  \label{tab:runbook-cd4021}
  \begin{tabularx}{\textwidth}{@{}l l X@{}}
    \toprule
    \textbf{Symptom} & \textbf{Ursache} & \textbf{Lösung} \\
    \midrule
    Keine Events & P/S-Logik invertiert & HIGH = Load, LOW = Shift \\
    Falsche Bits & Kaskadierung falsch & Q8 → DS prüfen \\
    Instabil & Fehlende Kondensatoren & \SI{100}{\nano\farad} an VDD/VSS \\
    Zufällige Trigger & DS floatet (letzter IC) & Pin 11 → VCC \\
    Falsche Taster-Zuordnung & Verdrahtung & BTN 1 → PI-8 (Pin 1) \\
    First-Bit fehlt & SPI-Timing & First-Bit-Rescue in Firmware \\
    \bottomrule
  \end{tabularx}
\end{table}

\subsubsection{74HC595 (LEDs)}

\begin{table}[H]
  \centering
  \caption{74HC595 Troubleshooting}
  \label{tab:runbook-hc595}
  \begin{tabularx}{\textwidth}{@{}l l X@{}}
    \toprule
    \textbf{Symptom} & \textbf{Ursache} & \textbf{Lösung} \\
    \midrule
    Keine LEDs & OE nicht auf GND/PWM & Pin 13 → D2 oder GND \\
    Alle LEDs an & SRCLR auf LOW & Pin 10 → VCC \\
    Falsche LED & Kaskadierung & QH' → SER prüfen \\
    \bottomrule
  \end{tabularx}
\end{table}

\subsubsection{ESP32-S3 XIAO}

\begin{table}[H]
  \centering
  \caption{ESP32-S3 Troubleshooting}
  \label{tab:runbook-esp32}
  \begin{tabularx}{\textwidth}{@{}l l X@{}}
    \toprule
    \textbf{Symptom} & \textbf{Ursache} & \textbf{Lösung} \\
    \midrule
    Keine Antwort & USB-Port belegt & Server stoppen: \texttt{sudo systemctl stop selection-panel} \\
    Kein Upload & Falscher Port & \texttt{ls /dev/serial/by-id/usb-Espressif*} \\
    Reboot-Schleife & Watchdog & Firmware prüfen \\
    Fragmentierte Serial-Daten & USB-CDC Timing & Firmware v2.5.3 mit \texttt{Serial.flush()} \\
    \bottomrule
  \end{tabularx}
\end{table}

% -----------------------------------------------------------------------------
\subsection{Serial-Debugging}
\label{subsec:runbook-serial}

\subsubsection{Port ermitteln (stabil)}

\begin{lstlisting}[style=shell]
# Stabilen by-id Pfad verwenden (empfohlen)
SERIAL_PORT=$(ls /dev/serial/by-id/usb-Espressif* 2>/dev/null | head -1)
echo "Port: $SERIAL_PORT"

# Port konfigurieren
stty -F $SERIAL_PORT 115200 raw -echo

# Daten empfangen (Ctrl+C zum Beenden)
cat $SERIAL_PORT

# Erwartete Ausgabe bei Tastendruck:
# PRESS 001
# RELEASE 001
\end{lstlisting}

\subsubsection{Befehle senden}

\begin{lstlisting}[style=shell]
echo "PING" > $SERIAL_PORT      # -> PONG
echo "STATUS" > $SERIAL_PORT    # -> active=N leds=0xNN
echo "VERSION" > $SERIAL_PORT   # -> FW SelectionPanel v2.5.3
echo "LEDSET 001" > $SERIAL_PORT  # LED 1 ein
echo "LEDCLR" > $SERIAL_PORT    # Alle LEDs aus
\end{lstlisting}

\subsubsection{STATUS-Ausgabe (v2.5.3)}

\begin{lstlisting}[style=shell,numbers=none]
STATUS active=5 leds=0x10
LEDS 0000100000      <- Bit-Vektor (MSB links)
BTNS 1111111111      <- Active-Low: 1 = losgelassen
\end{lstlisting}

\subsubsection{Häufige Serial-Probleme}

\begin{table}[H]
  \centering
  \caption{Serial-Probleme}
  \label{tab:runbook-serial-probleme}
  \begin{tabularx}{\textwidth}{@{}l l X@{}}
    \toprule
    \textbf{Symptom} & \textbf{Ursache} & \textbf{Lösung} \\
    \midrule
    \texttt{Permission denied} & Fehlende Rechte & \texttt{sudo usermod -aG dialout \$USER} → Neu einloggen \\
    \texttt{Device busy} & Port belegt & \texttt{sudo fuser /dev/ttyACM0} → Prozess beenden \\
    Fragmentierte Daten & USB-CDC & Firmware v2.5.3 verwenden \\
    Keine Daten & Falscher Port & \texttt{ls /dev/serial/by-id/usb-Espressif*} prüfen \\
    Port ändert sich & Instabiler Pfad & by-id Pfad verwenden \\
    \bottomrule
  \end{tabularx}
\end{table}

% -----------------------------------------------------------------------------
\subsection{Browser-Debugging}
\label{subsec:runbook-browser}

\subsubsection{Kein Ton}

\begin{enumerate}
  \item \glqq Sound aktivieren\grqq-Button klicken (Autoplay-Sperre)
  \item Warten: \glqq Lade Medien... 5/10\grqq{} → \glqq Warte auf Tastendruck...\grqq
  \item Audio-Status muss \textbf{grün} werden
  \item DevTools → Console auf Fehler prüfen
\end{enumerate}

\begin{infobox}[iOS/Safari]
  Falls Unlock fehlschlägt, Safari komplett schließen und neu öffnen. Dashboard v2.5.1 verwendet AudioContext API mit Preloading.
\end{infobox}

\subsubsection{Debug-Panel}

Button unten rechts oder \texttt{Ctrl+D}:

\begin{lstlisting}[style=shell,numbers=none]
Preloading 10 Medien...
Preload abgeschlossen: 10/10 OK (1823ms)
RX: {"type": "play", "id": 5}
Bild aus Cache: 005 (instant)
Audio aus Cache gestartet: 005 (12ms)
Audio beendet: 5
TX: {"type":"ended","id":5}
\end{lstlisting}

\subsubsection{Shortcuts}

\begin{table}[H]
  \centering
  \caption{Dashboard-Shortcuts}
  \label{tab:runbook-shortcuts}
  \begin{tabularx}{0.4\textwidth}{@{}l X@{}}
    \toprule
    \textbf{Taste} & \textbf{Funktion} \\
    \midrule
    \texttt{Space} & Play/Pause \\
    \texttt{Ctrl+D} & Debug-Panel \\
    \bottomrule
  \end{tabularx}
\end{table}

% -----------------------------------------------------------------------------
\subsection{Server-Debugging}
\label{subsec:runbook-server}

\begin{lstlisting}[style=shell]
# Logs live
journalctl -u selection-panel -f

# Letzte 10 Minuten
journalctl -u selection-panel --since "10 min ago"

# Service-Status
sudo systemctl status selection-panel

# Server manuell starten (fuer Debugging)
cd ~/selection-panel
source venv/bin/activate
python server.py
\end{lstlisting}

\subsubsection{Häufige Server-Probleme}

\begin{table}[H]
  \centering
  \caption{Server-Probleme}
  \label{tab:runbook-server-probleme}
  \begin{tabularx}{\textwidth}{@{}l l X@{}}
    \toprule
    \textbf{Symptom} & \textbf{Ursache} & \textbf{Lösung} \\
    \midrule
    \glqq Serial verbinde...\grqq{} hängt & Port belegt & \texttt{sudo fuser /dev/ttyACM0} \\
    Taster nicht erkannt & Serial-Fragment & Parser prüfen, Firmware v2.5.3 \\
    Medien fehlen & Falscher Pfad & \texttt{ls media/} (001.jpg - 010.jpg) \\
    WebSocket-Fehler & Client-Absturz & Browser neu laden \\
    Health 503 & Serial getrennt & ESP32 Verbindung prüfen \\
    \bottomrule
  \end{tabularx}
\end{table}

% -----------------------------------------------------------------------------
\subsection{End-to-End Tests}
\label{subsec:runbook-e2e}

\subsubsection{Latenz-Test}

\begin{enumerate}
  \item Button drücken
  \item \textbf{LED-Reaktion:} < \SI{1}{\milli\second} (lokal auf ESP32)
  \item \textbf{Dashboard-Wiedergabe:} < \SI{50}{\milli\second} (aus Cache)
  \item \textbf{Gesamt:} < \SI{70}{\milli\second}
\end{enumerate}

\subsubsection{Preempt-Test}

\begin{enumerate}
  \item Button 5 drücken → LED 5 an, Audio läuft
  \item Nach 1 s: Button 3 drücken
  \item \textbf{Erwartet:} Audio 5 stoppt sofort, LED 3 an, Audio 3 läuft
\end{enumerate}

\subsubsection{One-hot Test}

\begin{lstlisting}[style=shell]
curl http://rover:8080/test/play/5
curl http://rover:8080/test/play/3
# Erwartet: Nur LED 3 leuchtet
\end{lstlisting}

\subsubsection{Ende-Test}

\begin{enumerate}
  \item Audio abspielen lassen
  \item Warten bis Ende
  \item \textbf{Erwartet:} Alle LEDs aus
\end{enumerate}

\subsubsection{Race-Condition Test}

\begin{enumerate}
  \item Button 3 drücken, Audio läuft
  \item Kurz vor Ende: Button 5 drücken
  \item Audio 3 endet
  \item \textbf{Erwartet:} LED 5 bleibt an (nicht aus!)
\end{enumerate}

\subsubsection{Stresstest}

\begin{lstlisting}[style=shell]
for i in {1..200}; do
    curl -s "http://rover:8080/test/play/$((RANDOM % 10 + 1))"
    sleep 0.2
done
\end{lstlisting}

% -----------------------------------------------------------------------------
\subsection{Deployment-Checkliste}
\label{subsec:runbook-deployment}

\subsubsection{Hardware}

\begin{itemize}
  \item[$\square$] ESP32 mit Firmware v2.5.3 geflasht
  \item[$\square$] 10 Taster verdrahtet und funktionsfähig
  \item[$\square$] 10 LEDs verdrahtet und funktionsfähig
  \item[$\square$] USB-Kabel ESP32 ↔ Pi verbunden (Daten, nicht nur Laden!)
\end{itemize}

\subsubsection{Software}

\begin{itemize}
  \item[$\square$] Repository auf Pi geklont
  \item[$\square$] venv erstellt: \texttt{python3 -m venv venv}
  \item[$\square$] Pakete installiert: \texttt{pip install aiohttp}
  \item[$\square$] server.py vorhanden (v2.5.3)
\end{itemize}

\subsubsection{Medien}

\begin{itemize}
  \item[$\square$] 10 Bilder: \texttt{media/001.jpg} - \texttt{media/010.jpg}
  \item[$\square$] 10 Audio: \texttt{media/001.mp3} - \texttt{media/010.mp3}
\end{itemize}

\subsubsection{Tests}

\begin{itemize}
  \item[$\square$] Serial-Test: \texttt{cat /dev/serial/by-id/usb-Espressif*} zeigt PRESS/RELEASE
  \item[$\square$] Server startet ohne Fehler
  \item[$\square$] Dashboard öffnet sich
  \item[$\square$] \glqq Sound aktivieren\grqq{} funktioniert
  \item[$\square$] Preload abgeschlossen (\glqq 10/10 OK\grqq)
  \item[$\square$] Alle 10 Taster erkannt
  \item[$\square$] Zuordnung Taster → Medien korrekt
  \item[$\square$] LED-Reaktion < \SI{1}{\milli\second} ✓
  \item[$\square$] Dashboard-Latenz < \SI{50}{\milli\second} ✓
  \item[$\square$] Preempt-Test bestanden
\end{itemize}

\subsubsection{Autostart (optional)}

\begin{itemize}
  \item[$\square$] systemd-Service kopiert
  \item[$\square$] Service aktiviert: \texttt{sudo systemctl enable selection-panel}
  \item[$\square$] Reboot-Test bestanden
\end{itemize}

% -----------------------------------------------------------------------------
\subsection{Nützliche Einzeiler}
\label{subsec:runbook-einzeiler}

\begin{lstlisting}[style=shell]
# Medien zaehlen
ls ~/selection-panel/media/*.jpg 2>/dev/null | wc -l
ls ~/selection-panel/media/*.mp3 2>/dev/null | wc -l

# Service-Status
systemctl is-active selection-panel

# Health-Check
curl -s http://rover:8080/health | jq -r '.status'

# Port-Nutzung pruefen
sudo fuser /dev/ttyACM0

# USB-Geraete
lsusb | grep -i espressif

# Serial-Port finden (stabil)
ls /dev/serial/by-id/usb-Espressif*

# Firewall oeffnen (falls noetig)
sudo ufw allow 8080/tcp

# Server-Version
grep -m1 "VERSION" ~/selection-panel/server.py

# Firmware-Version (ueber Serial)
SERIAL_PORT=$(ls /dev/serial/by-id/usb-Espressif* | head -1)
echo "VERSION" > $SERIAL_PORT && sleep 0.5 && cat $SERIAL_PORT
\end{lstlisting}

% -----------------------------------------------------------------------------
\subsection{Bekannte Probleme und Lösungen}
\label{subsec:runbook-bekannt}

\begin{table}[H]
  \centering
  \caption{Bekannte Probleme}
  \label{tab:runbook-bekannt}
  \begin{tabularx}{\textwidth}{@{}X c l@{}}
    \toprule
    \textbf{Problem} & \textbf{Status} & \textbf{Lösung} \\
    \midrule
    iOS Audio-Unlock fehlgeschlagen & ✓ & Dashboard v2.5.1 mit AudioContext API \\
    ESP32-S3 USB-CDC fragmentiert & ✓ & Firmware v2.5.3 mit \texttt{Serial.flush()} \\
    pyserial funktioniert nicht & ✓ & Server v2.5.3 mit \texttt{os.open} \\
    CD4021B First-Bit-Problem & ✓ & First-Bit-Rescue in Firmware \\
    CD4021B Timing & ✓ & \SI{2}{\micro\second} Load-Pulse \\
    0-basiert vs 1-basiert & ✓ & Durchgängig 1-basiert \\
    LED-Latenz durch Roundtrip & ✓ & Firmware v2.5.3 (lokal < \SI{1}{\milli\second}) \\
    Dashboard-Latenz & ✓ & Dashboard v2.5.1 (Preload < \SI{50}{\milli\second}) \\
    Serial-Pfad instabil & ✓ & by-id Pfad verwenden \\
    \bottomrule
  \end{tabularx}
\end{table}
