% =============================================================================
% SPEC.tex – Spezifikation Auswahlpanel (Single Source of Truth)
% Modulares Fragment (kein \documentclass, kein \begin{document})
% Stand: 2026-01-08 | Version: 2.5.2
% =============================================================================

\section{Spezifikation (SPEC)}
\label{sec:spec}

Dieses Kapitel bildet die \textbf{Single Source of Truth} für Protokolle, Pinbelegung und Policies des Auswahlpanels. Wenn wir uns fragen, wie etwas funktionieren soll, finden wir hier die verbindliche Antwort.

\begin{table}[H]
  \centering
  \caption{Metadaten der Spezifikation}
  \label{tab:spec-meta}
  \begin{tabularx}{0.5\textwidth}{@{}l X@{}}
    \toprule
    \textbf{Metadaten} & \textbf{Wert} \\
    \midrule
    Version & 2.5.2 \\
    Datum & 2026-01-08 \\
    Status & ✓ Prototyp funktionsfähig (10×) \\
    \bottomrule
  \end{tabularx}
\end{table}

% -----------------------------------------------------------------------------
\subsection{Glossar}
\label{subsec:spec-glossar}

Bevor wir in die Details einsteigen, klären wir die wichtigsten Begriffe. \Cref{tab:spec-glossar} dient als Nachschlagewerk.

\begin{table}[H]
  \centering
  \caption{Begriffsdefinitionen}
  \label{tab:spec-glossar}
  \begin{tabularx}{\textwidth}{@{}l X@{}}
    \toprule
    \textbf{Begriff} & \textbf{Erklärung} \\
    \midrule
    One-hot & Genau ein Bit ist aktiv, alle anderen sind aus \\
    Preempt & Neue Aktion unterbricht sofort die laufende \\
    Race-Condition & Timing-Problem, wenn zwei Ereignisse fast gleichzeitig auftreten \\
    FreeRTOS & Echtzeit-Betriebssystem für Mikrocontroller \\
    Mutex & Sperre, die gleichzeitigen Zugriff auf Ressourcen verhindert \\
    CMOS & Chip-Technologie – Eingänge nie unbeschaltet lassen \\
    Pull-Up & Widerstand zieht Signal auf HIGH, Taster zieht auf LOW \\
    Kaskadierung & ICs in Reihe schalten, um mehr Ein-/Ausgänge zu erhalten \\
    DIP-16 & IC-Gehäuse mit 16 Pins in zwei Reihen \\
    USB-CDC & USB Communications Device Class (virtueller COM-Port) \\
    1-basiert & Nummerierung beginnt bei 1 (nicht 0) \\
    Preloading & Medien vorladen bevor sie benötigt werden \\
    \bottomrule
  \end{tabularx}
\end{table}

% -----------------------------------------------------------------------------
\subsection{Policy}
\label{subsec:spec-policy}

Zwei zentrale Regeln bestimmen das Verhalten des Systems: One-hot LED und Preempt.

\subsubsection{One-hot LED}

Zu jedem Zeitpunkt leuchtet \textbf{maximal eine LED}. Diese Einschränkung hat einen praktischen Grund: Der Maximalstrom beträgt so nur $1 \times \SI{20}{\milli\ampere}$ statt $100 \times \SI{20}{\milli\ampere} = \SI{2}{\ampere}$.

\begin{table}[H]
  \centering
  \caption{LED-Befehle im One-hot-Modus}
  \label{tab:spec-onehot}
  \begin{tabularx}{0.7\textwidth}{@{}l X@{}}
    \toprule
    \textbf{Befehl} & \textbf{Wirkung} \\
    \midrule
    \texttt{LEDSET n} & LED \textit{n} an, alle anderen aus \\
    \texttt{LEDCLR} & Alle LEDs aus \\
    \bottomrule
  \end{tabularx}
\end{table}

\subsubsection{Preempt („Umschalten gewinnt")}

Jeder neue Tastendruck unterbricht sofort die aktuelle Wiedergabe. Der Ablauf mit ESP32 v2.5.2 (lokale LED-Steuerung):

\begin{enumerate}
  \item ESP32 setzt LED sofort (< \SI{1}{\milli\second})
  \item ESP32 → Pi: \texttt{PRESS 005}
  \item Pi → Browser: \texttt{\{"type":"stop"\}} + \texttt{\{"type":"play","id":5\}} (parallel)
  \item Browser spielt aus Cache (< \SI{50}{\milli\second})
\end{enumerate}

\begin{infobox}[Race-Condition-Schutz]
  Nach Audio-Ende meldet der Browser \texttt{\{"type":"ended","id":5\}}. Der Pi sendet \texttt{LEDCLR} \textbf{nur wenn} \texttt{id == current\_id}. So bleibt die LED an, falls zwischenzeitlich ein neuer Taster gedrückt wurde.
\end{infobox}

% -----------------------------------------------------------------------------
\subsection{Nummerierung (1-basiert)}
\label{subsec:spec-nummerierung}

\textbf{Alle IDs sind 1-basiert} (001–100), nicht 0-basiert (000–099). Diese Konvention zieht sich durch alle Schichten – keine Konvertierung nötig.

\begin{table}[H]
  \centering
  \caption{Durchgängig 1-basierte Nummerierung}
  \label{tab:spec-nummerierung}
  \begin{tabularx}{\textwidth}{@{}l l X@{}}
    \toprule
    \textbf{Schicht} & \textbf{Format} & \textbf{Beispiel} \\
    \midrule
    Taster (physisch) & 1–100 & Taster 1, Taster 10 \\
    Serial-Protokoll & 001–100 & \texttt{PRESS 001}, \texttt{LEDSET 010} \\
    WebSocket & 1–100 & \texttt{\{"type":"play","id":1\}} \\
    Medien-Dateien & 001–100 & \filep{001.jpg}, \filep{010.mp3} \\
    Dashboard-Anzeige & 001–100 & „001", „010" \\
    \bottomrule
  \end{tabularx}
\end{table}

% -----------------------------------------------------------------------------
\subsection{Pinbelegung ESP32-S3 XIAO}
\label{subsec:spec-pinbelegung}

\Cref{tab:spec-pins} zeigt die Verbindungen zwischen ESP32 und den Schieberegistern.

\begin{table}[H]
  \centering
  \caption{ESP32-S3 XIAO Pinbelegung}
  \label{tab:spec-pins}
  \begin{tabularx}{\textwidth}{@{}l l l X@{}}
    \toprule
    \textbf{Signal} & \textbf{Pin} & \textbf{Ziel-IC} & \textbf{Funktion} \\
    \midrule
    MOSI & \pin{D10} & 74HC595 Pin 14 (SER) & Serielle Daten für LEDs \\
    SCK & \pin{D8} & 74HC595 Pin 11 + CD4021B Pin 10 & Gemeinsamer SPI-Takt \\
    RCK & \pin{D0} & 74HC595 Pin 12 (RCLK) & LED-Latch (Ausgabe freigeben) \\
    OE & \pin{D2} & 74HC595 Pin 13 (OE) & Output Enable (PWM für Helligkeit) \\
    MISO & \pin{D9} & CD4021B Pin 3 (Q8) & Serielle Daten von Tastern \\
    P/S & \pin{D1} & CD4021B Pin 9 (P/S) & Parallel Load (HIGH = Load) \\
    \bottomrule
  \end{tabularx}
\end{table}

\begin{warnbox}[SPI-Modi beachten]
  Die beiden ICs verwenden unterschiedliche SPI-Modi:
  \begin{itemize}
    \item \textbf{74HC595:} MODE0 (CPOL=0, CPHA=0) @ \SI{1}{\mega\hertz}
    \item \textbf{CD4021B:} MODE1 (CPOL=0, CPHA=1) @ \SI{500}{\kilo\hertz}
  \end{itemize}
\end{warnbox}

% -----------------------------------------------------------------------------
\subsection{CD4021B vs. 74HC165}
\label{subsec:spec-cd4021b}

Warum nutzen wir den CD4021B statt des häufiger verwendeten 74HC165? \Cref{tab:spec-ic-vergleich} zeigt die Unterschiede.

\begin{table}[H]
  \centering
  \caption{Vergleich der Eingabe-Schieberegister}
  \label{tab:spec-ic-vergleich}
  \begin{tabularx}{\textwidth}{@{}l X X@{}}
    \toprule
    \textbf{Aspekt} & \textbf{74HC165} & \textbf{CD4021B} \\
    \midrule
    Load-Signal & LOW-aktiv & \textbf{HIGH-aktiv} \\
    Shift-Signal & HIGH-aktiv & LOW-aktiv \\
    DIP-Verfügbarkeit & Schwer & Gut \\
    Load-Puls & \SI{1}{\micro\second} & \textbf{\SI{2}{\micro\second}} (CMOS) \\
    Clock-Puls & \SI{1}{\micro\second} & \SI{1}{\micro\second} \\
    \bottomrule
  \end{tabularx}
\end{table}

\begin{infobox}[Firmware-Anpassung]
  Die invertierte Load-Logik und längeren Pulse des CD4021B sind in der Firmware berücksichtigt. Bei einem Wechsel zum 74HC165 müssten diese Parameter angepasst werden.
\end{infobox}

% -----------------------------------------------------------------------------
\subsection{Verdrahtungsregeln}
\label{subsec:spec-verdrahtung}

CMOS-ICs vertragen keine offenen Eingänge – das führt zu undefiniertem Verhalten und erhöhtem Stromverbrauch. \Cref{tab:spec-verdrahtung} fasst die wichtigsten Regeln zusammen.

\begin{table}[H]
  \centering
  \caption{Verdrahtungsregeln für die Schieberegister}
  \label{tab:spec-verdrahtung}
  \begin{tabularx}{\textwidth}{@{}l l l X@{}}
    \toprule
    \textbf{IC} & \textbf{Pin} & \textbf{Regel} & \textbf{Grund} \\
    \midrule
    CD4021B (letzter) & Pin 11 (DS) & → VCC & CMOS-Eingänge nie floaten \\
    74HC595 (letzter) & Pin 9 (QH') & offen OK & Ausgang treibt aktiv \\
    74HC595 (alle) & Pin 10 (SRCLR) & → VCC & Clear deaktiviert \\
    74HC595 (alle) & Pin 13 (OE) & → \pin{D2} oder GND & Outputs aktiviert/PWM \\
    Alle ICs & VCC/VDD & \SI{100}{\nano\farad} → GND & Abblockkondensator \\
    \bottomrule
  \end{tabularx}
\end{table}

\subsubsection{Bit-Mapping}

Die Hardware-Verdrahtung bestimmt, welcher Taster welchem Bit entspricht.

\textbf{CD4021B (Eingabe):} BTN 1 → PI-8 (Pin 1), BTN 8 → PI-1 (Pin 7)

\begin{equation}
  \text{btn\_bit}(id) = (id - 1) \mod 8
\end{equation}

\textbf{74HC595 (Ausgabe):} LED 1 → QA, LED 8 → QH

\begin{equation}
  \text{led\_bit}(id) = (id - 1) \mod 8
\end{equation}

% -----------------------------------------------------------------------------
\subsection{Serial-Protokoll (ESP32 ↔ Pi)}
\label{subsec:spec-serial}

Die Kommunikation erfolgt mit \SI{115200}{\baud}, ASCII-kodiert und Newline-terminiert (\texttt{\textbackslash n}).

\begin{tipbox}[Stabiler Device-Pfad]
  Statt \filep{/dev/ttyACM0} (kann sich ändern) verwenden wir:\\
  \filep{/dev/serial/by-id/usb-Espressif*}
\end{tipbox}

\subsubsection{ESP32 → Pi}

\begin{table}[H]
  \centering
  \caption{Nachrichten vom ESP32 zum Pi}
  \label{tab:spec-serial-esp-pi}
  \begin{tabularx}{\textwidth}{@{}l X@{}}
    \toprule
    \textbf{Nachricht} & \textbf{Bedeutung} \\
    \midrule
    \texttt{READY} & Controller bereit \\
    \texttt{FW SelectionPanel v2.5.2} & Firmware-Version \\
    \texttt{PRESS 001} & Taster 1 gedrückt (001–100) \\
    \texttt{RELEASE 001} & Taster 1 losgelassen \\
    \texttt{OK} & Befehl erfolgreich \\
    \texttt{PONG} & Antwort auf PING \\
    \texttt{ERROR msg} & Fehlermeldung \\
    \bottomrule
  \end{tabularx}
\end{table}

\subsubsection{Pi → ESP32}

\begin{table}[H]
  \centering
  \caption{Befehle vom Pi zum ESP32}
  \label{tab:spec-serial-pi-esp}
  \begin{tabularx}{\textwidth}{@{}l X@{}}
    \toprule
    \textbf{Befehl} & \textbf{Funktion} \\
    \midrule
    \texttt{LEDSET 001} & LED 1 ein (one-hot) \\
    \texttt{LEDON 001} & LED 1 ein (additiv) \\
    \texttt{LEDOFF 001} & LED 1 aus \\
    \texttt{LEDCLR} & Alle LEDs aus \\
    \texttt{LEDALL} & Alle LEDs ein \\
    \texttt{PING} & Verbindungstest → PONG \\
    \texttt{STATUS} & Zustand abfragen \\
    \texttt{VERSION} & Firmware-Version \\
    \texttt{HELP} & Befehlsliste \\
    \bottomrule
  \end{tabularx}
\end{table}

\subsubsection{STATUS-Ausgabe (v2.5.2)}

\begin{lstlisting}[style=shell,numbers=none]
STATUS active=5 leds=0x10
LEDS 0000100000      # Bit-Vektor (MSB links)
BTNS 1111111111      # Bit-Vektor (Active-Low: 1 = losgelassen)
\end{lstlisting}

\subsubsection{USB-CDC Besonderheit}

Der ESP32-S3 XIAO nutzt USB-CDC statt UART. Ohne explizites \texttt{flush()} fragmentieren die Nachrichten:

\begin{lstlisting}[style=arduino,caption={Serial.flush() verhindert Fragmentierung}]
Serial.println(buffer);
Serial.flush();  // Wichtig bei USB-CDC!
\end{lstlisting}

% -----------------------------------------------------------------------------
\subsection{WebSocket-Protokoll (Pi ↔ Browser)}
\label{subsec:spec-websocket}

Endpoint: \texttt{ws://rover:8080/ws}

\begin{table}[H]
  \centering
  \caption{WebSocket-Nachrichten}
  \label{tab:spec-websocket}
  \begin{tabularx}{\textwidth}{@{}l l X@{}}
    \toprule
    \textbf{Richtung} & \textbf{Nachricht} & \textbf{Bedeutung} \\
    \midrule
    Pi → Browser & \texttt{\{"type":"stop"\}} & Wiedergabe stoppen \\
    Pi → Browser & \texttt{\{"type":"play","id":5\}} & Medien 5 abspielen \\
    Browser → Pi & \texttt{\{"type":"ended","id":5\}} & Audio 5 beendet \\
    Browser → Pi & \texttt{\{"type":"ping"\}} & Heartbeat \\
    \bottomrule
  \end{tabularx}
\end{table}

% -----------------------------------------------------------------------------
\subsection{HTTP-Endpoints}
\label{subsec:spec-http}

\begin{table}[H]
  \centering
  \caption{HTTP-Endpoints des Servers}
  \label{tab:spec-http}
  \begin{tabularx}{\textwidth}{@{}l X@{}}
    \toprule
    \textbf{Pfad} & \textbf{Funktion} \\
    \midrule
    \texttt{/} & Dashboard (index.html) \\
    \texttt{/ws} & WebSocket-Verbindung \\
    \texttt{/static/*} & CSS, JavaScript \\
    \texttt{/media/*} & Bilder und Audio \\
    \texttt{/test/play/\{id\}} & Wiedergabe simulieren (1-basiert) \\
    \texttt{/test/stop} & Wiedergabe stoppen \\
    \texttt{/status} & Server-Status (JSON) \\
    \texttt{/health} & Health-Check (200/503) \\
    \bottomrule
  \end{tabularx}
\end{table}

\subsubsection{Status-Response (v2.5.2)}

\begin{lstlisting}[style=json]
{
  "version": "2.5.2",
  "mode": "prototype",
  "num_media": 10,
  "current_button": 5,
  "ws_clients": 1,
  "serial_connected": true,
  "serial_port": "/dev/serial/by-id/usb-Espressif...",
  "media_missing": 0,
  "esp32_local_led": true
}
\end{lstlisting}

% -----------------------------------------------------------------------------
\subsection{Medien-Konvention}
\label{subsec:spec-medien}

\begin{table}[H]
  \centering
  \caption{Zuordnung von IDs zu Mediendateien}
  \label{tab:spec-medien}
  \begin{tabularx}{0.7\textwidth}{@{}r l l@{}}
    \toprule
    \textbf{ID} & \textbf{Bild} & \textbf{Audio} \\
    \midrule
    1 & \filep{media/001.jpg} & \filep{media/001.mp3} \\
    5 & \filep{media/005.jpg} & \filep{media/005.mp3} \\
    10 & \filep{media/010.jpg} & \filep{media/010.mp3} \\
    100 & \filep{media/100.jpg} & \filep{media/100.mp3} \\
    \bottomrule
  \end{tabularx}
\end{table}

\begin{infobox}[Dateinamen-Konvention]
  IDs: 1–100 (1-basiert), Dateien: 001–100 (zero-padded, 3 Stellen).
\end{infobox}

% -----------------------------------------------------------------------------
\subsection{Latenz-Budget}
\label{subsec:spec-latenz}

Wie schnell reagiert das System auf einen Tastendruck? \Cref{tab:spec-latenz} schlüsselt die einzelnen Komponenten auf.

\begin{table}[H]
  \centering
  \caption{Latenz-Budget vom Tastendruck bis zur Wiedergabe}
  \label{tab:spec-latenz}
  \begin{tabularx}{\textwidth}{@{}l r X@{}}
    \toprule
    \textbf{Komponente} & \textbf{Latenz} & \textbf{Beschreibung} \\
    \midrule
    ESP32 LED & < \SI{1}{\milli\second} & Lokale Steuerung (v2.5.2) \\
    Serial & $\sim$\SI{5}{\milli\second} & USB-CDC Übertragung \\
    Server & $\sim$\SI{1}{\milli\second} & asyncio.gather \\
    WebSocket & $\sim$\SI{5}{\milli\second} & Netzwerk \\
    Dashboard & < \SI{50}{\milli\second} & Aus Cache (Preloading) \\
    \midrule
    \textbf{Gesamt} & \textbf{< \SI{70}{\milli\second}} & Tastendruck → Wiedergabe \\
    \bottomrule
  \end{tabularx}
\end{table}

% -----------------------------------------------------------------------------
\subsection{Akzeptanztests}
\label{subsec:spec-tests}

\begin{table}[H]
  \centering
  \caption{Akzeptanztests für den Prototyp}
  \label{tab:spec-tests}
  \begin{tabularx}{\textwidth}{@{}l X c@{}}
    \toprule
    \textbf{Test} & \textbf{Erwartung} & \textbf{Status} \\
    \midrule
    Preempt & Neuer Taster unterbricht sofort & ✓ \\
    One-hot & Nur eine LED leuchtet & ✓ \\
    Ende & Nach Audio: alle LEDs aus & ✓ \\
    Race & LED bleibt an wenn neue ID aktiv & ✓ \\
    Debounce & Nur ein Event pro Tastendruck & ✓ \\
    Zuordnung & Taster 1 → Medien 001 & ✓ \\
    Alle Taster & 10/10 erkannt (Prototyp) & ✓ \\
    LED-Latenz & < \SI{1}{\milli\second} & ✓ \\
    Dashboard-Latenz & < \SI{50}{\milli\second} & ✓ \\
    \bottomrule
  \end{tabularx}
\end{table}

% -----------------------------------------------------------------------------
\subsection{Versionen}
\label{subsec:spec-versionen}

\begin{table}[H]
  \centering
  \caption{Aktuelle Komponenten-Versionen}
  \label{tab:spec-versionen}
  \begin{tabularx}{\textwidth}{@{}l l X@{}}
    \toprule
    \textbf{Komponente} & \textbf{Version} & \textbf{Änderung} \\
    \midrule
    Firmware & 2.5.2 & FreeRTOS, Hardware-SPI, First-Bit-Rescue \\
    Server & 2.5.2 & by-id Pfad, asyncio, ESP32\_SETS\_LED\_LOCALLY \\
    Dashboard & 2.5.1 & Preloading, Cache, Audio-Unlock \\
    \bottomrule
  \end{tabularx}
\end{table}

% -----------------------------------------------------------------------------
\subsection{Bekannte Einschränkungen}
\label{subsec:spec-einschraenkungen}

\Cref{tab:spec-einschraenkungen} dokumentiert bekannte Probleme und deren Lösungen.

\begin{table}[H]
  \centering
  \caption{Bekannte Einschränkungen und Lösungen}
  \label{tab:spec-einschraenkungen}
  \begin{tabularx}{\textwidth}{@{}X c X@{}}
    \toprule
    \textbf{Problem} & \textbf{Status} & \textbf{Lösung} \\
    \midrule
    ESP32-S3 USB-CDC fragmentiert & ✓ & \texttt{Serial.flush()} \\
    pyserial funktioniert nicht & ✓ & \texttt{os.open} + \texttt{stty} \\
    CD4021B braucht längere Pulse & ✓ & \SI{2}{\micro\second} Load, \SI{1}{\micro\second} Clock \\
    CD4021B First-Bit-Problem & ✓ & \texttt{digitalRead()} vor SPI \\
    LED-Latenz durch Roundtrip & ✓ & Lokale LED-Steuerung \\
    Dashboard-Latenz & ✓ & Medien-Preloading \\
    Serial-Pfad instabil & ✓ & by-id Pfad verwenden \\
    \bottomrule
  \end{tabularx}
\end{table}
