% =============================================================================
% SSH.tex – Pi einrichten und passwortloses SSH konfigurieren
% Modulares Fragment (kein \documentclass, kein \begin{document})
% Stand: 2026-01-08 | Version: 2.5.2
% =============================================================================

\section{SSH-Setup}
\label{sec:ssh}

Bevor wir mit der Entwicklung beginnen können, richten wir den Raspberry Pi ein und konfigurieren einen passwortlosen SSH-Zugang. Das Ziel: Mit einem simplen \texttt{ssh rover} landen wir direkt auf dem Pi – ohne Passwortabfrage.

% -----------------------------------------------------------------------------
\subsection{Pi OS installieren}
\label{subsec:ssh-pi-os}

Wir nutzen den offiziellen Raspberry Pi Imager, um das Betriebssystem auf die SD-Karte zu schreiben.

\subsubsection{Raspberry Pi Imager}

\begin{enumerate}
  \item \textbf{Download:} \href{https://www.raspberrypi.com/software/}{raspberrypi.com/software}
  \item \textbf{OS auswählen:} Raspberry Pi OS Lite (64-bit)
  \item \textbf{Einstellungen} über das Zahnrad-Symbol konfigurieren (\cref{tab:imager-settings})
  \item \textbf{Schreiben} – SD-Karte einlegen – Netzteil einstecken
\end{enumerate}

\begin{table}[H]
  \centering
  \caption{Einstellungen im Raspberry Pi Imager}
  \label{tab:imager-settings}
  \begin{tabularx}{0.7\textwidth}{@{}l X@{}}
    \toprule
    \textbf{Einstellung} & \textbf{Wert} \\
    \midrule
    Hostname & \texttt{rover} \\
    SSH & ✓ aktivieren \\
    Benutzer & \texttt{pi} \\
    Passwort & (eigenes Passwort) \\
    WLAN & SSID + Passwort \\
    Zeitzone & \texttt{Europe/Berlin} \\
    \bottomrule
  \end{tabularx}
\end{table}

\subsubsection{Nach dem ersten Boot}

Sobald der Pi hochgefahren ist, verbinden wir uns erstmals per SSH und führen die Grundkonfiguration durch:

\begin{lstlisting}[style=shell]
ssh pi@rover.local
sudo apt update && sudo apt upgrade -y
sudo usermod -aG dialout pi
\end{lstlisting}

Der letzte Befehl fügt den Benutzer \texttt{pi} zur Gruppe \texttt{dialout} hinzu – das benötigen wir später für den Zugriff auf den ESP32 über USB.

% -----------------------------------------------------------------------------
\subsection{SSH-Key einrichten}
\label{subsec:ssh-key}

Passwörter sind umständlich und unsicher. Wir generieren stattdessen ein Schlüsselpaar und kopieren den öffentlichen Schlüssel auf den Pi.

\subsubsection{Mac / Linux}

\begin{lstlisting}[style=shell]
ssh-keygen -t ed25519
ssh-copy-id pi@rover.local
\end{lstlisting}

Der erste Befehl erzeugt ein Ed25519-Schlüsselpaar (aktueller Standard, kompakt und sicher). Der zweite kopiert den Public Key automatisch in die \filep{authorized\_keys} des Pi.

\subsubsection{Windows (PowerShell)}

Unter Windows müssen wir zunächst den OpenSSH-Client aktivieren:

\begin{lstlisting}[style=shell,language={}]
# OpenSSH aktivieren (als Admin ausfuehren)
Add-WindowsCapability -Online -Name OpenSSH.Client~~~~0.0.1.0

# Key erstellen
ssh-keygen -t ed25519

# Key manuell kopieren
type $env:USERPROFILE\.ssh\id_ed25519.pub |
ssh pi@rover.local "mkdir -p ~/.ssh && cat >> ~/.ssh/authorized_keys && chmod 600 ~/.ssh/authorized_keys"
\end{lstlisting}

\begin{infobox}[Warum Ed25519?]
  Ed25519 bietet bei nur \SI{256}{\bit} Schlüssellänge eine Sicherheit vergleichbar mit RSA-3072. Die Schlüssel sind kompakter und die Signaturoperationen schneller.
\end{infobox}

% -----------------------------------------------------------------------------
\subsection{SSH-Config anlegen}
\label{subsec:ssh-config}

Mit einer SSH-Konfigurationsdatei sparen wir uns künftig die Tipparbeit. Statt \texttt{ssh pi@rover.local} genügt dann \texttt{ssh rover}.

\subsubsection{Mac / Linux}

Wir erstellen oder erweitern die Datei \filep{\textasciitilde/.ssh/config}:

\begin{lstlisting}[style=shell,numbers=none]
Host rover
    HostName rover.local
    User pi
    IdentityFile ~/.ssh/id_ed25519
\end{lstlisting}

Die Berechtigungen müssen stimmen:

\begin{lstlisting}[style=shell]
chmod 600 ~/.ssh/config
\end{lstlisting}

\subsubsection{Windows}

Die Config-Datei liegt unter \filep{\%USERPROFILE\%\textbackslash.ssh\textbackslash config}:

\begin{lstlisting}[style=shell,numbers=none]
Host rover
    HostName rover.local
    User pi
    IdentityFile ~/.ssh/id_ed25519
\end{lstlisting}

\begin{tipbox}[Verbindung testen]
  Nach der Konfiguration testen wir mit \texttt{ssh rover}. Wenn alles funktioniert, landen wir ohne Passwortabfrage direkt auf dem Pi.
\end{tipbox}

% -----------------------------------------------------------------------------
\subsection{GitHub-Zugang einrichten}
\label{subsec:ssh-github}

Für den Zugriff auf GitHub-Repositories erstellen wir einen separaten Schlüssel:

\begin{lstlisting}[style=shell]
ssh-keygen -t ed25519 -f ~/.ssh/id_ed25519_github
\end{lstlisting}

Die SSH-Config erweitern wir um einen Eintrag für GitHub:

\begin{lstlisting}[style=shell,numbers=none]
Host github.com
    HostName github.com
    User git
    IdentityFile ~/.ssh/id_ed25519_github
    IdentitiesOnly yes
\end{lstlisting}

Den Public Key hinterlegen wir bei GitHub unter \textit{Settings → SSH Keys → New SSH Key}. Den Inhalt der Datei \filep{\textasciitilde/.ssh/id\_ed25519\_github.pub} kopieren wir in das Textfeld.

\begin{lstlisting}[style=shell]
# Verbindung testen
ssh -T git@github.com
# Erwartete Ausgabe: "Hi username! You've successfully authenticated..."
\end{lstlisting}

% -----------------------------------------------------------------------------
\subsection{Serial-Port-Berechtigungen}
\label{subsec:ssh-serial}

Damit wir den ESP32 über USB ansprechen können, benötigt der Benutzer \texttt{pi} Zugriff auf die Serial-Ports:

\begin{lstlisting}[style=shell]
# Auf dem Pi ausfuehren
sudo usermod -aG dialout pi
# Danach neu einloggen (oder reboot)

# Stabilen by-id Pfad pruefen
ls -la /dev/serial/by-id/usb-Espressif*
\end{lstlisting}

\begin{warnbox}[Neu einloggen erforderlich]
  Gruppenmitgliedschaften werden erst nach einem neuen Login aktiv. Ein einfaches \texttt{exit} und erneutes \texttt{ssh rover} genügt.
\end{warnbox}

% -----------------------------------------------------------------------------
\subsection{Troubleshooting}
\label{subsec:ssh-troubleshooting}

\Cref{tab:ssh-troubleshooting} listet die häufigsten Probleme und deren Lösungen.

\begin{table}[H]
  \centering
  \caption{SSH-Fehlerbehebung}
  \label{tab:ssh-troubleshooting}
  \begin{tabularx}{\textwidth}{@{}l X@{}}
    \toprule
    \textbf{Problem} & \textbf{Lösung} \\
    \midrule
    \texttt{Permission denied (publickey)} & \texttt{ssh-copy-id} erneut ausführen \\
    \texttt{Could not resolve hostname} & IP direkt nutzen: \texttt{ssh pi@192.168.x.x} \\
    \texttt{UNPROTECTED PRIVATE KEY FILE} & Berechtigungen setzen: \texttt{chmod 600 \textasciitilde/.ssh/id\_ed25519} \\
    Serial-Port nicht zugänglich & Gruppe hinzufügen: \texttt{sudo usermod -aG dialout pi} \\
    mDNS funktioniert nicht & Avahi prüfen: \texttt{sudo systemctl status avahi-daemon} \\
    \bottomrule
  \end{tabularx}
\end{table}

\begin{infobox}[IP-Adresse ermitteln]
  Falls \texttt{rover.local} nicht auflöst, finden wir die IP über den Router oder – falls wir noch einen Monitor am Pi haben – mit \texttt{hostname -I}.
\end{infobox}
