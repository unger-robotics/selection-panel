% =============================================================================
% USB-PORT-VERWALTUNG.tex – Serial-Port-Exklusivität auf dem Pi 5
% Modulares Fragment (kein \documentclass, kein \begin{document})
% Stand: 2026-01-08 | Version: 2.5.2
% =============================================================================

\section{USB-Port-Verwaltung}
\label{sec:usb-port-verwaltung}

Auf dem Raspberry Pi 5 teilen sich zwei Projekte denselben ESP32: das Selection Panel und die AMR Platform. Doch der Serial-Port verträgt nur einen Zugriff gleichzeitig – sonst gehen Daten verloren oder Reads brechen ab. Wie lösen wir dieses Problem?

\subsection{Das Exklusivitätsprinzip}
\label{subsec:usb-exklusivitaet}

Die Lösung liegt in einem gemeinsamen Lock via \texttt{flock} auf die Datei \filep{/var/lock/esp32-serial.lock}. Beide Projekte respektieren diesen Lock, verhalten sich aber unterschiedlich:

\begin{itemize}
  \item \textbf{Selection Panel (systemd):} Nutzt \texttt{flock -n} – startet nur, wenn der Lock frei ist, sonst Abbruch.
  \item \textbf{AMR micro-ROS Agent (Docker):} Nutzt \texttt{flock} ohne \texttt{-n} – wartet geduldig, bis der Lock frei wird.
\end{itemize}

\begin{infobox}[Stabiler Device-Pfad]
  Statt \texttt{/dev/ttyACM0} (kann sich ändern) empfehlen wir den by-id-Pfad:\\
  \texttt{/dev/serial/by-id/usb-Espressif\_USB\_JTAG\_serial\_debug\_unit\_98:3D:AE:EA:08:1C-if00}
\end{infobox}

\Cref{tab:usb-komponenten} zeigt die beiden Projekte und ihre Zugriffsmethoden im Überblick.

\begin{table}[H]
  \centering
  \caption{Serial-Port-Nutzung der beiden Projekte}
  \label{tab:usb-komponenten}
  \begin{tabularx}{\textwidth}{@{}l l X l@{}}
    \toprule
    \textbf{Projekt} & \textbf{Prozess} & \textbf{Startart} & \textbf{Serial-Port} \\
    \midrule
    Selection Panel & \texttt{server.py} & systemd (\texttt{selection-panel.service}) & by-id (stabil) \\
    AMR Platform & \texttt{micro\_ros\_agent} & Docker Compose (\texttt{microros\_agent}) & by-id (empfohlen) \\
    \bottomrule
  \end{tabularx}
\end{table}

% -----------------------------------------------------------------------------
\subsection{Nach dem Reboot: Standard-Ablauf}
\label{subsec:usb-nach-reboot}

Nach einem Neustart des Pi müssen wir entscheiden, welches Projekt den Port nutzen soll. Schauen wir uns beide Modi an.

\subsubsection{Selection Panel Modus (UI + Taster/LEDs)}

Wir starten den Service und prüfen, ob alles läuft:

\begin{lstlisting}[style=shell]
sudo systemctl start selection-panel.service
sudo systemctl status selection-panel.service --no-pager
\end{lstlisting}

Das Dashboard erreichen wir im Browser unter \texttt{http://rover.local:8080/}. Falls mDNS nicht funktioniert, ermitteln wir die IP manuell:

\begin{lstlisting}[style=shell]
hostname -I
# Ausgabe z.B.: 192.168.1.24 172.17.0.1
# Browser: http://192.168.1.24:8080/
\end{lstlisting}

\begin{tipbox}[IP-Adressen verstehen]
  Die erste Adresse (hier \texttt{192.168.1.24}) ist die LAN/WLAN-IP für den Browser. Die \texttt{172.17.0.1} gehört zur Docker-Bridge und ist für den externen Zugriff nicht relevant.
\end{tipbox}

Für die Live-Logs nutzen wir:

\begin{lstlisting}[style=shell]
sudo journalctl -u selection-panel.service -f
\end{lstlisting}

Wenn wir jetzt Taster 1–10 drücken, sehen wir im Log Meldungen wie \texttt{Button X gedrueckt}.

\subsubsection{AMR Modus (micro-ROS Agent)}

Bevor der Agent starten kann, müssen wir das Selection Panel sauber beenden – das gibt Lock und Port frei:

\begin{lstlisting}[style=shell]
sudo systemctl stop selection-panel.service

cd /home/pi/amr/docker
sudo docker compose -p docker up -d microros_agent
\end{lstlisting}

Die Agent-Logs verfolgen wir mit:

\begin{lstlisting}[style=shell]
sudo docker compose -p docker logs -f microros_agent
\end{lstlisting}

Falls das Selection Panel noch laufen sollte, wartet der Agent dank \texttt{flock} geduldig, statt den Port zu blockieren.

% -----------------------------------------------------------------------------
\subsection{Schneller Wechsel ohne Reboot}
\label{subsec:usb-schneller-wechsel}

Im Entwicklungsalltag wechseln wir häufig zwischen beiden Modi. Die folgenden Befehle ermöglichen einen sauberen Übergang.

\subsubsection{Wechsel zum Selection Panel}

\begin{lstlisting}[style=shell]
cd /home/pi/amr/docker
sudo docker compose -p docker stop microros_agent

sudo systemctl start selection-panel.service
sudo journalctl -u selection-panel.service -f
\end{lstlisting}

\subsubsection{Wechsel zur AMR Platform}

\begin{lstlisting}[style=shell]
sudo systemctl stop selection-panel.service

cd /home/pi/amr/docker
sudo docker compose -p docker up -d microros_agent
sudo docker compose -p docker logs -f microros_agent
\end{lstlisting}

% -----------------------------------------------------------------------------
\subsection{Autostart konfigurieren}
\label{subsec:usb-autostart}

Soll das Selection Panel beim Boot automatisch starten?

\begin{lstlisting}[style=shell]
# Autostart aktivieren
sudo systemctl enable selection-panel.service

# Autostart deaktivieren
sudo systemctl disable selection-panel.service
\end{lstlisting}

\begin{warnbox}[Empfehlung für AMR]
  Den AMR-Agent starten wir bewusst manuell mit \texttt{docker compose up -d microros\_agent}. So ist immer klar definiert, wer den Port belegt.
\end{warnbox}

% -----------------------------------------------------------------------------
\subsection{Sanity Checks: Port und Lock prüfen}
\label{subsec:usb-sanity-checks}

Wenn etwas nicht funktioniert, helfen diese Befehle bei der Diagnose:

\begin{lstlisting}[style=shell]
# Wer haelt den USB-Port?
sudo fuser -v /dev/ttyACM0 || true

# Wer haelt den Lock?
sudo lslocks | grep esp32-serial || true
ls -l /var/lock/esp32-serial.lock || true
\end{lstlisting}

% -----------------------------------------------------------------------------
\subsection{One-time Setup: Docker Compose mit Serial-Lock}
\label{subsec:usb-docker-setup}

Damit der AMR-Agent den Lock respektiert, passen wir die Docker-Compose-Konfiguration einmalig an. Die Datei liegt unter \filep{/home/pi/amr/docker/docker-compose.yml}.

\subsubsection{Backup erstellen}

\begin{lstlisting}[style=shell]
cd /home/pi/amr/docker
cp -a docker-compose.yml docker-compose.yml.bak.$(date +%F_%H%M%S)
\end{lstlisting}

\subsubsection{Service-Definition anpassen}

Wir modifizieren nur den Service \texttt{microros\_agent}:

\begin{lstlisting}[style=json,caption={Lock-Wrapper für den microros\_agent}]
services:
  microros_agent:
    image: microros/micro-ros-agent:humble
    container_name: amr_agent
    network_mode: host
    privileged: true
    restart: always

    volumes:
      - /dev:/dev
      - /var/lock:/var/lock

    entrypoint: ["/bin/sh", "-lc"]
    command: >
      DEV="/dev/serial/by-id/usb-Espressif_USB_JTAG_serial_debug_unit_
      98:3D:AE:EA:08:1C-if00";
      echo "[microros_agent] waiting for lock (DEV=$DEV)";
      exec flock /var/lock/esp32-serial.lock
      /bin/sh /micro-ros_entrypoint.sh serial --dev "$DEV" -b 921600
\end{lstlisting}

\subsubsection{Änderungen anwenden}

\begin{lstlisting}[style=shell]
cd /home/pi/amr/docker
sudo docker compose -p docker config >/dev/null
sudo docker compose -p docker up -d --force-recreate microros_agent
\end{lstlisting}

% -----------------------------------------------------------------------------
\subsection{Troubleshooting}
\label{subsec:usb-troubleshooting}

\subsubsection{Selection Panel startet nicht}

Die häufigste Ursache: Der Lock ist belegt, weil der AMR-Agent läuft. Das Selection Panel bricht dann absichtlich ab.

\begin{lstlisting}[style=shell]
sudo journalctl -u selection-panel.service -n 120 --no-pager
sudo lslocks | grep esp32-serial || true
sudo fuser -v /dev/ttyACM0 || true
\end{lstlisting}

\textbf{Fix:} Den jeweils anderen Prozess stoppen – für Selection Panel also \texttt{sudo docker compose -p docker stop microros\_agent}.

\subsubsection{Device-Pfad hat sich geändert}

Nach einem Firmware-Update oder bei einem neuen ESP32 kann sich der by-id-Pfad ändern:

\begin{lstlisting}[style=shell]
ls -l /dev/serial/by-id/
ls /dev/ttyACM*
\end{lstlisting}

\textbf{Fix:} Den neuen Pfad in \filep{server.py} (Selection Panel) und im \texttt{DEV="..."} der Compose-Datei eintragen.

\subsubsection{Docker-Status prüfen}

\begin{lstlisting}[style=shell]
cd /home/pi/amr/docker
sudo docker compose -p docker ps
sudo docker compose -p docker logs --tail=120 microros_agent
\end{lstlisting}
