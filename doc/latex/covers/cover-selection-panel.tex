% =============================================================================
% cover-selection-panel.tex – Standalone Cover für techdoc.cls
% =============================================================================
% Kompilieren: xelatex cover-selection-panel.tex
% Erzeugt: cover-selection-panel.pdf (für \coverimage in main.tex)
% =============================================================================
\documentclass[tikz,border=10pt]{standalone}
\usepackage[ngerman]{babel}

% -----------------------------------------------------------------------------
% SCHRIFTEN (identisch mit techdoc.cls)
% -----------------------------------------------------------------------------
\usepackage{fontspec}
\defaultfontfeatures{Ligatures=TeX}

% TeX Gyre Schriften (Standard in TeX Live)
\setmainfont{TeX Gyre Pagella}
\setsansfont{TeX Gyre Heros}
\setmonofont[Scale=0.85]{TeX Gyre Cursor}

% -----------------------------------------------------------------------------
% FARBEN (identisch mit techdoc.cls)
% -----------------------------------------------------------------------------
\usepackage{xcolor}
\definecolor{ArduinoTeal}{HTML}{00979D}
\definecolor{ArduinoDark}{HTML}{005C5F}
\definecolor{RaspberryRed}{HTML}{C51A4A}
\definecolor{SuccessGreen}{HTML}{75A928}
\definecolor{WarningOrange}{HTML}{D97706}
\definecolor{MutedText}{HTML}{4A5568}
\definecolor{CodeBg}{HTML}{F8FAFC}
\definecolor{CodeFrame}{HTML}{C8D2DC}

% Zusätzliche Farben für Signale (konsistent mit Schaltplan)
\definecolor{SignalData}{HTML}{51CF66}
\definecolor{SignalClock}{HTML}{FFD43B}
\definecolor{SignalLatch}{HTML}{4DABF7}
\definecolor{BackgroundLight}{HTML}{FAFAFA}

% ALIASE für Kompatibilität
\colorlet{espcolor}{ArduinoTeal}
\colorlet{picolor}{RaspberryRed}
\colorlet{webcolor}{SuccessGreen}
\colorlet{warncolor}{WarningOrange}
\colorlet{hardwarecolor}{MutedText}
\colorlet{signalcolor}{ArduinoDark}
\colorlet{bgcolor}{BackgroundLight}

% -----------------------------------------------------------------------------
% TIKZ-BIBLIOTHEKEN
% -----------------------------------------------------------------------------
\usepackage{tikz}
\usetikzlibrary{shapes.geometric, arrows.meta, positioning, fit, backgrounds, calc, shadows.blur}

\begin{document}
\begin{tikzpicture}[
    % Styles
    mainblock/.style={
        rectangle,
        rounded corners=8pt,
        minimum width=2.8cm,
        minimum height=1.8cm,
        text centered,
        font=\sffamily\bfseries\normalsize,
        draw=none,
        blur shadow={shadow blur steps=5, shadow xshift=0.5mm, shadow yshift=-0.5mm}
    },
    hwblock/.style={
        rectangle,
        rounded corners=4pt,
        minimum width=2cm,
        minimum height=0.9cm,
        text centered,
        font=\sffamily\footnotesize,
        draw=hardwarecolor,
        line width=0.5pt,
        fill=white
    },
    ioblock/.style={
        rectangle,
        rounded corners=3pt,
        minimum width=1.6cm,
        minimum height=0.6cm,
        text centered,
        font=\sffamily\footnotesize,
        draw=hardwarecolor!60,
        line width=0.4pt,
        fill=hardwarecolor!5
    },
    arrow/.style={
        ->,
        >=Stealth,
        line width=1pt,
        signalcolor
    },
    doublearrow/.style={
        <->,
        >=Stealth,
        line width=1pt,
        signalcolor
    },
    lbl/.style={
        font=\sffamily\tiny,
        signalcolor,
        fill=bgcolor,
        inner sep=2pt
    }
]

% Hintergrund (mit Platz für Fußzeile in Präsentation)
\fill[bgcolor] (-9.5,-5.8) rectangle (9.5,3.2);

% ============================================
% Linke Seite: Hardware (Taster + LEDs)
% ============================================

% Taster-Block (Prototyp: 10×)
\node[ioblock, minimum height=1.6cm, minimum width=1.6cm] (taster) at (-8,1.8) {};
\node[font=\sffamily\bfseries\scriptsize, above=0.05cm of taster.south] {10× Taster};
\foreach \y in {1.5,1.7,1.9,2.1} {
    \fill[hardwarecolor!40] (-8.3,\y) circle (0.08);
    \fill[hardwarecolor!40] (-8.1,\y) circle (0.08);
    \fill[hardwarecolor!40] (-7.9,\y) circle (0.08);
    \fill[hardwarecolor!40] (-7.7,\y) circle (0.08);
}
\foreach \y in {1.5,1.7} {
    \fill[hardwarecolor!40] (-8.3,\y) circle (0.08);
    \fill[hardwarecolor!40] (-7.7,\y) circle (0.08);
}

% LED-Block (Prototyp: 10×) - Farben passend zum Farbschema
\node[ioblock, minimum height=1.6cm, minimum width=1.6cm] (leds) at (-8,-1.2) {};
\node[font=\sffamily\bfseries\scriptsize, above=0.05cm of leds.south] {10× LEDs};
\foreach \y in {-1.5,-1.3,-1.1,-0.9} {
    \fill[RaspberryRed!70] (-8.3,\y) circle (0.08);
    \fill[SuccessGreen!80] (-8.1,\y) circle (0.08);
    \fill[ArduinoTeal!80] (-7.9,\y) circle (0.08);
    \fill[SignalClock!90] (-7.7,\y) circle (0.08);
}
\foreach \y in {-1.5,-1.3} {
    \fill[SignalLatch!80] (-8.3,\y) circle (0.08);
    \fill[white] (-7.7,\y) circle (0.08);
    \draw[hardwarecolor!60] (-7.7,\y) circle (0.08);
}

% One-hot Indikator
\node[font=\sffamily\tiny\itshape, hardwarecolor] at (-8,-2.2) {(one-hot: max. 1 aktiv)};

% ============================================
% Schieberegister (Prototyp: 2×)
% ============================================

\node[hwblock, fill=SuccessGreen!10, draw=SuccessGreen!60] (sr4021) at (-5,1.8) {2× CD4021BE\\{\tiny Input-Shift}};
\node[hwblock, fill=RaspberryRed!10, draw=RaspberryRed!60] (sr595) at (-5,-1.2) {2× 74HC595\\{\tiny Output-Shift}};

% Pfeile Taster → SR
\draw[arrow] (taster.east) -- (sr4021.west);
\node[lbl] at (-6.5,2.2) {parallel};

% Pfeile SR → LEDs
\draw[arrow] (sr595.west) -- (leds.east);
\node[lbl] at (-6.5,-0.8) {parallel};

% ============================================
% ESP32-S3 - Arduino Teal
% ============================================

\node[mainblock, fill=ArduinoTeal, text=white, minimum height=2.2cm] (esp) at (-1.2,0.3) {
    \begin{tabular}{c}
    Seeed XIAO\\[2pt]
    ESP32-S3\\[3pt]
    {\scriptsize\normalfont FreeRTOS}
    \end{tabular}
};

% Pfeile SR ↔ ESP32
\draw[arrow] (sr4021.east) -- (esp.west |- sr4021);
\node[lbl] at (-3.1,2.2) {SPI};

\draw[arrow] (esp.west |- sr595) -- (sr595.east);
\node[lbl] at (-3.1,-0.8) {SPI};

% ============================================
% Raspberry Pi 5 - Raspberry Red
% ============================================

\node[mainblock, fill=RaspberryRed, text=white, minimum height=2.2cm] (pi) at (3.5,0.3) {
    \begin{tabular}{c}
    Raspberry\\[2pt]
    Pi 5\\[3pt]
    {\scriptsize\normalfont bridge.py}
    \end{tabular}
};

% USB Serial zwischen ESP und Pi - Farbverlauf
\draw[doublearrow, ArduinoTeal!60!RaspberryRed, line width=1.2pt] (esp.east) -- (pi.west);
\node[lbl, font=\sffamily\scriptsize\bfseries] at (1.15,0.8) {USB-CDC};
\node[lbl] at (1.15,-0.2) {COBS + CRC16};

% ============================================
% Web-Browser - Success Green
% ============================================

\node[mainblock, fill=SuccessGreen, text=white, minimum width=2.6cm, minimum height=1.4cm] (browser) at (3.5,-3.5) {
    \begin{tabular}{c}
    Web-Browser\\[2pt]
    {\scriptsize\normalfont React Dashboard}
    \end{tabular}
};

% WebSocket + HTTP
\draw[doublearrow, RaspberryRed!60!SuccessGreen, line width=1.2pt] (pi.south) -- (browser.north);
\node[lbl, font=\sffamily\scriptsize\bfseries] at (5.0,-1.6) {WebSocket};
\node[lbl] at (2.0,-1.6) {JSON :8081};

% ============================================
% Datenfluss-Legende
% ============================================

\node[font=\sffamily\scriptsize, anchor=north west, text width=4.2cm] at (-9.2,-3) {
    \textbf{Ablauf (Button-Druck):}\\[4pt]
    1. Taste drücken\\[2pt]
    2. ESP32: Debounce (50\,ms)\\[2pt]
    3. ESP32 $\rightarrow$ Pi: SELECTION\_STATE\\[2pt]
    4. Pi broadcastet \texttt{selection}\\[2pt]
    5. Browser zeigt aktive LED
};

% Policy-Box - Arduino Teal
\node[draw=ArduinoTeal, rounded corners=3pt, fill=ArduinoTeal!8,
      font=\sffamily\scriptsize, text width=2.8cm, align=center,
      anchor=north] at (-1.2,-2.8) {
    \textbf{Policy}\\[4pt]
    LowestIdWins\\
    One-hot LED
};

% WebSocket-Kommando Box
\node[draw=SuccessGreen, rounded corners=3pt, fill=SuccessGreen!8,
      font=\sffamily\scriptsize, text width=2.6cm, align=center,
      anchor=north] at (7.5,-0.5) {
    \textbf{WebSocket-API}\\[4pt]
    \texttt{setSelection}\\
    \texttt{selection}\\
    \texttt{hello}
};

% Skalierbarkeits-Hinweis
\node[font=\sffamily\tiny\itshape, hardwarecolor, text width=3.5cm, align=center] at (7.5,-3.5) {
    Skalierung auf 100×:\\
    13× Schieberegister
};

\end{tikzpicture}
\end{document}
