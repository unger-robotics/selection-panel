% =============================================================================
% farbschema.tex – Farbdefinitionen mit visueller Vorschau
% =============================================================================
% Projekt: Interaktives Auswahlpanel
% =============================================================================
\documentclass{techdoc}

% =============================================================================
% METADATEN
% =============================================================================
\title{Farbschema}
\subtitle{Arduino Teal + Raspberry Pi Red + GitHub Dark}
\tagline{Konsistente Farbdefinitionen für alle Projektdateien}
\scope{LaTeX \textbullet\ CSS \textbullet\ Schaltpläne}
\stack{Arduino Teal \textbullet\ Raspberry Red \textbullet\ Success Green \textbullet\ Warning Orange}
\version{2.2.3}
\date{11.01.2026}
\author{Jan Unger}
\credits{Claude 4.5}
\coverimage{cover-farbschema.pdf}

% Bilder liegen in ./images/ und ./covers/
\graphicspath{{images/}{covers/}}

\setheadertitle{Farbschema}

\hypersetup{
  pdftitle={Farbschema – Auswahlpanel-Projekt},
  pdfauthor={Jan Unger}
}

% =============================================================================
% FARBDEFINITIONEN (alle Farben explizit definieren)
% =============================================================================

% Primärfarben
\definecolor{ArduinoTeal}{RGB}{0,151,157}
\definecolor{ArduinoDark}{RGB}{0,92,95}
\definecolor{ArduinoTealLight}{RGB}{98,174,178}
\definecolor{RaspberryRed}{RGB}{197,26,74}
\definecolor{RaspberryPurple}{RGB}{117,38,74}

% Statusfarben
\definecolor{SuccessGreen}{RGB}{117,169,40}
\definecolor{WarningOrange}{RGB}{217,119,6}

% Hintergründe
\definecolor{GitHubDark}{RGB}{13,17,23}
\definecolor{SlateDark}{RGB}{30,41,51}
\definecolor{SlateMedium}{RGB}{45,58,69}
\definecolor{BackgroundLight}{RGB}{250,250,250}

% Text
\definecolor{LightText}{RGB}{230,237,243}
\definecolor{DarkText}{RGB}{13,17,23}
\definecolor{MutedText}{RGB}{74,85,104}
\definecolor{SecondaryText}{RGB}{139,148,158}

% Signalfarben
\definecolor{SignalData}{RGB}{81,207,102}
\definecolor{SignalClock}{RGB}{255,212,59}
\definecolor{SignalLatch}{RGB}{77,171,247}
\definecolor{SignalVCC}{RGB}{255,107,107}
\definecolor{SignalGND}{RGB}{107,114,128}
\definecolor{SignalCapacitor}{RGB}{255,169,77}

% Code-Farben
\definecolor{CodeBackground}{RGB}{248,250,252}
\definecolor{CodeFrame}{RGB}{200,210,220}

% Info-Box-Farben
\definecolor{InfoBoxBg}{RGB}{230,245,250}
\definecolor{WarnBoxBg}{RGB}{255,247,237}
\definecolor{TipBoxBg}{RGB}{240,250,240}
\definecolor{ErrorBoxBg}{RGB}{255,240,245}

% Tabellen
\definecolor{TableHeader}{RGB}{230,245,250}
\definecolor{TableRowAlt}{RGB}{248,250,252}

% =============================================================================
% HILFSBEFEHLE
% =============================================================================

% Farbkreis mit Hex-Code
\newcommand{\farbkreis}[2]{%
  \tikz[baseline=-0.5ex]{%
    \fill[fill=#1] (0,0) circle (0.35em);%
  }%
  \hspace{0.3em}%
  \texttt{\small #2}%
}

% Farbbox für Code-Namen
\newcommand{\codename}[1]{%
  \tikz[baseline=-0.5ex]{%
    \node[fill=CodeBackground, draw=CodeFrame, rounded corners=2pt,
          inner sep=2pt, font=\ttfamily\small] {#1};%
  }%
}

% Große Farbvorschau (dunkler Text)
\newcommand{\farbvorschau}[2]{%
  \tikz{%
    \fill[fill=#1] (0,0) rectangle (2.5cm,0.8cm);%
    \node[anchor=west, font=\ttfamily\small, text=white] at (0.15,0.4) {#2};%
  }%
}

% Farbvorschau mit dunklem Text (für helle Farben)
\newcommand{\farbvorschauhell}[2]{%
  \tikz{%
    \fill[fill=#1] (0,0) rectangle (2.5cm,0.8cm);%
    \draw[CodeFrame] (0,0) rectangle (2.5cm,0.8cm);%
    \node[anchor=west, font=\ttfamily\small, text=DarkText] at (0.15,0.4) {#2};%
  }%
}

% Highlight-Befehle (providecommand falls nicht in techdoc.cls definiert)
\providecommand{\highlight}[1]{\textcolor{ArduinoTeal}{\textbf{#1}}}
\providecommand{\improvement}[1]{\textcolor{SuccessGreen}{\textbf{#1}}}
\providecommand{\warning}[1]{\textcolor{WarningOrange}{\textbf{#1}}}
\providecommand{\critical}[1]{\textcolor{RaspberryRed}{\textbf{#1}}}

% =============================================================================
% DOKUMENT
% =============================================================================
\begin{document}

\maketitlepage

\pagenumbering{roman}
\tableofcontents
\clearpage
\pagenumbering{arabic}

% =============================================================================
% INHALT
% =============================================================================

\section{Strukturierte Farbdefinitionen}

\begin{table}[H]
\centering
\caption{Primär- und Statusfarben}
\label{tab:farben-primaer}
\begin{tabular}{@{} p{2.8cm} p{3.2cm} p{3.5cm} p{4cm} @{}}
\toprule
\textbf{Kategorie} & \textbf{Farbe} & \textbf{Hex} & \textbf{LaTeX-Name} \\
\midrule
\rowcolor{TableRowAlt}
\textbf{Primär} & Arduino Teal & \farbkreis{ArduinoTeal}{\ \#00979D} & \codename{ArduinoTeal} \\
 & Arduino Dark & \farbkreis{ArduinoDark}{\ \#005C5F} & \codename{ArduinoDark} \\
\rowcolor{TableRowAlt}
 & Raspberry Red & \farbkreis{RaspberryRed}{\ \#C51A4A} & \codename{RaspberryRed} \\
 & Raspberry Purple & \farbkreis{RaspberryPurple}{\ \#75264A} & \codename{RaspberryPurple} \\
\midrule
\rowcolor{TableRowAlt}
\textbf{Status} & Success Green & \farbkreis{SuccessGreen}{\ \#75A928} & \codename{SuccessGreen} \\
 & Warning Orange & \farbkreis{WarningOrange}{\ \#D97706} & \codename{WarningOrange} \\
\midrule
\rowcolor{TableRowAlt}
\textbf{Hintergrund} & GitHub Dark & \farbkreis{GitHubDark}{\ \#0D1117} & \codename{GitHubDark} \\
 & Slate Dark & \farbkreis{SlateDark}{\ \#1E2933} & \codename{SlateDark} \\
\rowcolor{TableRowAlt}
 & Slate Medium & \farbkreis{SlateMedium}{\ \#2D3A45} & \codename{SlateMedium} \\
\midrule
\textbf{Text} & Muted Text & \farbkreis{MutedText}{\ \#4A5568} & \codename{MutedText} \\
\rowcolor{TableRowAlt}
 & Light Text & \farbkreis{LightText}{\ \#E6EDF3} & \codename{LightText} \\
\bottomrule
\end{tabular}
\end{table}

\section{Abgeleitete Farben für Komponenten}

\begin{table}[H]
\centering
\caption{Box- und Code-Farben}
\label{tab:farben-komponenten}
\begin{tabular}{@{} p{3.5cm} p{5cm} p{4.5cm} @{}}
\toprule
\textbf{Komponente} & \textbf{Hintergrund} & \textbf{Rahmen} \\
\midrule
\rowcolor{TableRowAlt}
\textbf{Info-Box} & Arduino Teal (hell) & Arduino Teal \\
\textbf{Warn-Box} & Warning Orange (hell) & Warning Orange \\
\rowcolor{TableRowAlt}
\textbf{Tip-Box} & Success Green (hell) & Success Green \\
\textbf{Code-Blöcke} & Sehr helles Grau & Grau \\
\bottomrule
\end{tabular}
\end{table}

\section{Farbvorschau}

\begin{figure}[H]
\centering
\begin{tabular}{@{} c c c c @{}}
\farbvorschau{ArduinoTeal}{\#00979D} &
\farbvorschau{ArduinoDark}{\#005C5F} &
\farbvorschau{RaspberryRed}{\#C51A4A} &
\farbvorschau{RaspberryPurple}{\#75264A} \\[0.3em]
{\small Arduino Teal} & {\small Arduino Dark} & {\small Raspberry Red} & {\small Raspberry Purple} \\[1em]

\farbvorschau{SuccessGreen}{\#75A928} &
\farbvorschau{WarningOrange}{\#D97706} &
\farbvorschau{GitHubDark}{\#0D1117} &
\farbvorschau{SlateMedium}{\#2D3A45} \\[0.3em]
{\small Success Green} & {\small Warning Orange} & {\small GitHub Dark} & {\small Slate Medium} \\
\end{tabular}
\caption{Visuelle Farbvorschau der Hauptfarben}
\label{fig:farbvorschau}
\end{figure}

\section{Signalfarben (Schaltplan)}

\begin{table}[H]
\centering
\caption{Signalfarben für Schaltpläne}
\label{tab:signalfarben}
\begin{tabular}{@{} p{2.8cm} p{3.2cm} p{3.5cm} p{5cm} @{}}
\toprule
\textbf{Signal} & \textbf{Farbe} & \textbf{Hex} & \textbf{Verwendung} \\
\midrule
\rowcolor{TableRowAlt}
Data & \farbkreis{SignalData}{\ \#51CF66} & \codename{SignalData} & SER, Q8 \\
Clock & \farbkreis{SignalClock}{\ \#FFD43B} & \codename{SignalClock} & SRCLK, CLK \\
\rowcolor{TableRowAlt}
Latch & \farbkreis{SignalLatch}{\ \#4DABF7} & \codename{SignalLatch} & RCLK, P/S \\
VCC & \farbkreis{SignalVCC}{\ \#FF6B6B} & \codename{SignalVCC} & Versorgung \\
\rowcolor{TableRowAlt}
GND & \farbkreis{SignalGND}{\ \#6B7280} & \codename{SignalGND} & Masse \\
\bottomrule
\end{tabular}
\end{table}

\section{Semantische Zuordnung}

\begin{table}[H]
\centering
\caption{Hardware-Komponenten und ihre Farben}
\label{tab:semantik}
\begin{tabular}{@{} p{3.5cm} p{4cm} p{3cm} @{}}
\toprule
\textbf{Komponente} & \textbf{Farbe} & \textbf{Hex} \\
\midrule
\rowcolor{TableRowAlt}
ESP32-S3 XIAO & \farbkreis{ArduinoTeal}{\ Arduino Teal} & \texttt{\#00979D} \\
74HC595 (Output) & \farbkreis{RaspberryRed}{\ Raspberry Red} & \texttt{\#C51A4A} \\
\rowcolor{TableRowAlt}
CD4021BE (Input) & \farbkreis{SuccessGreen}{\ Success Green} & \texttt{\#75A928} \\
Raspberry Pi 5 & \farbkreis{RaspberryRed}{\ Raspberry Red} & \texttt{\#C51A4A} \\
\rowcolor{TableRowAlt}
Web-Browser & \farbkreis{SuccessGreen}{\ Success Green} & \texttt{\#75A928} \\
Warnung & \farbkreis{WarningOrange}{\ Warning Orange} & \texttt{\#D97706} \\
\bottomrule
\end{tabular}
\end{table}

\section{Highlight-Befehle}

\begin{table}[H]
\centering
\caption{Verfügbare Highlight-Befehle}
\label{tab:highlights}
\begin{tabular}{@{} p{4.5cm} p{4cm} p{5cm} @{}}
\toprule
\textbf{Befehl} & \textbf{Ergebnis} & \textbf{Verwendung} \\
\midrule
\rowcolor{TableRowAlt}
\texttt{\textbackslash highlight\{...\}} & \highlight{Wichtiger Text} & Hauptakzent, Kernaussagen \\
\texttt{\textbackslash improvement\{...\}} & \improvement{Erledigt} & Erfolge, OK \\
\rowcolor{TableRowAlt}
\texttt{\textbackslash warning\{...\}} & \warning{Achtung} & Warnungen, Hinweise \\
\texttt{\textbackslash critical\{...\}} & \critical{Fehler} & Kritische Punkte \\
\bottomrule
\end{tabular}
\end{table}

\section{Info-Box-Vorschau}

\begin{figure}[H]
\centering
\begin{tabular}{@{} c c @{}}
% Info-Box
\tikz{
  \fill[fill=InfoBoxBg] (0,0) rectangle (5.5cm,1.2cm);
  \draw[ArduinoTeal, line width=1pt] (0,0) rectangle (5.5cm,1.2cm);
  \node[anchor=west, font=\sffamily\bfseries\small, text=ArduinoTeal] at (0.2,0.85) {Info-Box};
  \node[anchor=west, font=\small] at (0.2,0.35) {Hinweise, Erklärungen};
}
&
% Warn-Box
\tikz{
  \fill[fill=WarnBoxBg] (0,0) rectangle (5.5cm,1.2cm);
  \draw[WarningOrange, line width=1pt] (0,0) rectangle (5.5cm,1.2cm);
  \node[anchor=west, font=\sffamily\bfseries\small, text=WarningOrange] at (0.2,0.85) {Warn-Box};
  \node[anchor=west, font=\small] at (0.2,0.35) {Warnungen, Achtung};
}
\\[1em]
% Tip-Box
\tikz{
  \fill[fill=TipBoxBg] (0,0) rectangle (5.5cm,1.2cm);
  \draw[SuccessGreen, line width=1pt] (0,0) rectangle (5.5cm,1.2cm);
  \node[anchor=west, font=\sffamily\bfseries\small, text=SuccessGreen] at (0.2,0.85) {Tip-Box};
  \node[anchor=west, font=\small] at (0.2,0.35) {Tipps, Best Practices};
}
&
% Error-Box
\tikz{
  \fill[fill=ErrorBoxBg] (0,0) rectangle (5.5cm,1.2cm);
  \draw[RaspberryRed, line width=1pt] (0,0) rectangle (5.5cm,1.2cm);
  \node[anchor=west, font=\sffamily\bfseries\small, text=RaspberryRed] at (0.2,0.85) {Error-Box};
  \node[anchor=west, font=\small] at (0.2,0.35) {Fehler, Kritisch};
}
\\
\end{tabular}
\caption{Vorschau der Info-Box-Varianten}
\label{fig:infoboxen}
\end{figure}

\section{Verwendung}

\subsection{In LaTeX-Dokumenten}

\begin{lstlisting}[style=shell]
\usepackage[table]{xcolor}
\input{farbschema.tex}

% Dann verfuegbar:
\textcolor{ArduinoTeal}{Text}
\highlight{Wichtig}
\end{lstlisting}

\subsection{CSS-Variablen}

\begin{lstlisting}[style=shell]
:root {
    --arduino-teal: #00979D;
    --arduino-dark: #005C5F;
    --raspberry-red: #C51A4A;
    --success-green: #75A928;
    --warning-orange: #D97706;
    --bg-primary: #0D1117;
}
\end{lstlisting}

\begin{infobox}[Konsistenz-Hinweis]
Dieses Farbschema ist identisch in folgenden Dateien definiert:
\texttt{praesentation.tex}, \texttt{main.tex}, \texttt{cover-diagram.tex},
\texttt{schematic.css}, \texttt{create\_amr\_placeholders.sh}
\end{infobox}

\end{document}
