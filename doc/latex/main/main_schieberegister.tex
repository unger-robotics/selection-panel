% =============================================================================
% main.tex – Technische Dokumentation
% =============================================================================
% Optionen: print (Links ausblenden), draft (Overfull + TODOs)
% =============================================================================
\documentclass{techdoc}
% \documentclass[print]{techdoc} % Warum: PDF fuer Ausdruck ohne Link-Dekoration.
% \documentclass[draft]{techdoc} % Warum: Layout-Probleme/TODOs frueh sichtbar machen.

% =============================================================================
% METADATEN (SSOT: alles an einem Ort, damit nichts divergiert)
% =============================================================================
\title{Bit-Mapping bei Schieberegistern}
\subtitle{CD4021B (Input) \& 74HC595 (Output)}
\tagline{\glqq MSB vs. LSB -- Warum die Asymmetrie?\grqq}
\scope{ESP32-S3 \textbullet\ Selection Panel \textbullet\ SPI-Kommunikation}
\stack{CD4021B \textbullet\ 74HC595 \textbullet\ PISO/SIPO \textbullet\ MSB/LSB \textbullet\ Bit-Masken}
\version{1.0.0}
\date{20.01.2026}
\author{Jan Unger}
\credits{Claude Opus 4.5, ChatGPT 5.2}
\coverimage{cover-schieberegister.pdf}

% Bilder liegen in ./images/ und ./covers/
\graphicspath{{images/}{covers/}}

\setheadertitle{Schieberegister Bit-Mapping}

% PDF-Metadaten für Suchmaschinen und Dokumentenverwaltung.
\hypersetup{
  pdftitle={Bit-Mapping bei Schieberegistern: MSB vs. LSB},
  pdfauthor={Jan Unger},
  pdfsubject={CD4021B und 74HC595 Schieberegister},
  pdfkeywords={Schieberegister, PISO, SIPO, MSB, LSB, Bit-Masken, ESP32-S3}
}

% =============================================================================
% DOKUMENT
% =============================================================================
\begin{document}

\maketitlepage

\pagenumbering{roman}
\tableofcontents
\clearpage
\pagenumbering{arabic}

% =============================================================================
% INHALT (main bleibt schlank, Inhalt liegt in content/)
% =============================================================================
% =============================================================================
% grundlagen-bit-operationen.tex – Zahlensysteme und Bit-Operationen
% =============================================================================

\section{Zahlensysteme und Bit-Operationen – Grundlagen}

\subsection{Begriffsrahmen und Zielsetzung}

Bei der Arbeit mit Schieberegistern (CD4021, 74HC595) manipulierst du einzelne Bits in Bytes. Dafür brauchst du drei Fähigkeiten: Zahlensysteme verstehen, Datentypen korrekt wählen, Bit-Operationen sicher anwenden.

\begin{table}[htbp]
\centering
\begin{tabular}{@{}ll@{}}
\toprule
\textbf{Begriff} & \textbf{Definition} \\
\midrule
Bit    & Kleinste Informationseinheit (0 oder 1) \\
Byte   & 8 Bits, Wertebereich 0–255 \\
Nibble & 4 Bits, entspricht einer Hex-Ziffer \\
MSB    & Most Significant Bit – Bit 7, höchstwertiges Bit \\
LSB    & Least Significant Bit – Bit 0, niedrigstwertiges Bit \\
Maske  & Bit-Muster zur gezielten Manipulation bestimmter Bits \\
\bottomrule
\end{tabular}
\caption{Begriffsrahmen Bit-Operationen}
\label{tab:bit-begriffe}
\end{table}

\textbf{Ziel:} Du kannst eine Button-ID (\zB 7) in eine Bit-Position umrechnen und das entsprechende Bit lesen oder setzen.

% -----------------------------------------------------------------------------
\subsection{Zahlensysteme – Drei Darstellungen, ein Wert}

\textbf{Problem:} Der Wert \glqq zwölf\grqq{} lässt sich auf verschiedene Arten schreiben. In der Embedded-Entwicklung begegnen dir alle drei Darstellungen – oft im selben Code.

\begin{table}[htbp]
\centering
\begin{tabular}{@{}lclll@{}}
\toprule
\textbf{System} & \textbf{Basis} & \textbf{Ziffern} & \textbf{C/C++ Präfix} & \textbf{Beispiel (12)} \\
\midrule
Dezimal     & 10 & 0–9      & (keiner) & \code{12} \\
Binär       &  2 & 0, 1     & \code{0b} & \code{0b00001100} \\
Hexadezimal & 16 & 0–9, A–F & \code{0x} & \code{0x0C} \\
\bottomrule
\end{tabular}
\caption{Zahlensysteme im Vergleich}
\label{tab:zahlensysteme}
\end{table}

\textbf{Wann verwendest du welches System?}

\begin{itemize}
\item \textbf{Dezimal} für Zählwerte, Indizes, Zeitangaben: \code{count = 10}
\item \textbf{Binär} wenn du einzelne Bits siehst: \code{0b10000000} (Bit 7 gesetzt)
\item \textbf{Hex} für Byte-Werte und Adressen: \code{0x80} (kompakter als binär)
\end{itemize}

\paragraph{Nibble-Prinzip – Hex lesen lernen}

Ein Byte besteht aus zwei Nibbles (je 4 Bits). Jedes Nibble entspricht einer Hex-Ziffer:

\begin{lstlisting}[style=wrap,numbers=none]
Byte:     0b1100.0101
Nibbles:    C     5
Hex:      0xC5
\end{lstlisting}

\textbf{Umrechnungstabelle (auswendig lernen lohnt sich):}

\begin{table}[htbp]
\centering
\begin{tabular}{@{}clr|clr@{}}
\toprule
\textbf{Hex} & \textbf{Binär} & \textbf{Dez} & \textbf{Hex} & \textbf{Binär} & \textbf{Dez} \\
\midrule
0 & \code{0000} &  0 & 8 & \code{1000} &  8 \\
1 & \code{0001} &  1 & 9 & \code{1001} &  9 \\
2 & \code{0010} &  2 & A & \code{1010} & 10 \\
3 & \code{0011} &  3 & B & \code{1011} & 11 \\
4 & \code{0100} &  4 & C & \code{1100} & 12 \\
5 & \code{0101} &  5 & D & \code{1101} & 13 \\
6 & \code{0110} &  6 & E & \code{1110} & 14 \\
7 & \code{0111} &  7 & F & \code{1111} & 15 \\
\bottomrule
\end{tabular}
\caption{Hex-Binär-Dezimal Umrechnungstabelle}
\label{tab:hex-binaer}
\end{table}

\textbf{Beispiel – 0x80 in Binär umwandeln:}

\begin{lstlisting}[style=wrap,numbers=none]
0x80 = 0x8 + 0x0
     = 0b1000 + 0b0000
     = 0b10000000
\end{lstlisting}

Das ist Bit 7 gesetzt – genau die Maske für ID 1 im MSB-first-Mapping.

% -----------------------------------------------------------------------------
\subsection{Bit-Positionen – Wo ist was im Byte?}

\textbf{Aufbau eines Bytes:}

\begin{figure}[htbp]
\centering
\begin{tikzpicture}[
    bit/.style={draw, minimum width=1.2cm, minimum height=0.8cm, font=\ttfamily\small},
    label/.style={font=\footnotesize}
]
% Bit cells
\foreach \i in {0,...,7} {
    \node[bit] (b\i) at ({(7-\i)*1.2},0) {\i};
}
% Position labels
\node[label, above=0.1cm of b7] {7};
\node[label, above=0.1cm of b6] {6};
\node[label, above=0.1cm of b5] {5};
\node[label, above=0.1cm of b4] {4};
\node[label, above=0.1cm of b3] {3};
\node[label, above=0.1cm of b2] {2};
\node[label, above=0.1cm of b1] {1};
\node[label, above=0.1cm of b0] {0};
% Value labels
\node[label, below=0.1cm of b7] {128};
\node[label, below=0.1cm of b6] {64};
\node[label, below=0.1cm of b5] {32};
\node[label, below=0.1cm of b4] {16};
\node[label, below=0.1cm of b3] {8};
\node[label, below=0.1cm of b2] {4};
\node[label, below=0.1cm of b1] {2};
\node[label, below=0.1cm of b0] {1};
% MSB/LSB labels
\node[font=\footnotesize\sffamily, color=ArduinoDark] at (-0.8,0) {MSB};
\node[font=\footnotesize\sffamily, color=ArduinoDark] at (9.2,0) {LSB};
\end{tikzpicture}
\caption{Aufbau eines Bytes mit Bit-Positionen und Stellenwerten}
\label{fig:byte-aufbau}
\end{figure}

\textbf{Stellenwert-Formel:} Bit $n$ hat den Wert $2^n$.

\begin{table}[htbp]
\centering
\begin{tabular}{@{}crr@{}}
\toprule
\textbf{Bit} & \textbf{Stellenwert} & \textbf{Hex-Maske} \\
\midrule
7 & 128 & \code{0x80} \\
6 &  64 & \code{0x40} \\
5 &  32 & \code{0x20} \\
4 &  16 & \code{0x10} \\
3 &   8 & \code{0x08} \\
2 &   4 & \code{0x04} \\
1 &   2 & \code{0x02} \\
0 &   1 & \code{0x01} \\
\bottomrule
\end{tabular}
\caption{Stellenwerte und Hex-Masken}
\label{tab:stellenwerte}
\end{table}

\textbf{Beispiel – Welchen Wert hat} \code{0b01001010}?

\begin{lstlisting}[style=wrap,numbers=none]
Gesetzte Bits: 6, 3, 1
Wert = 64 + 8 + 2 = 74 = 0x4A
\end{lstlisting}

% -----------------------------------------------------------------------------
\subsection{Datentypen – Die richtige Größe wählen}

\textbf{Problem:} C/C++ bietet viele Integer-Typen. Bei Embedded-Systemen ist die exakte Bitbreite entscheidend – ein Schieberegister liefert genau 8 Bits, nicht mehr, nicht weniger.

\textbf{Typen mit fester Breite (aus} \code{<stdint.h>}):

\begin{table}[htbp]
\centering
\begin{tabular}{@{}lcll@{}}
\toprule
\textbf{Typ} & \textbf{Bits} & \textbf{Wertebereich} & \textbf{Typischer Einsatz} \\
\midrule
\code{uint8\_t}  &  8 & 0 – 255           & Schieberegister-Bytes \\
\code{int8\_t}   &  8 & $-128$ – $+127$   & Vorzeichen-behaftete kleine Werte \\
\code{uint16\_t} & 16 & 0 – 65.535        & Sequenznummern, Timer \\
\code{uint32\_t} & 32 & 0 – 4.294.967.295 & Timestamps, große Zähler \\
\bottomrule
\end{tabular}
\caption{Integer-Typen mit fester Breite}
\label{tab:datentypen}
\end{table}

\textbf{Namenskonvention:} \code{u} = unsigned, Zahl = Bits, \code{\_t} = type

\textbf{Warum} \code{uint8\_t} \textbf{für Schieberegister?}

\begin{lstlisting}[style=arduino]
// Korrekt: exakt 8 Bit
uint8_t buttonState = 0;
uint8_t ledState[2] = {0, 0};

// Problematisch: Groesse plattformabhaengig
int buttonState = 0;  // 16 oder 32 Bit - verschwendet RAM
\end{lstlisting}

\begin{table}[htbp]
\centering
\begin{tabular}{@{}ll@{}}
\toprule
\textbf{Typ} & \textbf{Empfehlung} \\
\midrule
\code{uint8\_t}       & Empfohlen – exakt 8 Bit, portabel \\
\code{byte}           & Nur Arduino – nicht portabel \\
\code{unsigned char}  & Funktioniert, aber weniger explizit \\
\code{int}            & Zu groß, verschwendet RAM \\
\bottomrule
\end{tabular}
\caption{Typ-Empfehlungen für Byte-Daten}
\label{tab:typ-empfehlungen}
\end{table}

\paragraph{Überlauf beachten}

\begin{lstlisting}[style=arduino]
uint8_t x = 255;
x = x + 1;  // x = 0, nicht 256!

uint8_t y = 0;
y = y - 1;  // y = 255, nicht -1!
\end{lstlisting}

\begin{infobox}[Regel]
Bei \code{uint8\_t} \glqq wickelt\grqq{} der Wert um – von 255 auf 0 bzw. von 0 auf 255.
\end{infobox}

% -----------------------------------------------------------------------------
\subsection{Bit-Operationen – Die vier Grundoperationen}

\textbf{Übersicht:}

\begin{table}[htbp]
\centering
\begin{tabular}{@{}lclll@{}}
\toprule
\textbf{Operation} & \textbf{Operator} & \textbf{Code} & \textbf{Wirkung} \\
\midrule
Bit setzen  & OR      & \code{x |= mask}  & 0→1, 1→1 \\
Bit löschen & AND NOT & \code{x \&= \textasciitilde mask} & 1→0, 0→0 \\
Bit toggeln & XOR     & \code{x \^{}= mask} & 0→1, 1→0 \\
Bit prüfen  & AND     & \code{x \& mask}  & Ergebnis $\neq$ 0 wenn gesetzt \\
\bottomrule
\end{tabular}
\caption{Die vier Bit-Grundoperationen}
\label{tab:bit-operationen}
\end{table}

\paragraph{Bit setzen (OR)}

\textbf{Prinzip:} \code{x | 0 = x} (bleibt), \code{x | 1 = 1} (wird gesetzt)

\begin{lstlisting}[style=arduino]
uint8_t ledState = 0b00000000;
ledState |= 0b00010000;  // Bit 4 setzen

// Visualisierung:
//   0b00000000
// | 0b00010000
// -----------
//   0b00010000
\end{lstlisting}

\textbf{Anwendung:} LED einschalten

\begin{lstlisting}[style=arduino]
ledState |= (1u << 4);  // Bit 4 setzen -> LED 5 an (MSB-first)
\end{lstlisting}

\paragraph{Bit löschen (AND NOT)}

\textbf{Prinzip:} \code{x \& 1 = x} (bleibt), \code{x \& 0 = 0} (wird gelöscht)

\begin{lstlisting}[style=arduino]
uint8_t ledState = 0b00010000;
ledState &= ~0b00010000;  // Bit 4 loeschen

// Visualisierung:
// ~0b00010000 = 0b11101111
//
//   0b00010000
// & 0b11101111
// -----------
//   0b00000000
\end{lstlisting}

\textbf{Anwendung:} LED ausschalten

\begin{lstlisting}[style=arduino]
ledState &= ~(1u << 4);  // Bit 4 loeschen -> LED 5 aus
\end{lstlisting}

\paragraph{Bit toggeln (XOR)}

\textbf{Prinzip:} \code{x \^{} 0 = x} (bleibt), \code{x \^{} 1 = \textasciitilde x} (wechselt)

\begin{lstlisting}[style=arduino]
uint8_t ledState = 0b00010000;
ledState ^= 0b00010000;  // Bit 4 toggeln -> 0

ledState ^= 0b00010000;  // Bit 4 toggeln -> 1
\end{lstlisting}

\textbf{Anwendung:} Blinken ohne Zustandsvariable

\begin{lstlisting}[style=arduino]
ledState ^= (1u << 4);  // Jeder Aufruf wechselt den Zustand
\end{lstlisting}

\paragraph{Bit prüfen (AND)}

\textbf{Prinzip:} Ergebnis ist 0 wenn Bit nicht gesetzt, sonst $\neq$ 0

\begin{lstlisting}[style=arduino]
uint8_t buttonState = 0b11011111;  // Bit 5 = 0 (Button gedrueckt)

if (buttonState & 0b00100000) {
    // Bit 5 ist gesetzt -> Button NICHT gedrueckt
} else {
    // Bit 5 ist 0 -> Button gedrueckt (active-low)
}
\end{lstlisting}

\textbf{Anwendung:} Button-Zustand abfragen

\begin{lstlisting}[style=arduino]
bool isPressed = (buttonState & (1u << 5)) == 0;  // Active-low
\end{lstlisting}

% -----------------------------------------------------------------------------
\subsection{Warum \code{1u} statt \code{1}?}

\textbf{Problem:} Der Literal \code{1} ist vom Typ \code{int} (signed). Bei Bit-Shifts kann das zu undefiniertem Verhalten führen.

\begin{lstlisting}[style=arduino]
// Problematisch:
1 << 31  // Undefiniert! Signed overflow

// Sicher:
1u << 31  // OK, unsigned
\end{lstlisting}

\begin{warnbox}[Regel]
Verwende immer \code{1u} bei Bit-Operationen:
\begin{lstlisting}[style=arduino,numbers=none]
ledState |= (1u << bitPos);     // Setzen
ledState &= ~(1u << bitPos);    // Loeschen
if (state & (1u << bitPos))     // Pruefen
\end{lstlisting}
\end{warnbox}

% -----------------------------------------------------------------------------
\subsection{Maske erstellen – Vom Bit zur Maske}

\textbf{Formel:} \code{mask = 1u << bitPosition}

\begin{table}[htbp]
\centering
\begin{tabular}{@{}clll@{}}
\toprule
\textbf{Bit-Position} & \textbf{Shift} & \textbf{Maske (binär)} & \textbf{Maske (hex)} \\
\midrule
0 & \code{1u << 0} & \code{0b00000001} & \code{0x01} \\
1 & \code{1u << 1} & \code{0b00000010} & \code{0x02} \\
2 & \code{1u << 2} & \code{0b00000100} & \code{0x04} \\
3 & \code{1u << 3} & \code{0b00001000} & \code{0x08} \\
4 & \code{1u << 4} & \code{0b00010000} & \code{0x10} \\
5 & \code{1u << 5} & \code{0b00100000} & \code{0x20} \\
6 & \code{1u << 6} & \code{0b01000000} & \code{0x40} \\
7 & \code{1u << 7} & \code{0b10000000} & \code{0x80} \\
\bottomrule
\end{tabular}
\caption{Masken für Bit-Positionen 0–7}
\label{tab:masken}
\end{table}

\begin{tipbox}[Merkregel]
Die Hex-Werte folgen dem Muster 1, 2, 4, 8 (pro Nibble).
\end{tipbox}

% -----------------------------------------------------------------------------
\subsection{MSB-first Mapping – Von der ID zur Bit-Position}

\textbf{Projektstandard:} Das Selection Panel verwendet MSB-first. ID 1 liegt in Bit 7, ID 8 in Bit 0.

\textbf{Formeln:}

\begin{lstlisting}[style=arduino]
uint8_t byteIndex   = (id - 1) / 8;
uint8_t bitPosition = 7 - ((id - 1) % 8);  // MSB-first!
uint8_t mask        = 1u << bitPosition;
\end{lstlisting}

\textbf{Herleitung am Beispiel – ID 5:}

\begin{lstlisting}[style=wrap,numbers=none]
id = 5

byteIndex   = (5 - 1) / 8       = 4 / 8 = 0
bitPosition = 7 - ((5 - 1) % 8) = 7 - 4  = 3
mask        = 1u << 3           = 0x08

// Visualisierung:
// Byte 0:  ID  1   2   3   4   5   6   7   8
//          Bit 7   6   5   4   3   2   1   0
//                              ^
//                            ID 5 -> Bit 3 -> Maske 0x08
\end{lstlisting}

\textbf{Mapping-Tabelle (10 IDs):}

\begin{table}[htbp]
\centering
\begin{tabular}{@{}rrrr@{}}
\toprule
\textbf{ID} & \textbf{Byte} & \textbf{Bit} & \textbf{Maske} \\
\midrule
1  & 0 & 7 & \code{0x80} \\
2  & 0 & 6 & \code{0x40} \\
3  & 0 & 5 & \code{0x20} \\
4  & 0 & 4 & \code{0x10} \\
5  & 0 & 3 & \code{0x08} \\
6  & 0 & 2 & \code{0x04} \\
7  & 0 & 1 & \code{0x02} \\
8  & 0 & 0 & \code{0x01} \\
9  & 1 & 7 & \code{0x80} \\
10 & 1 & 6 & \code{0x40} \\
\bottomrule
\end{tabular}
\caption{MSB-first Mapping für 10 IDs}
\label{tab:msb-mapping}
\end{table}

% -----------------------------------------------------------------------------
\subsection{Praktische Funktionen}

\paragraph{LED setzen/löschen/abfragen}

\begin{lstlisting}[style=arduino]
#define LED_COUNT 10
#define LED_BYTES ((LED_COUNT + 7) / 8)

uint8_t ledState[LED_BYTES] = {0};

void setLed(uint8_t id) {
    if (id == 0 || id > LED_COUNT) return;
    uint8_t byteIdx = (id - 1) / 8;
    uint8_t bitPos  = 7 - ((id - 1) % 8);
    ledState[byteIdx] |= (1u << bitPos);
}

void clearLed(uint8_t id) {
    if (id == 0 || id > LED_COUNT) return;
    uint8_t byteIdx = (id - 1) / 8;
    uint8_t bitPos  = 7 - ((id - 1) % 8);
    ledState[byteIdx] &= ~(1u << bitPos);
}

bool getLed(uint8_t id) {
    if (id == 0 || id > LED_COUNT) return false;
    uint8_t byteIdx = (id - 1) / 8;
    uint8_t bitPos  = 7 - ((id - 1) % 8);
    return (ledState[byteIdx] & (1u << bitPos)) != 0;
}
\end{lstlisting}

\paragraph{Button prüfen (Active-Low)}

\begin{lstlisting}[style=arduino]
bool isButtonPressed(uint8_t id, const uint8_t* raw) {
    if (id == 0 || id > BTN_COUNT) return false;
    uint8_t byteIdx = (id - 1) / 8;
    uint8_t bitPos  = 7 - ((id - 1) % 8);

    // Active-low: gedrueckt wenn Bit = 0
    return (raw[byteIdx] & (1u << bitPos)) == 0;
}
\end{lstlisting}

\paragraph{One-Hot setzen}

\begin{lstlisting}[style=arduino]
void setOneHot(uint8_t selectedId, uint8_t* out, size_t bytes) {
    memset(out, 0, bytes);  // Alle Bits loeschen

    if (selectedId >= 1 && selectedId <= LED_COUNT) {
        uint8_t byteIdx = (selectedId - 1) / 8;
        uint8_t bitPos  = 7 - ((selectedId - 1) % 8);
        out[byteIdx] = 1u << bitPos;
    }
}
\end{lstlisting}

% -----------------------------------------------------------------------------
\subsection{Last-Byte-Maskierung}

\textbf{Problem:} Bei 10 LEDs brauchst du 2 Bytes (16 Bits), aber nur 10 sind gültig. Die Bits 5–0 im zweiten Byte sind \glqq überzählig\grqq.

\textbf{Berechnung:}

\begin{lstlisting}[style=arduino]
uint8_t validBits = LED_COUNT % 8;
if (validBits == 0) validBits = 8;

// MSB-first: die obersten validBits sind gueltig
uint8_t LAST_MASK = 0xFF << (8 - validBits);
\end{lstlisting}

\begin{table}[htbp]
\centering
\begin{tabular}{@{}rrrl@{}}
\toprule
\textbf{Count} & \textbf{validBits} & \textbf{LAST\_MASK} & \textbf{Gültige Bits} \\
\midrule
8   & 8 & \code{0xFF} & alle \\
10  & 2 & \code{0xC0} & Bit 7, 6 \\
12  & 4 & \code{0xF0} & Bit 7, 6, 5, 4 \\
100 & 4 & \code{0xF0} & Bit 7, 6, 5, 4 \\
\bottomrule
\end{tabular}
\caption{Last-Byte-Masken für verschiedene Kanalzahlen}
\label{tab:last-mask-bit}
\end{table}

\textbf{Anwendung:}

\begin{lstlisting}[style=arduino]
// LEDs: ungueltige Bits auf 0 setzen (bevor an HC595 gesendet)
ledState[LED_BYTES - 1] &= LED_LAST_MASK;

// Buttons: ungueltige Bits auf 1 setzen (active-low idle)
raw[BTN_BYTES - 1] |= ~BTN_LAST_MASK;
\end{lstlisting}

% -----------------------------------------------------------------------------
\subsection{Zusammenfassung – Merkregeln}

\paragraph{Zahlensysteme}

\begin{table}[htbp]
\centering
\begin{tabular}{@{}ll@{}}
\toprule
\textbf{Situation} & \textbf{Verwende} \\
\midrule
Zählwerte, Indizes     & Dezimal: \code{10} \\
Einzelne Bits sichtbar & Binär: \code{0b10000000} \\
Byte-Werte             & Hex: \code{0x80} \\
\bottomrule
\end{tabular}
\caption{Zahlensystem-Wahl}
\label{tab:zahlensystem-wahl}
\end{table}

\paragraph{Bit-Operationen}

\begin{table}[htbp]
\centering
\begin{tabular}{@{}ll@{}}
\toprule
\textbf{Aufgabe} & \textbf{Code} \\
\midrule
Bit setzen  & \code{x |= (1u << n)} \\
Bit löschen & \code{x \&= \textasciitilde(1u << n)} \\
Bit toggeln & \code{x \^{}= (1u << n)} \\
Bit prüfen  & \code{if (x \& (1u << n))} \\
\bottomrule
\end{tabular}
\caption{Bit-Operationen Kurzreferenz}
\label{tab:bit-kurzreferenz}
\end{table}

\paragraph{MSB-first Mapping}

\begin{table}[htbp]
\centering
\begin{tabular}{@{}ll@{}}
\toprule
\textbf{Formel} & \textbf{Berechnung} \\
\midrule
Byte-Anzahl  & \code{(count + 7) / 8} \\
Byte-Index   & \code{(id - 1) / 8} \\
Bit-Position & \code{7 - ((id - 1) \% 8)} \\
Maske        & \code{1u << bitPosition} \\
\bottomrule
\end{tabular}
\caption{MSB-first Mapping-Formeln}
\label{tab:mapping-formeln}
\end{table}

\paragraph{Hex-Masken (auswendig)}

\begin{table}[htbp]
\centering
\begin{tabular}{@{}lcccccccc@{}}
\toprule
\textbf{Bit} & 7 & 6 & 5 & 4 & 3 & 2 & 1 & 0 \\
\midrule
\textbf{Hex} & \code{0x80} & \code{0x40} & \code{0x20} & \code{0x10} & \code{0x08} & \code{0x04} & \code{0x02} & \code{0x01} \\
\textbf{ID (MSB-first)} & 1 & 2 & 3 & 4 & 5 & 6 & 7 & 8 \\
\bottomrule
\end{tabular}
\caption{Hex-Masken und IDs Schnellreferenz}
\label{tab:hex-masken-referenz}
\end{table}

% -----------------------------------------------------------------------------
\subsection{Weiterführende Dokumentation}

\begin{description}
\item[\code{docs/bit-mapping-schieberegister.md}] Detailliertes Mapping
\item[\code{firmware/docs/overview.md}] Projektübersicht
\item[\code{firmware/include/hw/io\_counts.h}] SSOT für Counts/Masks
\end{description}

% =============================================================================
% bit-mapping-schieberegister.tex – Schieberegister verstehen
% =============================================================================

\section{Bit-Mapping – Schieberegister verstehen}

\subsection{Begriffsrahmen und Zielsetzung}

Das Selection Panel verwendet Schieberegister, um 10–100 Buttons und LEDs mit wenigen GPIO-Pins anzusteuern. Die zentrale Herausforderung: Wie ordnen wir eine Button-ID (\zB \glqq Button 7\grqq) einem konkreten Bit in einem Byte-Array zu?

\begin{table}[htbp]
\centering
\begin{tabular}{@{}ll@{}}
\toprule
\textbf{Begriff} & \textbf{Definition} \\
\midrule
Bit-Mapping  & Zuordnungsregel zwischen logischer ID und physischer Bit-Position \\
MSB-first    & Most Significant Bit first – Bit 7 wird zuerst übertragen/gelesen \\
Active-Low   & Logisch aktiv bei Pegel 0 (GND) \\
Active-High  & Logisch aktiv bei Pegel 1 (VCC) \\
One-Hot      & Genau ein Bit gesetzt, alle anderen 0 \\
Kaskadierung & Mehrere ICs in Reihe geschaltet (Daisy-Chain) \\
\bottomrule
\end{tabular}
\caption{Begriffsrahmen Bit-Mapping}
\label{tab:bitmapping-begriffe}
\end{table}

\textbf{Ziel:} Eindeutige, skalierbare Zuordnung von IDs zu Bits – funktioniert identisch für 10 wie für 100 Kanäle.

% -----------------------------------------------------------------------------
\subsection{Das Grundproblem – IDs vs. Bytes}

\textbf{Problem:} Die Firmware-Logik arbeitet mit Button-IDs (1, 2, 3, \ldots, 100). Die Hardware arbeitet mit Byte-Arrays. Zwischen beiden muss eine eindeutige Übersetzung existieren.

\begin{figure}[htbp]
\centering
\begin{tikzpicture}[
    box/.style={draw, fill=CodeBg, minimum width=3.5cm, minimum height=0.8cm,
                font=\ttfamily\small, rounded corners=2pt},
    arr/.style={-{Stealth[length=2mm]}, thick, color=ArduinoDark}
]
\node[box] (logic) at (0,2) {selectedId = 7};
\node[font=\small\sffamily, color=MutedText] (map) at (0,1) {[Bit-Mapping]};
\node[box] (hw) at (0,0) {ledState[0]=0x02, [1]=0x00};
\draw[arr] (logic.south) -- (map.north);
\draw[arr] (map.south) -- (hw.north);
\end{tikzpicture}
\caption{Übersetzung zwischen Logik-IDs und Hardware-Bytes}
\label{fig:id-byte-mapping}
\end{figure}

\textbf{Regel:} IDs sind 1-basiert (1..N), Bytes sind 0-basiert (0..bytes-1), Bits sind 0–7.

\textbf{Byte-Anzahl berechnen:}

\begin{lstlisting}[style=arduino,numbers=none]
bytes = (count + 7) / 8
\end{lstlisting}

\begin{table}[htbp]
\centering
\begin{tabular}{@{}llr@{}}
\toprule
\textbf{Kanäle} & \textbf{Rechnung} & \textbf{Bytes} \\
\midrule
8   & (8+7)/8 = 1    & 1 \\
10  & (10+7)/8 = 2   & 2 \\
16  & (16+7)/8 = 2   & 2 \\
100 & (100+7)/8 = 13 & 13 \\
\bottomrule
\end{tabular}
\caption{Byte-Anzahl für verschiedene Kanalzahlen}
\label{tab:byte-anzahl}
\end{table}

\textbf{Konsequenz:} Die Logik kennt nur IDs. Die Treiber kennen nur Bytes. Das Bit-Mapping ist die Brücke dazwischen.

% -----------------------------------------------------------------------------
\subsection{MSB-first – Warum Bit 7 zuerst?}

\textbf{Problem:} Bei SPI-Kommunikation gibt es zwei Konventionen: MSB-first (Bit 7 zuerst) oder LSB-first (Bit 0 zuerst). Die Wahl beeinflusst das gesamte Mapping.

\textbf{Entscheidung im Projekt:} MSB-first für beide Schieberegister-Typen.

\textbf{Warum MSB-first?} Wenn wir die Bytes in der Reihenfolge senden, wie sie im Array stehen, und MSB-first verwenden, ergibt sich eine intuitive Zuordnung:

\begin{lstlisting}[style=wrap,numbers=none]
Array:     [Byte 0] [Byte 1] [Byte 2] ...
Bits:      76543210 76543210 76543210 ...
IDs:       12345678 9...16   17...24  ...
\end{lstlisting}

Die erste ID (1) landet im höchstwertigen Bit (7) des ersten Bytes (0). Das ist leicht nachvollziehbar und debugbar.

\textbf{Mapping-Formel (für ID 1..N):}

\begin{lstlisting}[style=arduino]
byte_index = (id - 1) / 8
bit_position = 7 - ((id - 1) % 8)
mask = 1 << bit_position
\end{lstlisting}

\textbf{Herleitung am Beispiel:}

Für ID = 7:
\begin{itemize}
\item \code{byte\_index = (7-1) / 8 = 0} → Byte 0
\item \code{bit\_position = 7 - ((7-1) \% 8) = 7 - 6 = 1} → Bit 1
\item \code{mask = 1 << 1 = 0x02}
\end{itemize}

Für ID = 9:
\begin{itemize}
\item \code{byte\_index = (9-1) / 8 = 1} → Byte 1
\item \code{bit\_position = 7 - ((9-1) \% 8) = 7 - 0 = 7} → Bit 7
\item \code{mask = 1 << 7 = 0x80}
\end{itemize}

% -----------------------------------------------------------------------------
\subsection{Buttons (CD4021) – Active-Low lesen}

\textbf{Funktionsweise:} Der CD4021 ist ein Parallel-zu-Seriell-Wandler. Er liest 8 parallele Eingänge und schiebt sie seriell zum ESP32.

\textbf{Active-Low bedeutet:}

\begin{table}[htbp]
\centering
\begin{tabular}{@{}lll@{}}
\toprule
\textbf{Physischer Zustand} & \textbf{Pegel am Pin} & \textbf{Bit-Wert} \\
\midrule
Button losgelassen & HIGH (Pullup)  & \code{1} \\
Button gedrückt    & LOW (nach GND) & \code{0} \\
\bottomrule
\end{tabular}
\caption{Active-Low-Logik beim CD4021}
\label{tab:active-low}
\end{table}

\begin{infobox}[Idle-Erwartung]
Wenn kein Button gedrückt ist, liest du \code{0xFF} pro Byte (alle Bits = 1).
\end{infobox}

\textbf{Mapping-Tabelle (10 Buttons):}

\begin{table}[htbp]
\centering
\begin{tabular}{@{}rrrll@{}}
\toprule
\textbf{ID} & \textbf{Byte} & \textbf{Bit} & \textbf{Maske} & \textbf{Gedrückt =} \\
\midrule
1  & 0 & 7 & \code{0x80} & Bit 7 = 0 \\
2  & 0 & 6 & \code{0x40} & Bit 6 = 0 \\
3  & 0 & 5 & \code{0x20} & Bit 5 = 0 \\
4  & 0 & 4 & \code{0x10} & Bit 4 = 0 \\
5  & 0 & 3 & \code{0x08} & Bit 3 = 0 \\
6  & 0 & 2 & \code{0x04} & Bit 2 = 0 \\
7  & 0 & 1 & \code{0x02} & Bit 1 = 0 \\
8  & 0 & 0 & \code{0x01} & Bit 0 = 0 \\
9  & 1 & 7 & \code{0x80} & Bit 7 = 0 \\
10 & 1 & 6 & \code{0x40} & Bit 6 = 0 \\
\bottomrule
\end{tabular}
\caption{Bit-Mapping für 10 Buttons (MSB-first)}
\label{tab:button-mapping}
\end{table}

\textbf{Beispiel – Button 3 gedrückt:}

\begin{lstlisting}[style=arduino,numbers=none]
Byte 0: 0b11011111 = 0xDF  (Bit 5 = 0)
Byte 1: 0b11111111 = 0xFF  (keine Aenderung)
\end{lstlisting}

\textbf{Prüfung im Code:}

\begin{lstlisting}[style=arduino]
bool isPressed(uint8_t id, const uint8_t* raw) {
    uint8_t byteIdx = (id - 1) / 8;
    uint8_t bitPos  = 7 - ((id - 1) % 8);
    uint8_t mask    = 1 << bitPos;

    // Active-low: gedrueckt wenn Bit = 0
    return (raw[byteIdx] & mask) == 0;
}
\end{lstlisting}

% -----------------------------------------------------------------------------
\subsection{LEDs (74HC595) – Active-High schreiben}

\textbf{Funktionsweise:} Der 74HC595 ist ein Seriell-zu-Parallel-Wandler. Er empfängt Daten seriell vom ESP32 und gibt sie parallel an 8 Ausgänge aus.

\textbf{Active-High bedeutet:}

\begin{table}[htbp]
\centering
\begin{tabular}{@{}lll@{}}
\toprule
\textbf{Gewünschter Zustand} & \textbf{Bit-Wert} & \textbf{Ergebnis} \\
\midrule
LED aus & \code{0} & Ausgang LOW \\
LED an  & \code{1} & Ausgang HIGH \\
\bottomrule
\end{tabular}
\caption{Active-High-Logik beim 74HC595}
\label{tab:active-high}
\end{table}

\textbf{Mapping-Tabelle (identisch zu Buttons):}

\begin{table}[htbp]
\centering
\begin{tabular}{@{}rrrll@{}}
\toprule
\textbf{ID} & \textbf{Byte} & \textbf{Bit} & \textbf{Maske} & \textbf{LED an =} \\
\midrule
1  & 0 & 7 & \code{0x80} & \code{0x80 0x00} \\
2  & 0 & 6 & \code{0x40} & \code{0x40 0x00} \\
\ldots & \ldots & \ldots & \ldots & \ldots \\
8  & 0 & 0 & \code{0x01} & \code{0x01 0x00} \\
9  & 1 & 7 & \code{0x80} & \code{0x00 0x80} \\
10 & 1 & 6 & \code{0x40} & \code{0x00 0x40} \\
\bottomrule
\end{tabular}
\caption{Bit-Mapping für 10 LEDs (MSB-first)}
\label{tab:led-mapping}
\end{table}

\textbf{One-Hot-Ausgabe:}

\begin{lstlisting}[style=arduino]
void setOneHot(uint8_t selectedId, uint8_t* ledState, size_t bytes) {
    // 1. Alle LEDs aus
    memset(ledState, 0x00, bytes);

    // 2. Genau eine LED setzen (wenn ID > 0)
    if (selectedId >= 1 && selectedId <= LED_COUNT) {
        uint8_t byteIdx = (selectedId - 1) / 8;
        uint8_t bitPos  = 7 - ((selectedId - 1) % 8);
        ledState[byteIdx] |= (1 << bitPos);
    }
}
\end{lstlisting}

\textbf{Beispiel – LED 9 an:}

\begin{lstlisting}[style=arduino,numbers=none]
Byte 0: 0b00000000 = 0x00
Byte 1: 0b10000000 = 0x80  (Bit 7 = 1)
\end{lstlisting}

% -----------------------------------------------------------------------------
\subsection{Last-Byte-Maskierung – Ungültige Bits behandeln}

\textbf{Problem:} Bei 10 Kanälen brauchst du 2 Bytes (16 Bits), aber nur 10 sind gültig. Die Bits 5–0 im zweiten Byte sind \glqq überzählig\grqq. Wenn sie durch Speicherfehler oder Bugs gesetzt werden, könnte das System falsch reagieren.

\textbf{Regel:} Ungültige Bits müssen maskiert werden – bei Buttons auf 1 (idle), bei LEDs auf 0 (aus).

\textbf{Berechnung der Maske:}

\begin{lstlisting}[style=arduino]
uint8_t valid_bits = count % 8;
if (valid_bits == 0) valid_bits = 8;  // Volle 8 Bits

// MSB-first: die obersten valid_bits sind gueltig
uint8_t LAST_MASK = 0xFF << (8 - valid_bits);
\end{lstlisting}

\textbf{Beispiele:}

\begin{table}[htbp]
\centering
\begin{tabular}{@{}rrlll@{}}
\toprule
\textbf{Count} & \textbf{valid\_bits} & \textbf{LAST\_MASK} & \textbf{Binär} & \textbf{Gültige Bits} \\
\midrule
8   & 8 & \code{0xFF} & \code{11111111} & alle \\
10  & 2 & \code{0xC0} & \code{11000000} & Bit 7, 6 \\
12  & 4 & \code{0xF0} & \code{11110000} & Bit 7, 6, 5, 4 \\
100 & 4 & \code{0xF0} & \code{11110000} & Bit 7, 6, 5, 4 \\
\bottomrule
\end{tabular}
\caption{Last-Byte-Masken für verschiedene Kanalzahlen}
\label{tab:last-mask}
\end{table}

\textbf{Anwendung:}

\begin{lstlisting}[style=arduino]
// Nach dem Lesen (Buttons): ungueltige Bits auf 1 setzen
raw[last_byte] |= ~BTN_LAST_MASK;

// Vor dem Schreiben (LEDs): ungueltige Bits auf 0 setzen
led[last_byte] &= LED_LAST_MASK;
\end{lstlisting}

\begin{warnbox}[Warum ist das wichtig?]
Bei 10 LEDs und \code{LAST\_MASK = 0xC0}: Wenn \code{led[1] = 0xFF} (durch Bug), würden nach Maskierung nur \code{led[1] = 0xC0} an den HC595 gehen. Ohne Maskierung: 6 \glqq Phantom-LEDs\grqq{} würden leuchten.
\end{warnbox}

% -----------------------------------------------------------------------------
\subsection{Kaskadierung – Mehrere ICs verketten}

\textbf{Problem:} Ein CD4021 oder 74HC595 hat nur 8 Kanäle. Für 10, 50 oder 100 Kanäle müssen mehrere ICs in Reihe geschaltet werden.

\paragraph{CD4021-Kette (Input)}

\begin{figure}[htbp]
\centering
\begin{tikzpicture}[
    ic/.style={draw, fill=CodeBg, minimum width=2.5cm, minimum height=1.5cm,
               font=\small, rounded corners=2pt, align=center},
    arr/.style={-{Stealth[length=2mm]}, thick, color=ArduinoDark},
    label/.style={font=\footnotesize\itshape, color=MutedText}
]
\node[ic] (ic1) at (0,0) {CD4021\\IC 1\\(ID 9–16)};
\node[ic] (ic0) at (4,0) {CD4021\\IC 0\\(ID 1–8)};
\node[ic] (esp) at (8,0) {ESP32\\MISO};

\draw[arr] (ic1.east) -- node[above, label] {Q} (ic0.west);
\draw[arr] (ic0.east) -- node[above, label] {Q} (esp.west);

\node[label, below=0.3cm of ic1] {Buttons 9–16};
\node[label, below=0.3cm of ic0] {Buttons 1–8};
\node[label, below=0.3cm of esp] {SCK, PS};
\end{tikzpicture}
\caption{CD4021-Kaskadierung für Button-Input}
\label{fig:cd4021-kette}
\end{figure}

\textbf{Reihenfolge:} Das erste gelesene Bit kommt vom IC, der direkt am ESP32 hängt (IC 0). Die Daten der kaskadierten ICs werden \glqq durchgeschoben\grqq.

\paragraph{74HC595-Kette (Output)}

\begin{figure}[htbp]
\centering
\begin{tikzpicture}[
    ic/.style={draw, fill=CodeBg, minimum width=2.5cm, minimum height=1.5cm,
               font=\small, rounded corners=2pt, align=center},
    arr/.style={-{Stealth[length=2mm]}, thick, color=ArduinoDark},
    label/.style={font=\footnotesize\itshape, color=MutedText}
]
\node[ic] (esp) at (0,0) {ESP32\\MOSI};
\node[ic] (ic0) at (4,0) {HC595\\IC 0\\(ID 1–8)};
\node[ic] (ic1) at (8,0) {HC595\\IC 1\\(ID 9–16)};

\draw[arr] (esp.east) -- (ic0.west);
\draw[arr] (ic0.east) -- node[above, label] {Q7'} (ic1.west);

\node[label, below=0.3cm of ic0] {LED 1–8};
\node[label, below=0.3cm of ic1] {LED 9–16};
\end{tikzpicture}
\caption{74HC595-Kaskadierung für LED-Output}
\label{fig:hc595-kette}
\end{figure}

\textbf{Reihenfolge:} Die zuerst gesendeten Bits landen im hintersten IC der Kette. Byte 0 geht an IC 0, Byte 1 an IC 1 usw.

\begin{warnbox}[Typischer Fehler]
Wenn die ICs falsch herum verkabelt sind, erscheinen die IDs \glqq gespiegelt\grqq{} – Button 1 löst LED 8 aus, Button 9 löst LED 16 aus.
\end{warnbox}

% -----------------------------------------------------------------------------
\subsection{Vollständiges Beispiel – Button 7 drücken}

Wir verfolgen den Datenfluss, wenn Button 7 gedrückt wird:

\textbf{1. Hardware-Zustand:}
\begin{lstlisting}[style=wrap,numbers=none]
Button 7 zieht Pin auf GND -> Bit 1 von Byte 0 wird 0
\end{lstlisting}

\textbf{2. CD4021 liest (2 Bytes):}
\begin{lstlisting}[style=arduino,numbers=none]
raw[0] = 0b11111101 = 0xFD  (Bit 1 = 0)
raw[1] = 0b11111111 = 0xFF
\end{lstlisting}

\textbf{3. Logik erkennt Button 7:}
\begin{lstlisting}[style=arduino]
id = 7
byteIdx = (7-1)/8 = 0
bitPos  = 7 - ((7-1) % 8) = 7 - 6 = 1
mask    = 0x02

// Pruefung: (raw[0] & 0x02) == 0?
// -> (0xFD & 0x02) = 0x00 -> JA, gedrueckt!
\end{lstlisting}

\textbf{4. Selection-Logik setzt} \code{selectedId = 7}

\textbf{5. One-Hot erzeugt LED-Bytes:}
\begin{lstlisting}[style=arduino]
led[0] = 0x00;  // clear
led[1] = 0x00;

// Set Bit fuer ID 7:
byteIdx = 0
bitPos  = 1
led[0] |= (1 << 1) = 0x02

// Ergebnis:
led[0] = 0b00000010 = 0x02
led[1] = 0b00000000 = 0x00
\end{lstlisting}

\textbf{6. HC595 schreibt:}
\begin{lstlisting}[style=wrap,numbers=none]
IC 0 erhaelt 0x02 -> Ausgang Q1 = HIGH -> LED 7 leuchtet
IC 1 erhaelt 0x00 -> alle Ausgaenge LOW
\end{lstlisting}

% -----------------------------------------------------------------------------
\subsection{Skalierung – 10 → 100 ohne Code-Änderung}

\textbf{Änderungen in} \code{io\_counts.h}:

\begin{lstlisting}[style=arduino]
// Vorher (10 Kanaele)
#define BTN_COUNT  10
#define LED_COUNT  10

// Nachher (100 Kanaele)
#define BTN_COUNT  100
#define LED_COUNT  100
\end{lstlisting}

\textbf{Automatisch abgeleitete Werte:}

\begin{table}[htbp]
\centering
\begin{tabular}{@{}lrr@{}}
\toprule
\textbf{Parameter} & \textbf{10 Kanäle} & \textbf{100 Kanäle} \\
\midrule
\code{BTN\_BYTES}     & 2  & 13 \\
\code{LED\_BYTES}     & 2  & 13 \\
\code{BTN\_LAST\_MASK} & \code{0xC0} & \code{0xF0} \\
\code{LED\_LAST\_MASK} & \code{0xC0} & \code{0xF0} \\
\bottomrule
\end{tabular}
\caption{Automatisch abgeleitete Parameter bei Skalierung}
\label{tab:skalierung-bit-mapping}
\end{table}

\textbf{Keine Änderungen nötig in:}
\begin{itemize}
\item Bit-Mapping-Formeln (arbeiten mit beliebigen IDs)
\item Treiber-Code (iteriert über \code{*\_BYTES})
\item Logik-Code (arbeitet mit IDs, nicht Bytes)
\end{itemize}

\begin{tipbox}[Konsequenz]
Die Architektur ist skalierbar, solange die SSOT-Werte angepasst werden.
\end{tipbox}

% -----------------------------------------------------------------------------
\subsection{Fehlerbehebung}

\begin{table}[htbp]
\centering
\begin{tabularx}{\textwidth}{@{}lXl@{}}
\toprule
\textbf{Symptom} & \textbf{Wahrscheinliche Ursache} & \textbf{Diagnose} \\
\midrule
RAW = 0xFF (immer) & Pullups fehlen oder PS/LOAD-Signal falsch & Oszilloskop an PS/LOAD \\
RAW = 0x00 (immer) & Kurzschluss nach GND & Durchgangsprüfung \\
Falsche ID erkannt & MSB/LSB vertauscht oder Kaskadierung falsch & E2E-Test mit einzelnen Buttons \\
LEDs spiegelverkehrt & Byte-Reihenfolge beim HC595 falsch & Verkabelung Q7' prüfen \\
Letzte IDs reagieren nicht & LAST\_MASK fehlt oder falsch & Masken-Berechnung prüfen \\
\bottomrule
\end{tabularx}
\caption{Fehlerbehebung Schieberegister}
\label{tab:fehlersuche}
\end{table}

\textbf{Diagnose-Befehl:}

\begin{lstlisting}[style=shell]
pio test -d firmware -e seeed_xiao_esp32s3_e2e -f test_e2e_hw -v
\end{lstlisting}

% -----------------------------------------------------------------------------
\subsection{Weiterführende Dokumentation}

\begin{description}
\item[\code{firmware/include/hw/io\_counts.h}] SSOT für Counts/Masks
\item[\code{firmware/docs/wiring.md}] Hardware-Verdrahtung
\item[\code{firmware/docs/architecture.md}] Schichtenmodell
\item[\code{firmware/docs/debug-playbook.md}] Erweiterte Fehlerbehebung
\end{description}


\end{document}
