% =============================================================================
% main.tex – Technische Dokumentation
% =============================================================================
% Build-Standard: ueber "make" aus ./latex/.
% Warum: TEXINPUTS wird im Makefile gesetzt, damit Builds aus Terminal/IDE/CI
% reproduzierbar funktionieren.
% =============================================================================
\documentclass{techdoc}
% \documentclass[print]{techdoc} % Warum: Druckversion ohne Link-Dekoration.
% \documentclass[draft]{techdoc} % Warum: Layout-Probleme/TODOs frueh sichtbar machen.

% =============================================================================
% METADATEN (SSOT: alles an einem Ort, damit Titel/Version nicht divergieren)
% =============================================================================
\title{Titel}
\subtitle{Untertitel}
\tagline{Kurzbeschreibung}
\scope{Anwendungsbereich}
\stack{Keyword 1 \textbullet\ Keyword 2 \textbullet\ Keyword 3}
\version{1.0.0}
\date{11.01.2026}
\author{Jan Unger}
\credits{Claude 4.5, ChatGPT 5.2}

% Bilder liegen in ./latex/images/
\graphicspath{{images/}}

\setheadertitle{Kurztitel}

% Warum hier: PDF-Metadaten bleiben gebuendelt und konsistent.
\hypersetup{
  pdftitle={Titel},
  pdfauthor={Jan Unger}
}

% =============================================================================
% DOKUMENT
% =============================================================================
\begin{document}

\maketitlepage

\pagenumbering{roman}
\tableofcontents
\clearpage
\pagenumbering{arabic}

% =============================================================================
% INHALT (SSOT: Inhalt liegt in content/, main bleibt schlank)
% =============================================================================
%% =============================================================================
% content/kapitel.tex – Beispiel-Inhalt
% =============================================================================

\section{Einleitung}

Beispieltext für die Dokumentation. Dieser Abschnitt demonstriert die verfügbaren Elemente, \zB Boxen, Code und Tabellen.\todo{Einleitung erweitern}

\subsection{Text-Hervorhebungen}

Inline-Code: \code{funktionsName()}. 
Wichtige Begriffe \highlight{hervorheben}, \warning{Warnungen} oder \critical{kritische Hinweise} markieren.

Abkürzungen: \zB \dhei \ua \bzw \ggf \evtl

\subsection{Info-Boxen}

\begin{infobox}[Hinweis]
  Allgemeine Information mit \code{cleveref}: Siehe \cref{tab:beispiel}.
\end{infobox}

\begin{warnbox}[Achtung]
  Warnhinweis – wichtig bei Spannungspegeln.
\end{warnbox}

\begin{tipbox}[Tipp]
  Best Practice für bessere Performance.
\end{tipbox}

\section{Code-Beispiele}

Code wird in \code{footnotesize} ohne automatischen Zeilenumbruch dargestellt.
Für lange Ausgaben gibt es den Style \code{wrap}.

\begin{lstlisting}[style=arduino, caption={Arduino/C++ Beispiel}]
void setup() {
  pinMode(LED_BUILTIN, OUTPUT);
  Serial.begin(115200);
}

void loop() {
  digitalWrite(LED_BUILTIN, HIGH);
  delay(1000);
}
\end{lstlisting}

\begin{lstlisting}[style=python, caption={Python Beispiel}]
async def main():
    await asyncio.sleep(1)
    return True
\end{lstlisting}

\begin{lstlisting}[style=shell, caption={Shell Beispiel}]
sudo apt update && sudo apt upgrade -y
\end{lstlisting}

\begin{lstlisting}[style=json, caption={JSON Beispiel}]
{
  "name": "sensor",
  "value": 42,
  "active": true
}
\end{lstlisting}

\section{Tabellen}

\begin{table}[H]
  \centering
  \caption{Beispiel-Tabelle}
  \label{tab:beispiel}
  \begin{tabular}{lll}
    \toprule
    Spalte A & Spalte B & Spalte C \\
    \midrule
    Wert 1   & Wert 2   & Wert 3 \\
    Wert 4   & Wert 5   & Wert 6 \\
    \bottomrule
  \end{tabular}
\end{table}

\section{Einheiten}

Spannung: \SI{3.3}{\volt}, Strom: \SI{20}{\milli\ampere}, Frequenz: \SI{10}{\mega\hertz}, Widerstand: \SI{4.7}{\kilo\ohm}.


\end{document}
